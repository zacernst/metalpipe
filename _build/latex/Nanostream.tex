%% Generated by Sphinx.
\def\sphinxdocclass{report}
\documentclass[letterpaper,10pt,english]{sphinxmanual}
\ifdefined\pdfpxdimen
   \let\sphinxpxdimen\pdfpxdimen\else\newdimen\sphinxpxdimen
\fi \sphinxpxdimen=.75bp\relax

\PassOptionsToPackage{warn}{textcomp}
\usepackage[utf8]{inputenc}
\ifdefined\DeclareUnicodeCharacter
% support both utf8 and utf8x syntaxes
\edef\sphinxdqmaybe{\ifdefined\DeclareUnicodeCharacterAsOptional\string"\fi}
  \DeclareUnicodeCharacter{\sphinxdqmaybe00A0}{\nobreakspace}
  \DeclareUnicodeCharacter{\sphinxdqmaybe2500}{\sphinxunichar{2500}}
  \DeclareUnicodeCharacter{\sphinxdqmaybe2502}{\sphinxunichar{2502}}
  \DeclareUnicodeCharacter{\sphinxdqmaybe2514}{\sphinxunichar{2514}}
  \DeclareUnicodeCharacter{\sphinxdqmaybe251C}{\sphinxunichar{251C}}
  \DeclareUnicodeCharacter{\sphinxdqmaybe2572}{\textbackslash}
\fi
\usepackage{cmap}
\usepackage[T1]{fontenc}
\usepackage{amsmath,amssymb,amstext}
\usepackage{babel}
\usepackage{times}
\usepackage[Bjarne]{fncychap}
\usepackage{sphinx}

\fvset{fontsize=\small}
\usepackage{geometry}

% Include hyperref last.
\usepackage{hyperref}
% Fix anchor placement for figures with captions.
\usepackage{hypcap}% it must be loaded after hyperref.
% Set up styles of URL: it should be placed after hyperref.
\urlstyle{same}
\addto\captionsenglish{\renewcommand{\contentsname}{Contents:}}

\addto\captionsenglish{\renewcommand{\figurename}{Fig.}}
\addto\captionsenglish{\renewcommand{\tablename}{Table}}
\addto\captionsenglish{\renewcommand{\literalblockname}{Listing}}

\addto\captionsenglish{\renewcommand{\literalblockcontinuedname}{continued from previous page}}
\addto\captionsenglish{\renewcommand{\literalblockcontinuesname}{continues on next page}}
\addto\captionsenglish{\renewcommand{\sphinxnonalphabeticalgroupname}{Non-alphabetical}}
\addto\captionsenglish{\renewcommand{\sphinxsymbolsname}{Symbols}}
\addto\captionsenglish{\renewcommand{\sphinxnumbersname}{Numbers}}

\addto\extrasenglish{\def\pageautorefname{page}}

\setcounter{tocdepth}{1}



\title{Nanostream Documentation}
\date{Feb 21, 2019}
\release{0.1}
\author{Zachary Ernst}
\newcommand{\sphinxlogo}{\vbox{}}
\renewcommand{\releasename}{Release}
\makeindex
\begin{document}

\pagestyle{empty}
\maketitle
\pagestyle{plain}
\sphinxtableofcontents
\pagestyle{normal}
\phantomsection\label{\detokenize{index::doc}}



\chapter{Overview}
\label{\detokenize{overview:overview}}\label{\detokenize{overview::doc}}
NanoStream is a package of classes and functions that help you write consistent, efficient, configuration-driven ETL pipelines in Python. It is open-source and
as simple as possible (but not simpler).

This overview tells you why NanoStream exists, and how it can help you escape from ETL hell.


\section{Why is it?}
\label{\detokenize{overview:why-is-it}}
Tolstoy said that every happy family is the same, but every unhappy family is
unhappy in its own way. ETL pipelines are unhappy families.

Why are they so unhappy? Every engineer who does more than one project involving
ETL eventually goes through the same stages of ETL grief. First, they think it’s
not so bad. Then they do another project and discover that they have to rewrite
very similar code. Then they think, “Surely, I could have just written a few
library functions and reused that code, saving lots of time.” But when they try
to do this, they discover that although their ETL projects are very similar,
they are just different enough that their code isn’t reusable. So they resign
themselves to rewriting code over and over again. The code is unreliable,
difficult to maintain, and usually poorly tested and documented because it’s
such a pain to write in the first place. The task of writing ETL pipelines is
so lousy that engineering best practices tend to go out the window because
the engineer has better things to do.


\section{What is it?}
\label{\detokenize{overview:what-is-it}}
NanoStream is an ETL framework for the real world. It aims to provide structure and consistency to your ETL pipelines, while still allowing you to write bespoke code for all of the weird little idiosyncratic features of your data. It is opinionated without being bossy.

The overall idea of NanoStream is simple. On the surface, it looks a lot like streaming frameworks such as Spark or Storm. You hook up various tasks in a directed graph called a “pipeline”. The pipeline ingests data from or more places, transforms it, and loads the data somewhere else. But it differs from Spark-like systems in important ways:
\begin{enumerate}
\def\theenumi{\arabic{enumi}}
\def\labelenumi{\theenumi .}
\makeatletter\def\p@enumii{\p@enumi \theenumi .}\makeatother
\item {} 
It is agnostic between stream and batch. Batches of data can be turned into streams and vice-versa.

\item {} 
It is lightweight, requiring no specialized infrastructure or network configuration.

\item {} 
Its built-in functionality is specifically designed for ETL tasks.

\item {} 
It is meant to accommodate 90\% of your ETL needs entirely by writing configuration files.

\end{enumerate}


\section{What isn’t it?}
\label{\detokenize{overview:what-isn-t-it}}
There are many things that NanoStream is not:
\begin{enumerate}
\def\theenumi{\arabic{enumi}}
\def\labelenumi{\theenumi .}
\makeatletter\def\p@enumii{\p@enumi \theenumi .}\makeatother
\item {} 
It is not a Big Data(tm) tool. If you’re handling petabytes of data, you do
not want to use NanoStream.

\item {} 
It is not suitable for large amounts of computation. If you need to use
dataframes to calculate lots of complex statistical information in real-time,
this is not the tool for you.

\end{enumerate}

Basically, NanoStream deliberately makes two trade-offs: (1) it gives up Big Data(tm) for simplicity; and (2) it gives up being a general-purpose analytic tool in favor of being very good at ETL.


\section{NanoStream pipelines}
\label{\detokenize{overview:nanostream-pipelines}}
An ETL pipeline in NanoStream is a series of nodes connected by queues. Data
is generated or processed in each node, and the output is placed on a queue to
be picked up by downstream nodes.

\begin{figure}[htbp]
\centering
\capstart

\noindent\sphinxincludegraphics[width=600\sphinxpxdimen]{{30k_view}.png}
\caption{Very high-level view of a NanoStream pipeline}\label{\detokenize{overview:id1}}\end{figure}

For the sake of convenience, we distinguish between three types of nodes
(although there’s no real difference in their use or implementation):
\begin{enumerate}
\def\theenumi{\arabic{enumi}}
\def\labelenumi{\theenumi .}
\makeatletter\def\p@enumii{\p@enumi \theenumi .}\makeatother
\item {} 
Source nodes. These are nodes that generate data and send it to the rest
of the pipeline. They might, for example, read data from an external
data source such as an API endpoint or a database.

\item {} 
Worker nodes. The workers process data by picking up messages from their
incoming queues. Their output is placed onto any number of outgoing queues
to be further processed by downstream nodes.

\item {} 
Sink nodes. These are worker nodes with no outgoing queue. They will
typically perform tasks such as inserting data into a database or generating
statistics to be sent somewhere outside the pipeline.

\end{enumerate}

All pipelines are implemented in pure Python (version \textgreater{}=3.5). Each node is
instantiated from a class that inherits from the \sphinxcode{\sphinxupquote{NanoNode}} class. Queues
are never instantiated directly by the user; they are created automatically
whenever two nodes are linked together.

There is a large (and growing) number of specialized \sphinxcode{\sphinxupquote{NanoNode}} subclasses,
each geared toward a specific task. Such tasks include:
\begin{enumerate}
\def\theenumi{\arabic{enumi}}
\def\labelenumi{\theenumi .}
\makeatletter\def\p@enumii{\p@enumi \theenumi .}\makeatother
\item {} 
Querying a table in a SQL database and sending the results downstream.

\item {} 
Making a request to a REST API, paging through the responses until there are
no more results.

\item {} 
Ingesting individual messages from an upstream node and batching them
together into a single message, or doing the reverse.

\item {} 
Reading environment variables.

\item {} 
Watching a directory for new files and sending the names of those files
down the pipeline when they appear.

\item {} 
Filtering messages, letting them through the pipeline only if a particular
test is passed.

\end{enumerate}

\begin{figure}[htbp]
\centering
\capstart

\noindent\sphinxincludegraphics[width=600\sphinxpxdimen]{{10k_view}.png}
\caption{Somewhat high-level view of a NanoStream pipeline}\label{\detokenize{overview:id2}}\end{figure}

All results and messages passed among the nodes must be dictionary-like
objects. By default, messages retain any keys and values that were created by upstream nodes.

The goal is for NanoStream to be “batteries included”, with built-in
\sphinxcode{\sphinxupquote{NanoNode}} subclasses for every common ETL task. But because ETL pipelines
generally have something weird going on somewhere, NanoNode makes it easy to
roll your own node classes;


\section{Installing NanoStream}
\label{\detokenize{overview:installing-nanostream}}
NanoStream is installed in the usual way, with pip:

\fvset{hllines={, ,}}%
\begin{sphinxVerbatim}[commandchars=\\\{\}]
\PYG{n}{pip} \PYG{n}{install} \PYG{n}{nanostream}
\end{sphinxVerbatim}

To test your installation, try typing

\fvset{hllines={, ,}}%
\begin{sphinxVerbatim}[commandchars=\\\{\}]
\PYG{n}{nanostream} \PYG{o}{\PYGZhy{}}\PYG{o}{\PYGZhy{}}\PYG{n}{help}
\end{sphinxVerbatim}

If NanoStream is installed correctly, you should see a help message.


\section{Using NanoStream}
\label{\detokenize{overview:using-nanostream}}
You use NanoStream by (1) writing a configuration file that describes your pipeline, and (2) running the \sphinxcode{\sphinxupquote{nanostream}} command, specifying the location of your
configuration file. NanoStream will read the configuration, create the pipeline,
and run it.

The configuration file is written in YAML. It has three parts:
\begin{enumerate}
\def\theenumi{\arabic{enumi}}
\def\labelenumi{\theenumi .}
\makeatletter\def\p@enumii{\p@enumi \theenumi .}\makeatother
\item {} 
A list of global variables (optional)

\item {} 
The nodes and their options (required)

\item {} 
A list of edges connecting those nodes to each other.

\end{enumerate}

This is a simple configuration file. If you want to, you can copy it into a
file called \sphinxcode{\sphinxupquote{sample\_config.yaml}}:

\fvset{hllines={, ,}}%
\begin{sphinxVerbatim}[commandchars=\\\{\}]
\PYG{o}{\PYGZhy{}}\PYG{o}{\PYGZhy{}}\PYG{o}{\PYGZhy{}}
\PYG{n}{pipeline\PYGZus{}name}\PYG{p}{:} \PYG{n}{Sample} \PYG{n}{NanoStream} \PYG{n}{configuration}
\PYG{n}{pipeline\PYGZus{}description}\PYG{p}{:} \PYG{n}{Reads} \PYG{n}{some} \PYG{n}{environment} \PYG{n}{variables} \PYG{o+ow}{and} \PYG{n}{prints} \PYG{n}{them}

\PYG{n}{nodes}\PYG{p}{:}
  \PYG{n}{get\PYGZus{}environment\PYGZus{}variables}\PYG{p}{:}
    \PYG{n}{class}\PYG{p}{:} \PYG{n}{GetEnvironmentVariables}
    \PYG{n}{summary}\PYG{p}{:} \PYG{n}{Gets} \PYG{n+nb}{all} \PYG{n}{the} \PYG{n}{necessary} \PYG{n}{environment} \PYG{n}{variables}
    \PYG{n}{options}\PYG{p}{:}
      \PYG{n}{environment\PYGZus{}variables}\PYG{p}{:}
        \PYG{o}{\PYGZhy{}} \PYG{n}{API\PYGZus{}KEY}
        \PYG{o}{\PYGZhy{}} \PYG{n}{API\PYGZus{}USER\PYGZus{}ID}

  \PYG{n}{print\PYGZus{}variables}\PYG{p}{:}
    \PYG{n}{class}\PYG{p}{:} \PYG{n}{PrinterOfThings}
    \PYG{n}{summary}\PYG{p}{:} \PYG{n}{Prints} \PYG{n}{the} \PYG{n}{environment} \PYG{n}{variables} \PYG{n}{to} \PYG{n}{the} \PYG{n}{terminal}
    \PYG{n}{options}\PYG{p}{:}
      \PYG{n}{prepend}\PYG{p}{:} \PYG{l+s+s2}{\PYGZdq{}}\PYG{l+s+s2}{Environment variables: }\PYG{l+s+s2}{\PYGZdq{}}

\PYG{n}{paths}\PYG{p}{:}
  \PYG{o}{\PYGZhy{}}
    \PYG{o}{\PYGZhy{}} \PYG{n}{get\PYGZus{}environment\PYGZus{}variables}
    \PYG{o}{\PYGZhy{}} \PYG{n}{print\PYGZus{}variables}
\end{sphinxVerbatim}

If you’ve installed NanoStream and copied this configuration into \sphinxcode{\sphinxupquote{sample\_config.yaml}}, then you can execute the pipeline:

\fvset{hllines={, ,}}%
\begin{sphinxVerbatim}[commandchars=\\\{\}]
\PYG{n}{nanostream} \PYG{n}{run} \PYG{o}{\PYGZhy{}}\PYG{o}{\PYGZhy{}}\PYG{n}{filename} \PYG{n}{sample\PYGZus{}config}\PYG{o}{.}\PYG{n}{yaml}
\end{sphinxVerbatim}

The output should look like this (you might also see some log messages):

\fvset{hllines={, ,}}%
\begin{sphinxVerbatim}[commandchars=\\\{\}]
\PYG{n}{Environment} \PYG{n}{variables}\PYG{p}{:}
\PYG{p}{\PYGZob{}}\PYG{l+s+s1}{\PYGZsq{}}\PYG{l+s+s1}{API\PYGZus{}USER\PYGZus{}ID}\PYG{l+s+s1}{\PYGZsq{}}\PYG{p}{:} \PYG{k+kc}{None}\PYG{p}{,} \PYG{l+s+s1}{\PYGZsq{}}\PYG{l+s+s1}{API\PYGZus{}KEY}\PYG{l+s+s1}{\PYGZsq{}}\PYG{p}{:} \PYG{k+kc}{None}\PYG{p}{\PYGZcb{}}
\end{sphinxVerbatim}

The NanoStream pipeline has found the values of two environment variables (\sphinxcode{\sphinxupquote{API\_KEY}} and \sphinxcode{\sphinxupquote{API\_USER\_ID}}) and printed them to the terminal. If those environmet variables have not been set, their values will be \sphinxcode{\sphinxupquote{None}}. But if you were to set any of them, their values would be printed.

Although this is a very trivial example, it is enough to show the main functionality of NanoStream. Let’s look at the configuration file one
part at a time.

The configuration starts with two top-level options, \sphinxcode{\sphinxupquote{pipeline\_name}} and \sphinxcode{\sphinxupquote{pipeline\_description}}. These are optional, and are only used for the user’s convenience.

Below those are two sections: \sphinxcode{\sphinxupquote{nodes}} and \sphinxcode{\sphinxupquote{paths}}. Each \sphinxcode{\sphinxupquote{nodes}} section contains one or more blocks that always have this form:

\fvset{hllines={, ,}}%
\begin{sphinxVerbatim}[commandchars=\\\{\}]
\PYG{n}{do\PYGZus{}something}\PYG{p}{:}
  \PYG{n}{class}\PYG{p}{:} \PYG{n}{node} \PYG{k}{class}
  \PYG{n+nc}{summary}\PYG{p}{:} \PYG{n}{optional} \PYG{n}{string} \PYG{n}{describing} \PYG{n}{what} \PYG{n}{this} \PYG{n}{node} \PYG{n}{does}
  \PYG{n}{options}\PYG{p}{:}
    \PYG{n}{option\PYGZus{}1}\PYG{p}{:} \PYG{n}{value} \PYG{n}{of} \PYG{n}{this} \PYG{n}{option}
    \PYG{n}{option\PYGZus{}2}\PYG{p}{:} \PYG{n}{value} \PYG{n}{of} \PYG{n}{another} \PYG{n}{option}
\end{sphinxVerbatim}

Let’s go through this one line at a time.

Each node block describes a single node in the NanoStream pipeline. A node
must be given a name, which can be any arbitrary string. This should be a
short, descriptive string describing its action, such as \sphinxcode{\sphinxupquote{get\_environment\_variables}} or \sphinxcode{\sphinxupquote{parse\_json}}, for example. We encourage
you to stick to a clear naming convention. We like nodes to have names of
the form \sphinxcode{\sphinxupquote{verb\_noun}} (as in \sphinxcode{\sphinxupquote{print\_name}}).

NanoStream contains a number of node classes, each of which is designed
for a specific type of ETL task. In the sample configuration, we’re used
the built-in classes \sphinxcode{\sphinxupquote{GetEnvironmentVariables}} and \sphinxcode{\sphinxupquote{PrinterOfThings}}; these are the value following \sphinxcode{\sphinxupquote{class}}. You can also roll your own node classes (we’ll describe how to do this later in the documentation).

Next is a set of keys and values for the various options that are supported by that class. Because each node class does something different,
the options are different as well. In the sample configuration, the
\sphinxcode{\sphinxupquote{GetEnvironmentVariables}} node class requires a list of environment variables to retrieve, so as you would expect, we specify that list under the \sphinxcode{\sphinxupquote{environment\_variables}} option. The various options are explained in
the documentation for each class. In addition to the options that are specific to each node, there are also options that are common to every type of node. These will be explained later.

The structure of the pipeline is given in the \sphinxcode{\sphinxupquote{paths}} section, which contains a list of lists. Each list is a set of nodes that are to be linked together in
order. In our example, the \sphinxcode{\sphinxupquote{paths}} value says that
\sphinxcode{\sphinxupquote{get\_environment\_variables}} will send its output to \sphinxcode{\sphinxupquote{print\_variables}}.
Paths can be arbitrarily long.

If you wanted to send the environment variables down two different execution
paths, you add another list to the \sphinxcode{\sphinxupquote{paths}}, like so:

\fvset{hllines={, ,}}%
\begin{sphinxVerbatim}[commandchars=\\\{\}]
\PYG{n}{paths}\PYG{p}{:}
  \PYG{o}{\PYGZhy{}}
    \PYG{o}{\PYGZhy{}} \PYG{n}{get\PYGZus{}environment\PYGZus{}variables}
    \PYG{o}{\PYGZhy{}} \PYG{n}{print\PYGZus{}variables}
  \PYG{o}{\PYGZhy{}}
    \PYG{o}{\PYGZhy{}} \PYG{n}{get\PYGZus{}environment\PYGZus{}variables}
    \PYG{o}{\PYGZhy{}} \PYG{n}{do\PYGZus{}something\PYGZus{}else}
    \PYG{o}{\PYGZhy{}} \PYG{n}{and\PYGZus{}then\PYGZus{}do\PYGZus{}this}
\end{sphinxVerbatim}

With this set of \sphinxcode{\sphinxupquote{paths}}, the pipeline looks like a very simple tree, with
\sphinxcode{\sphinxupquote{get\_environment\_variables}} at the root, which branches to
\sphinxcode{\sphinxupquote{print\_variables}} and \sphinxcode{\sphinxupquote{do\_something\_else}}.

When you have written the configuration file, you’re ready to use the
NanoStream CLI. It accepts a command, followed by some options. As of now, the
commands it accepts are \sphinxcode{\sphinxupquote{run}}, which executes the pipeline, and \sphinxcode{\sphinxupquote{draw}},
which generates a diagram of the pipeline. The relevant command(s) are:

\fvset{hllines={, ,}}%
\begin{sphinxVerbatim}[commandchars=\\\{\}]
\PYG{n}{python} \PYG{n}{nanostream\PYGZus{}cli}\PYG{o}{.}\PYG{n}{py} \PYG{p}{[}\PYG{n}{run} \PYG{o}{\textbar{}} \PYG{n}{draw}\PYG{p}{]} \PYG{o}{\PYGZhy{}}\PYG{o}{\PYGZhy{}}\PYG{n}{filename} \PYG{n}{my\PYGZus{}sample\PYGZus{}config}\PYG{o}{.}\PYG{n}{yaml}
\end{sphinxVerbatim}

The \sphinxcode{\sphinxupquote{nanostream}} command can generate a pdf file containing a drawing of the pipeline, showing the flow of data through the various nodes. Just speciy \sphinxcode{\sphinxupquote{draw}} instead of \sphinxcode{\sphinxupquote{run}} to generate the diagram. For our simple little pipeline, we get this:

\begin{figure}[htbp]
\centering
\capstart

\noindent\sphinxincludegraphics[width=240\sphinxpxdimen]{{sample_config_drawing}.pdf}
\caption{The pipeline drawing for the simple configuration example}\label{\detokenize{overview:id3}}\end{figure}

It is also possible to skip using the configuration file and define your
pipelines directly in code. In general, it’s better to use the configuration
file for a variety of reasons, but you always have the option of doing this
in Python.

Nodes are defined in code by instantiating classes that inherit from
\sphinxcode{\sphinxupquote{NanoNode}}. Upon instantiation, the constructor takes the same set of
keyword arguments as you see in the configuration. Nodes are linked together
by the \sphinxcode{\sphinxupquote{\textgreater{}}} operator, as in \sphinxcode{\sphinxupquote{node\_1 \textgreater{} node\_2}}. After the pipeline has been
built in this way, it is started by calling \sphinxcode{\sphinxupquote{node.global\_start()}} on any
of the nodes in the pipeline.

The code corresponding to the configuration file above would look like this:

\fvset{hllines={, ,}}%
\begin{sphinxVerbatim}[commandchars=\\\{\}]
\PYG{c+c1}{\PYGZsh{} Define the nodes using the various subclasses of NanoNode}
\PYG{n}{get\PYGZus{}environment\PYGZus{}variables} \PYG{o}{=}
\PYG{n}{GetEnvironmentVariables}\PYG{p}{(}
    \PYG{n}{environment\PYGZus{}variables}\PYG{o}{=}\PYG{p}{[}\PYG{l+s+s1}{\PYGZsq{}}\PYG{l+s+s1}{API\PYGZus{}KEY}\PYG{l+s+s1}{\PYGZsq{}}\PYG{p}{,} \PYG{l+s+s1}{\PYGZsq{}}\PYG{l+s+s1}{API\PYGZus{}USER\PYGZus{}ID}\PYG{l+s+s1}{\PYGZsq{}}\PYG{p}{]}\PYG{p}{)}
\PYG{n}{print\PYGZus{}variables} \PYG{o}{=} \PYG{n}{PrinterOfThings}\PYG{p}{(}\PYG{n}{prepend}\PYG{o}{=}\PYG{l+s+s1}{\PYGZsq{}}\PYG{l+s+s1}{Environment variables: }\PYG{l+s+s1}{\PYGZsq{}}\PYG{p}{)}

\PYG{c+c1}{\PYGZsh{} The \PYGZsq{}\PYGZgt{}\PYGZsq{} operator can also be chained, as in:}
\PYG{c+c1}{\PYGZsh{} node\PYGZus{}1 \PYGZgt{} node\PYGZus{}2 \PYGZgt{} node\PYGZus{}3 \PYGZgt{} ...}
\PYG{n}{get\PYGZus{}environment\PYGZus{}variables} \PYG{o}{\PYGZgt{}} \PYG{n}{print\PYGZus{}variables}

\PYG{c+c1}{\PYGZsh{} Run the pipeline. This command will not block.}
\PYG{n}{get\PYGZus{}environment\PYGZus{}variables}\PYG{o}{.}\PYG{n}{global\PYGZus{}start}\PYG{p}{(}\PYG{p}{)}
\end{sphinxVerbatim}


\section{Rolling your own \sphinxstyleliteralintitle{\sphinxupquote{NanoNode}} class}
\label{\detokenize{overview:rolling-your-own-nanonode-class}}
If there are no built-in \sphinxcode{\sphinxupquote{NanoNode}} classes suitable for your ETL pipeline,
it is easy to write your own.

For example, suppose you want to create a source node for your pipeline
that simply emits a user-defined string every few seconds forever. The user
would be able to specify the string and the number of seconds to pause after
each message has been sent. The class could be defined like so:

\fvset{hllines={, ,}}%
\begin{sphinxVerbatim}[commandchars=\\\{\}]
\PYG{k}{class} \PYG{n+nc}{FooEmitter}\PYG{p}{(}\PYG{n}{NanoNode}\PYG{p}{)}\PYG{p}{:}  \PYG{c+c1}{\PYGZsh{} inherit from NanoNode}
    \PYG{l+s+sd}{\PYGZsq{}\PYGZsq{}\PYGZsq{}}
\PYG{l+s+sd}{    Sends {}`{}`self.output\PYGZus{}string{}`{}` every {}`{}`self.interval{}`{}` seconds.}
\PYG{l+s+sd}{    \PYGZsq{}\PYGZsq{}\PYGZsq{}}
    \PYG{k}{def} \PYG{n+nf}{\PYGZus{}\PYGZus{}init\PYGZus{}\PYGZus{}}\PYG{p}{(}\PYG{n+nb+bp}{self}\PYG{p}{,} \PYG{n}{output\PYGZus{}string}\PYG{o}{=}\PYG{l+s+s1}{\PYGZsq{}}\PYG{l+s+s1}{\PYGZsq{}}\PYG{p}{,} \PYG{n}{interval}\PYG{o}{=}\PYG{l+m+mi}{1}\PYG{p}{,} \PYG{o}{*}\PYG{o}{*}\PYG{n}{kwargs}\PYG{p}{)}\PYG{p}{:}
        \PYG{n+nb+bp}{self}\PYG{o}{.}\PYG{n}{output\PYGZus{}string} \PYG{o}{=} \PYG{n}{output\PYGZus{}string}
        \PYG{n+nb+bp}{self}\PYG{o}{.}\PYG{n}{interval} \PYG{o}{=} \PYG{n}{interval}
        \PYG{n+nb}{super}\PYG{p}{(}\PYG{n}{FooEmitter}\PYG{p}{,} \PYG{n+nb+bp}{self}\PYG{p}{)}\PYG{o}{.}\PYG{n+nf+fm}{\PYGZus{}\PYGZus{}init\PYGZus{}\PYGZus{}}\PYG{p}{(}\PYG{p}{)}  \PYG{c+c1}{\PYGZsh{} Must call the {}`NanoNode{}` \PYGZus{}\PYGZus{}init\PYGZus{}\PYGZus{}}

    \PYG{k}{def} \PYG{n+nf}{generator}\PYG{p}{(}\PYG{n+nb+bp}{self}\PYG{p}{)}\PYG{p}{:}
        \PYG{k}{while} \PYG{k+kc}{True}\PYG{p}{:}
            \PYG{n}{time}\PYG{o}{.}\PYG{n}{sleep}\PYG{p}{(}\PYG{n+nb+bp}{self}\PYG{o}{.}\PYG{n}{interval}\PYG{p}{)}
            \PYG{k}{yield} \PYG{n+nb+bp}{self}\PYG{o}{.}\PYG{n}{output\PYGZus{}string}  \PYG{c+c1}{\PYGZsh{} Output must be yielded, not returned}
\end{sphinxVerbatim}

Let’s look at each part of this class.

The first thing to note is that the class inherits from \sphinxcode{\sphinxupquote{NanoNode}} \textendash{} this
is the mix-in class that gives the node all of its functionality within the
NanoStream framework.

The \sphinxcode{\sphinxupquote{\_\_init\_\_}} method should take only keyword arguments, not positional
arguments. This restriction is to guarantee that the configuration files have
names for any options that are specified in the pipeline. In the \sphinxcode{\sphinxupquote{\_\_init\_\_}}
function, you should also be sure to accept \sphinxcode{\sphinxupquote{**kwargs}}, because options that
are common to all \sphinxcode{\sphinxupquote{NanoNode}} objects are expected to be there.

After any attributes have been defined, the \sphinxcode{\sphinxupquote{\_\_init\_\_}} method \sphinxstylestrong{must}
invoke the parent class’s constructor through the use of the \sphinxcode{\sphinxupquote{super}}
function. Be sure to pass the \sphinxcode{\sphinxupquote{**kwargs}} argument into the function as
shown in the example.

If the node class is intended to be used as a source node, then you need to
define a \sphinxcode{\sphinxupquote{generator}} method. This method can be virtually anything, so long
as it sends its output via a \sphinxcode{\sphinxupquote{yield}} statement.

If you need to define a worker node (that is, a node that accepts input
from a queue), you will provide a \sphinxcode{\sphinxupquote{process\_item}} method instead of a
\sphinxcode{\sphinxupquote{generator}}. But the structure of that method is the same, with the single
exception that you will have access to a \sphinxcode{\sphinxupquote{\_\_message\_\_}} attribute which
contains the incoming message data. The structure of a typical \sphinxcode{\sphinxupquote{process\_item}}
method is shown in the figure.

\begin{figure}[htbp]
\centering
\capstart

\noindent\sphinxincludegraphics[width=400\sphinxpxdimen]{{process_item}.png}
\caption{A typical \sphinxcode{\sphinxupquote{process\_item}} method for \sphinxcode{\sphinxupquote{NanoNode}} objects}\label{\detokenize{overview:id4}}\end{figure}

For example, let’s suppose you want to create a node that is passed a string as a
message, and returns \sphinxcode{\sphinxupquote{True}} if the message has an even number of
characters, \sphinxcode{\sphinxupquote{False}} otherwise. The class definition would look like
this:

\fvset{hllines={, ,}}%
\begin{sphinxVerbatim}[commandchars=\\\{\}]
\PYG{k}{class} \PYG{n+nc}{MessageLengthTester}\PYG{p}{(}\PYG{n}{NanoNode}\PYG{p}{)}\PYG{p}{:}
    \PYG{k}{def} \PYG{n+nf}{\PYGZus{}\PYGZus{}init\PYGZus{}\PYGZus{}}\PYG{p}{(}\PYG{n+nb+bp}{self}\PYG{p}{)}\PYG{p}{:}
        \PYG{c+c1}{\PYGZsh{} No particular initialization required in this example}
        \PYG{n+nb}{super}\PYG{p}{(}\PYG{n}{MessageLengthTester}\PYG{p}{,} \PYG{n+nb+bp}{self}\PYG{p}{)}\PYG{o}{.}\PYG{n+nf+fm}{\PYGZus{}\PYGZus{}init\PYGZus{}\PYGZus{}}\PYG{p}{(}\PYG{p}{)}

    \PYG{k}{def} \PYG{n+nf}{process\PYGZus{}item}\PYG{p}{(}\PYG{n+nb+bp}{self}\PYG{p}{)}\PYG{p}{:}
        \PYG{k}{if} \PYG{n+nb}{len}\PYG{p}{(}\PYG{n+nb+bp}{self}\PYG{o}{.}\PYG{n}{\PYGZus{}\PYGZus{}message\PYGZus{}\PYGZus{}}\PYG{p}{)} \PYG{o}{\PYGZpc{}} \PYG{l+m+mi}{2} \PYG{o}{==} \PYG{l+m+mi}{0}\PYG{p}{:}
            \PYG{k}{yield} \PYG{k+kc}{True}
        \PYG{k}{else}\PYG{p}{:}
            \PYG{k}{yield} \PYG{k+kc}{False}
\end{sphinxVerbatim}


\section{Composing and configuring \sphinxstyleliteralintitle{\sphinxupquote{NanoNode}} objects}
\label{\detokenize{overview:composing-and-configuring-nanonode-objects}}
\begin{sphinxadmonition}{warning}{Warning:}
The code described in this section is experimental and very
unstable. It would be bad to use it for anything important.
\end{sphinxadmonition}

Let’s suppose you’ve worked very hard to create the pipeline from the
last example. Now, your boss says that another engineering team wants to
use it, but they want to rename parameters and “freeze” the values of
certain other parameters to specific values. Once that’s done, they want
to use it as just one part of a more complicated \sphinxcode{\sphinxupquote{NanoStream}}
pipeline.

This can be accomplished using a configuration file. When \sphinxcode{\sphinxupquote{NanoStream}}
parses the configuration file, it will dynamically create the desired
class, which can be instantiated and used as if it were a single node in
another pipeline.

The configuration file is written in YAML, and it would look like this:

\fvset{hllines={, ,}}%
\begin{sphinxVerbatim}[commandchars=\\\{\}]
\PYG{n}{name}\PYG{p}{:} \PYG{n}{FooMessageTester}

\PYG{n}{nodes}\PYG{p}{:}
  \PYG{o}{\PYGZhy{}} \PYG{n}{name}\PYG{p}{:} \PYG{n}{foo\PYGZus{}generator}
    \PYG{k}{class} \PYG{n+nc}{FooEmitter}
    \PYG{n}{frozen\PYGZus{}arguments}\PYG{p}{:}
      \PYG{n}{message}\PYG{p}{:} \PYG{n}{foobar}
    \PYG{n}{arg\PYGZus{}mapping}\PYG{p}{:}
      \PYG{n}{interval}\PYG{p}{:} \PYG{n}{foo\PYGZus{}interval}
  \PYG{o}{\PYGZhy{}} \PYG{n}{name}\PYG{p}{:} \PYG{n}{length\PYGZus{}tester}
    \PYG{n}{class}\PYG{p}{:} \PYG{n}{MessageLengthTester}
    \PYG{n}{arg\PYGZus{}mapping}\PYG{p}{:} \PYG{n}{null}
\end{sphinxVerbatim}

With this file saved as (e.g.) \sphinxcode{\sphinxupquote{foo\_message.yaml}}, the following code
will create a \sphinxcode{\sphinxupquote{FooMessageTester}} class and instantiate it:

\fvset{hllines={, ,}}%
\begin{sphinxVerbatim}[commandchars=\\\{\}]
\PYG{n}{foo\PYGZus{}message\PYGZus{}config} \PYG{o}{=} \PYG{n}{yaml}\PYG{o}{.}\PYG{n}{load}\PYG{p}{(}\PYG{n+nb}{open}\PYG{p}{(}\PYG{l+s+s1}{\PYGZsq{}}\PYG{l+s+s1}{./foo\PYGZus{}message.yaml}\PYG{l+s+s1}{\PYGZsq{}}\PYG{p}{,} \PYG{l+s+s1}{\PYGZsq{}}\PYG{l+s+s1}{r}\PYG{l+s+s1}{\PYGZsq{}}\PYG{p}{)}\PYG{o}{.}\PYG{n}{read}\PYG{p}{(}\PYG{p}{)}\PYG{p}{)}
\PYG{n}{class\PYGZus{}factory}\PYG{p}{(}\PYG{n}{foo\PYGZus{}message\PYGZus{}config}\PYG{p}{)}
\PYG{c+c1}{\PYGZsh{} At this point, there is now a {}`FooMessageTester{}` class}
\PYG{n}{foo\PYGZus{}node} \PYG{o}{=} \PYG{n}{FooMessageTester}\PYG{p}{(}\PYG{n}{foo\PYGZus{}interval}\PYG{o}{=}\PYG{l+m+mi}{1}\PYG{p}{)}
\end{sphinxVerbatim}

You can now use \sphinxcode{\sphinxupquote{foo\_node}} just as you would any other node. So in
order to run it, you just do:

\fvset{hllines={, ,}}%
\begin{sphinxVerbatim}[commandchars=\\\{\}]
\PYG{n}{foo\PYGZus{}node}\PYG{o}{.}\PYG{n}{global\PYGZus{}start}\PYG{p}{(}\PYG{p}{)}
\end{sphinxVerbatim}

Because \sphinxcode{\sphinxupquote{foo\_node}} is just another node, you can insert it into a
larger pipeline and reuse it. For example, suppose that other
engineering team wants to add a \sphinxcode{\sphinxupquote{PrinterOfThings}} to the end of the
pipeline. They’d do this:

\fvset{hllines={, ,}}%
\begin{sphinxVerbatim}[commandchars=\\\{\}]
\PYG{n}{printer} \PYG{o}{=} \PYG{n}{PrinterOfThings}\PYG{p}{(}\PYG{p}{)}
\PYG{n}{foo\PYGZus{}node} \PYG{o}{\PYGZgt{}} \PYG{n}{printer}
\end{sphinxVerbatim}


\chapter{The Data Journey}
\label{\detokenize{data_journey:the-data-journey}}\label{\detokenize{data_journey::doc}}

\section{Overview}
\label{\detokenize{data_journey:overview}}
NanoStream pipelines create dictionary-like objects as messages, and those
messages move through the various nodes until they reach a sink. As they move
through the nodes, they are modified in one or more of the following ways:
\begin{enumerate}
\def\theenumi{\arabic{enumi}}
\def\labelenumi{\theenumi .}
\makeatletter\def\p@enumii{\p@enumi \theenumi .}\makeatother
\item {} 
New keys and values are added to the dictionary.

\item {} 
Keys and values are removed from the dictionary.

\item {} 
Values are modified through in-place operations.

\item {} 
The structure of the dictionary is changed (e.g. keys are renamed, the
dictionary is flattened, and so on).

\end{enumerate}

Which of these operations is used depends on the particular type of node that
processes the message, and how the various options are specified for that
node.

By default, each time a message is processed, all of its existing keys and
values are retained in the message as it’s passed to the next node. This
behavior is to enable the message to accumulate information over several steps
in the pipeline because some nodes require data that is generated by various
operations.

Nodes may have access to the entire message, or it is possible to specify
which key-value pair is passed to it. This is done by using the \sphinxcode{\sphinxupquote{key}}
option in the node definition. If the node will be generating results to be
passed downstream, then we need to either (1) specify the output key for those results; or
(2) make sure that the node is generating a dictionary. If (2), then by default
the dictionary will be merged into the incoming message, and the
combined dictionary will be placed on the node’s outgoing queue. If a specific
key is specified for the generated data, then we use the \sphinxcode{\sphinxupquote{output\_key}}
option in the node definition.


\section{Example: Making a GET request}
\label{\detokenize{data_journey:example-making-a-get-request}}
Let’s consider a very common ETL task. We want to make a GET request to an
API endpoint and return the result. The GET request will take a couple of
parameters, such as an endpoint name, date and username. For our example,
the URL will just be:

\fvset{hllines={, ,}}%
\begin{sphinxVerbatim}[commandchars=\\\{\}]
http://example.api.com/ENDPOINT?date=DATE\PYGZam{}username=USERNAME
\end{sphinxVerbatim}

As you would expect, the username and endpoint can be specified in advance;
but the date will change each time the pipeline is run. So the date has to
be generated at runtime and passed to the node. In our example, the
date will be passed to the pipeline as an environment variables \sphinxcode{\sphinxupquote{DATE}}
when the pipeline is executed.

NanoNode contains node classes for reading environment variables and for making
GET requests. They are called \sphinxcode{\sphinxupquote{GetEnvironmentVariables}} and
\sphinxcode{\sphinxupquote{HttpGetRequest}}. They take the following options:
\begin{enumerate}
\def\theenumi{\arabic{enumi}}
\def\labelenumi{\theenumi .}
\makeatletter\def\p@enumii{\p@enumi \theenumi .}\makeatother
\item {} 
\sphinxcode{\sphinxupquote{GetEnvironmentVariables}}
\begin{itemize}
\item {} 
\sphinxcode{\sphinxupquote{environment\_variables}}: A list of the names of the environment
variables to be fetched. The results will be put in keys named after
those environment variables.

\end{itemize}

\item {} 
\sphinxcode{\sphinxupquote{HttpGetRequest}}
\begin{itemize}
\item {} 
\sphinxcode{\sphinxupquote{url}} (required): The URL for the GET request. Any parameters that will be filled-in
at runtime should be put into curly braces. See the example configuration
file below.

\item {} 
\sphinxcode{\sphinxupquote{json}} (optional: default \sphinxcode{\sphinxupquote{True}}) Whether the response should be parsed as JSON.

\item {} 
\sphinxcode{\sphinxupquote{endpoint\_dict}} (optional) Keys and values to be substituted into the
url.

\end{itemize}

\end{enumerate}


\chapter{Treehorn}
\label{\detokenize{treehorn:treehorn}}\label{\detokenize{treehorn::doc}}
Treehorn is a set of classes for manipulating dictionary- and list-like objects in a declarative style. It is meant to be useful for the sort of tasks required for ETL, such as extracting structured data from JSON objects.


\section{Using Treehorn}
\label{\detokenize{treehorn:using-treehorn}}
Treehorn allows you to search for information in a dictionary- or list-like object by specifying conditions. Structures that match those conditions can be returned, or they can be labeled. If they are labeled, you can use those labels to build more complex searches later, or retrieve the data. The style of Treehorn is somewhat like JQuery and similar languages that are good for manipulating tree-like data structures such as web pages.

We’ll explain Treehorn by stepping through an example of how we would extract data from the following JSON blob:

\fvset{hllines={, ,}}%
\begin{sphinxVerbatim}[commandchars=\\\{\}]
\PYG{p}{\PYGZob{}}
    \PYG{l+s+s2}{\PYGZdq{}}\PYG{l+s+s2}{source}\PYG{l+s+s2}{\PYGZdq{}}\PYG{p}{:} \PYG{l+s+s2}{\PYGZdq{}}\PYG{l+s+s2}{users}\PYG{l+s+s2}{\PYGZdq{}}\PYG{p}{,}
    \PYG{l+s+s2}{\PYGZdq{}}\PYG{l+s+s2}{hash}\PYG{l+s+s2}{\PYGZdq{}}\PYG{p}{:} \PYG{l+s+s2}{\PYGZdq{}}\PYG{l+s+s2}{Ch8KFgjQj67igOnVto4BELHgwMD7iNfjkQEYlrfjtZAt}\PYG{l+s+s2}{\PYGZdq{}}\PYG{p}{,}
    \PYG{l+s+s2}{\PYGZdq{}}\PYG{l+s+s2}{events}\PYG{l+s+s2}{\PYGZdq{}}\PYG{p}{:} \PYG{p}{[}
        \PYG{p}{\PYGZob{}}
            \PYG{l+s+s2}{\PYGZdq{}}\PYG{l+s+s2}{appName}\PYG{l+s+s2}{\PYGZdq{}}\PYG{p}{:} \PYG{l+s+s2}{\PYGZdq{}}\PYG{l+s+s2}{mobileapp}\PYG{l+s+s2}{\PYGZdq{}}\PYG{p}{,}
            \PYG{l+s+s2}{\PYGZdq{}}\PYG{l+s+s2}{browser}\PYG{l+s+s2}{\PYGZdq{}}\PYG{p}{:} \PYG{p}{\PYGZob{}}
                \PYG{l+s+s2}{\PYGZdq{}}\PYG{l+s+s2}{name}\PYG{l+s+s2}{\PYGZdq{}}\PYG{p}{:} \PYG{l+s+s2}{\PYGZdq{}}\PYG{l+s+s2}{Google Chrome}\PYG{l+s+s2}{\PYGZdq{}}\PYG{p}{,}
                \PYG{l+s+s2}{\PYGZdq{}}\PYG{l+s+s2}{version}\PYG{l+s+s2}{\PYGZdq{}}\PYG{p}{:} \PYG{p}{[}\PYG{p}{]}
            \PYG{p}{\PYGZcb{}}\PYG{p}{,}
            \PYG{l+s+s2}{\PYGZdq{}}\PYG{l+s+s2}{duration}\PYG{l+s+s2}{\PYGZdq{}}\PYG{p}{:} \PYG{l+m+mi}{0}\PYG{p}{,}
            \PYG{l+s+s2}{\PYGZdq{}}\PYG{l+s+s2}{created}\PYG{l+s+s2}{\PYGZdq{}}\PYG{p}{:} \PYG{l+m+mi}{1550596005797}\PYG{p}{,}
            \PYG{l+s+s2}{\PYGZdq{}}\PYG{l+s+s2}{location}\PYG{l+s+s2}{\PYGZdq{}}\PYG{p}{:} \PYG{p}{\PYGZob{}}
                \PYG{l+s+s2}{\PYGZdq{}}\PYG{l+s+s2}{country}\PYG{l+s+s2}{\PYGZdq{}}\PYG{p}{:} \PYG{l+s+s2}{\PYGZdq{}}\PYG{l+s+s2}{United States}\PYG{l+s+s2}{\PYGZdq{}}\PYG{p}{,}
                \PYG{l+s+s2}{\PYGZdq{}}\PYG{l+s+s2}{state}\PYG{l+s+s2}{\PYGZdq{}}\PYG{p}{:} \PYG{l+s+s2}{\PYGZdq{}}\PYG{l+s+s2}{Massachusetts}\PYG{l+s+s2}{\PYGZdq{}}\PYG{p}{,}
                \PYG{l+s+s2}{\PYGZdq{}}\PYG{l+s+s2}{city}\PYG{l+s+s2}{\PYGZdq{}}\PYG{p}{:} \PYG{l+s+s2}{\PYGZdq{}}\PYG{l+s+s2}{Boston}\PYG{l+s+s2}{\PYGZdq{}}
            \PYG{p}{\PYGZcb{}}\PYG{p}{,}
            \PYG{l+s+s2}{\PYGZdq{}}\PYG{l+s+s2}{id}\PYG{l+s+s2}{\PYGZdq{}}\PYG{p}{:} \PYG{l+s+s2}{\PYGZdq{}}\PYG{l+s+s2}{af6de71b}\PYG{l+s+s2}{\PYGZdq{}}\PYG{p}{,}
            \PYG{l+s+s2}{\PYGZdq{}}\PYG{l+s+s2}{smtpId}\PYG{l+s+s2}{\PYGZdq{}}\PYG{p}{:} \PYG{n}{null}\PYG{p}{,}
            \PYG{l+s+s2}{\PYGZdq{}}\PYG{l+s+s2}{portalId}\PYG{l+s+s2}{\PYGZdq{}}\PYG{p}{:} \PYG{l+m+mi}{537105}\PYG{p}{,}
            \PYG{l+s+s2}{\PYGZdq{}}\PYG{l+s+s2}{email}\PYG{l+s+s2}{\PYGZdq{}}\PYG{p}{:} \PYG{l+s+s2}{\PYGZdq{}}\PYG{l+s+s2}{alice@gmail.com}\PYG{l+s+s2}{\PYGZdq{}}\PYG{p}{,}
            \PYG{l+s+s2}{\PYGZdq{}}\PYG{l+s+s2}{sentBy}\PYG{l+s+s2}{\PYGZdq{}}\PYG{p}{:} \PYG{p}{\PYGZob{}}
                \PYG{l+s+s2}{\PYGZdq{}}\PYG{l+s+s2}{id}\PYG{l+s+s2}{\PYGZdq{}}\PYG{p}{:} \PYG{l+s+s2}{\PYGZdq{}}\PYG{l+s+s2}{befa29c9}\PYG{l+s+s2}{\PYGZdq{}}\PYG{p}{,}
                \PYG{l+s+s2}{\PYGZdq{}}\PYG{l+s+s2}{created}\PYG{l+s+s2}{\PYGZdq{}}\PYG{p}{:} \PYG{l+m+mi}{1550518557458}
            \PYG{p}{\PYGZcb{}}\PYG{p}{,}
            \PYG{l+s+s2}{\PYGZdq{}}\PYG{l+s+s2}{type}\PYG{l+s+s2}{\PYGZdq{}}\PYG{p}{:} \PYG{l+s+s2}{\PYGZdq{}}\PYG{l+s+s2}{OPEN}\PYG{l+s+s2}{\PYGZdq{}}\PYG{p}{,}
            \PYG{l+s+s2}{\PYGZdq{}}\PYG{l+s+s2}{filteredEvent}\PYG{l+s+s2}{\PYGZdq{}}\PYG{p}{:} \PYG{n}{false}\PYG{p}{,}
            \PYG{l+s+s2}{\PYGZdq{}}\PYG{l+s+s2}{deviceType}\PYG{l+s+s2}{\PYGZdq{}}\PYG{p}{:} \PYG{l+s+s2}{\PYGZdq{}}\PYG{l+s+s2}{COMPUTER}\PYG{l+s+s2}{\PYGZdq{}}
        \PYG{p}{\PYGZcb{}}\PYG{p}{,}
        \PYG{p}{\PYGZob{}}
            \PYG{l+s+s2}{\PYGZdq{}}\PYG{l+s+s2}{appName}\PYG{l+s+s2}{\PYGZdq{}}\PYG{p}{:} \PYG{l+s+s2}{\PYGZdq{}}\PYG{l+s+s2}{desktopapp}\PYG{l+s+s2}{\PYGZdq{}}\PYG{p}{,}
            \PYG{l+s+s2}{\PYGZdq{}}\PYG{l+s+s2}{browser}\PYG{l+s+s2}{\PYGZdq{}}\PYG{p}{:} \PYG{p}{\PYGZob{}}
                \PYG{l+s+s2}{\PYGZdq{}}\PYG{l+s+s2}{name}\PYG{l+s+s2}{\PYGZdq{}}\PYG{p}{:} \PYG{l+s+s2}{\PYGZdq{}}\PYG{l+s+s2}{Firefox}\PYG{l+s+s2}{\PYGZdq{}}\PYG{p}{,}
                \PYG{l+s+s2}{\PYGZdq{}}\PYG{l+s+s2}{version}\PYG{l+s+s2}{\PYGZdq{}}\PYG{p}{:} \PYG{p}{[}\PYG{p}{]}
            \PYG{p}{\PYGZcb{}}\PYG{p}{,}
            \PYG{l+s+s2}{\PYGZdq{}}\PYG{l+s+s2}{duration}\PYG{l+s+s2}{\PYGZdq{}}\PYG{p}{:} \PYG{l+m+mi}{0}\PYG{p}{,}
            \PYG{l+s+s2}{\PYGZdq{}}\PYG{l+s+s2}{created}\PYG{l+s+s2}{\PYGZdq{}}\PYG{p}{:} \PYG{l+m+mi}{1550596005389}\PYG{p}{,}
            \PYG{l+s+s2}{\PYGZdq{}}\PYG{l+s+s2}{location}\PYG{l+s+s2}{\PYGZdq{}}\PYG{p}{:} \PYG{p}{\PYGZob{}}
                \PYG{l+s+s2}{\PYGZdq{}}\PYG{l+s+s2}{country}\PYG{l+s+s2}{\PYGZdq{}}\PYG{p}{:} \PYG{l+s+s2}{\PYGZdq{}}\PYG{l+s+s2}{United States}\PYG{l+s+s2}{\PYGZdq{}}\PYG{p}{,}
                \PYG{l+s+s2}{\PYGZdq{}}\PYG{l+s+s2}{state}\PYG{l+s+s2}{\PYGZdq{}}\PYG{p}{:} \PYG{l+s+s2}{\PYGZdq{}}\PYG{l+s+s2}{New York}\PYG{l+s+s2}{\PYGZdq{}}\PYG{p}{,}
                \PYG{l+s+s2}{\PYGZdq{}}\PYG{l+s+s2}{city}\PYG{l+s+s2}{\PYGZdq{}}\PYG{p}{:} \PYG{l+s+s2}{\PYGZdq{}}\PYG{l+s+s2}{New York}\PYG{l+s+s2}{\PYGZdq{}}
            \PYG{p}{\PYGZcb{}}\PYG{p}{,}
            \PYG{l+s+s2}{\PYGZdq{}}\PYG{l+s+s2}{id}\PYG{l+s+s2}{\PYGZdq{}}\PYG{p}{:} \PYG{l+s+s2}{\PYGZdq{}}\PYG{l+s+s2}{12aadd80}\PYG{l+s+s2}{\PYGZdq{}}\PYG{p}{,}
            \PYG{l+s+s2}{\PYGZdq{}}\PYG{l+s+s2}{smtpId}\PYG{l+s+s2}{\PYGZdq{}}\PYG{p}{:} \PYG{n}{null}\PYG{p}{,}
            \PYG{l+s+s2}{\PYGZdq{}}\PYG{l+s+s2}{portalId}\PYG{l+s+s2}{\PYGZdq{}}\PYG{p}{:} \PYG{l+m+mi}{537105}\PYG{p}{,}
            \PYG{l+s+s2}{\PYGZdq{}}\PYG{l+s+s2}{email}\PYG{l+s+s2}{\PYGZdq{}}\PYG{p}{:} \PYG{l+s+s2}{\PYGZdq{}}\PYG{l+s+s2}{bob@gmail.com}\PYG{l+s+s2}{\PYGZdq{}}\PYG{p}{,}
            \PYG{l+s+s2}{\PYGZdq{}}\PYG{l+s+s2}{sentBy}\PYG{l+s+s2}{\PYGZdq{}}\PYG{p}{:} \PYG{p}{\PYGZob{}}
                \PYG{l+s+s2}{\PYGZdq{}}\PYG{l+s+s2}{id}\PYG{l+s+s2}{\PYGZdq{}}\PYG{p}{:} \PYG{l+s+s2}{\PYGZdq{}}\PYG{l+s+s2}{2cd1e257}\PYG{l+s+s2}{\PYGZdq{}}\PYG{p}{,}
                \PYG{l+s+s2}{\PYGZdq{}}\PYG{l+s+s2}{created}\PYG{l+s+s2}{\PYGZdq{}}\PYG{p}{:} \PYG{l+m+mi}{1550581974777}
            \PYG{p}{\PYGZcb{}}\PYG{p}{,}
            \PYG{l+s+s2}{\PYGZdq{}}\PYG{l+s+s2}{type}\PYG{l+s+s2}{\PYGZdq{}}\PYG{p}{:} \PYG{l+s+s2}{\PYGZdq{}}\PYG{l+s+s2}{OPEN}\PYG{l+s+s2}{\PYGZdq{}}\PYG{p}{,}
            \PYG{l+s+s2}{\PYGZdq{}}\PYG{l+s+s2}{filteredEvent}\PYG{l+s+s2}{\PYGZdq{}}\PYG{p}{:} \PYG{n}{false}\PYG{p}{,}
            \PYG{l+s+s2}{\PYGZdq{}}\PYG{l+s+s2}{deviceType}\PYG{l+s+s2}{\PYGZdq{}}\PYG{p}{:} \PYG{l+s+s2}{\PYGZdq{}}\PYG{l+s+s2}{COMPUTER}\PYG{l+s+s2}{\PYGZdq{}}
        \PYG{p}{\PYGZcb{}}
    \PYG{p}{]}
\PYG{p}{\PYGZcb{}}
\end{sphinxVerbatim}

As you can see, this JSON blob is similar to a typical response from a REST API (in fact, this is actually an example from a real REST API, with all personal information deleted).

Let’s suppose you need to extract the email address and corresponding city name for each entry in \sphinxcode{\sphinxupquote{events}}. This example is simple enough that you might not see the usefulness of Treehorn, but it’s complex enough to get a sense of how Treehorn works. Later, we’ll look at circumstances where Treehorn’s declarative style is especially useful.

There are three kinds of classes that are important for Treehorn:
\begin{enumerate}
\def\theenumi{\arabic{enumi}}
\def\labelenumi{\theenumi .}
\makeatletter\def\p@enumii{\p@enumi \theenumi .}\makeatother
\item {} 
\sphinxcode{\sphinxupquote{Conditions}} \textendash{} These are classes that test a particular location in a tree (e.g. a dictionary) for some condition. Examples of useful conditions are being a dictionary with a certain key, being a non-empty list, having an integer value, and so on.

\item {} 
\sphinxcode{\sphinxupquote{Traversal}} \textendash{} These classes move throughout a tree, recursively applying tests to each node that they visit. Traversals can be upward (toward the root) or downward (toward the leaves).

\item {} 
\sphinxcode{\sphinxupquote{Label}} \textendash{} These are nothing more than strings that are attached to particular locations in the tree. Typically, we apply a label to locations in the tree that match particular conditions.

\item {} 
\sphinxcode{\sphinxupquote{Relations}} \textendash{} Finally, this class represents n-tuples of locations in the tree. For example, if an email address is present in the tree, and the user’s city is present, a \sphinxcode{\sphinxupquote{Relation}} can be used to denote that the person with that email address lives in that city.

\end{enumerate}

The workflow for a typical Treehorn query is that we (1) define some conditions (such as being an email address field); (2) traverse the tree, searching for locations that match those conditions; (3) label those locations; and (4) define a relationship from those labels, which we can use to extract the right information. We’ll gradually build up a query by adding each of these steps one at a time.


\subsection{Condition objects}
\label{\detokenize{treehorn:condition-objects}}
For this example, let’s suppose you’ve loaded the JSON into a dictionary, like so:

\fvset{hllines={, ,}}%
\begin{sphinxVerbatim}[commandchars=\\\{\}]
\PYG{k+kn}{import} \PYG{n+nn}{json}

\PYG{k}{with} \PYG{n+nb}{open}\PYG{p}{(}\PYG{l+s+s1}{\PYGZsq{}}\PYG{l+s+s1}{./sample\PYGZus{}api\PYGZus{}response.json}\PYG{l+s+s1}{\PYGZsq{}}\PYG{p}{,} \PYG{l+s+s1}{\PYGZsq{}}\PYG{l+s+s1}{r}\PYG{l+s+s1}{\PYGZsq{}}\PYG{p}{)} \PYG{k}{as} \PYG{n}{infile}\PYG{p}{:}
    \PYG{n}{api\PYGZus{}response} \PYG{o}{=} \PYG{n}{json}\PYG{o}{.}\PYG{n}{load}\PYG{p}{(}\PYG{n}{infile}\PYG{p}{)}
\end{sphinxVerbatim}

Let’s extract the email addresses and corresonding cities for each user in the API response. First, we create a couple of \sphinxcode{\sphinxupquote{Condition}} objects using the built-in class \sphinxcode{\sphinxupquote{HasKey}}:

\fvset{hllines={, ,}}%
\begin{sphinxVerbatim}[commandchars=\\\{\}]
\PYG{n}{has\PYGZus{}email\PYGZus{}key} \PYG{o}{=} \PYG{n}{HasKey}\PYG{p}{(}\PYG{l+s+s1}{\PYGZsq{}}\PYG{l+s+s1}{email}\PYG{l+s+s1}{\PYGZsq{}}\PYG{p}{)}
\PYG{n}{has\PYGZus{}city\PYGZus{}key} \PYG{o}{=} \PYG{n}{HasKey}\PYG{p}{(}\PYG{l+s+s1}{\PYGZsq{}}\PYG{l+s+s1}{city}\PYG{l+s+s1}{\PYGZsq{}}\PYG{p}{)}
\end{sphinxVerbatim}

The \sphinxcode{\sphinxupquote{HasKey}} class is a subclass of \sphinxcode{\sphinxupquote{MeetsCondition}}, all of which are callable and return \sphinxcode{\sphinxupquote{True}} or \sphinxcode{\sphinxupquote{False}}. For example, you could do the following:

\fvset{hllines={, ,}}%
\begin{sphinxVerbatim}[commandchars=\\\{\}]
\PYG{n}{d} \PYG{o}{=} \PYG{p}{\PYGZob{}}\PYG{l+s+s1}{\PYGZsq{}}\PYG{l+s+s1}{email}\PYG{l+s+s1}{\PYGZsq{}}\PYG{p}{:} \PYG{l+s+s1}{\PYGZsq{}}\PYG{l+s+s1}{myemail.com}\PYG{l+s+s1}{\PYGZsq{}}\PYG{p}{,} \PYG{l+s+s1}{\PYGZsq{}}\PYG{l+s+s1}{name}\PYG{l+s+s1}{\PYGZsq{}}\PYG{p}{:} \PYG{l+s+s1}{\PYGZsq{}}\PYG{l+s+s1}{carol}\PYG{l+s+s1}{\PYGZsq{}}\PYG{p}{\PYGZcb{}}
\PYG{n}{has\PYGZus{}email\PYGZus{}key}\PYG{p}{(}\PYG{n}{d}\PYG{p}{)}  \PYG{c+c1}{\PYGZsh{} Returns True}
\PYG{n}{has\PYGZus{}city\PYGZus{}key}\PYG{p}{(}\PYG{n}{d}\PYG{p}{)}   \PYG{c+c1}{\PYGZsh{} Returns False}
\end{sphinxVerbatim}

What if you want to test for two conditions on a single node? \sphinxcode{\sphinxupquote{MeetsCondition}} objects can be combined into larger boolean expressions using \sphinxcode{\sphinxupquote{\&}}, \sphinxcode{\sphinxupquote{\textbar{}}}, and \sphinxcode{\sphinxupquote{\textasciitilde{}}} like so:

\fvset{hllines={, ,}}%
\begin{sphinxVerbatim}[commandchars=\\\{\}]
\PYG{p}{(}\PYG{n}{has\PYGZus{}email\PYGZus{}key} \PYG{o}{\PYGZam{}} \PYG{n}{has\PYGZus{}city\PYGZus{}key}\PYG{p}{)}\PYG{p}{(}\PYG{n}{d}\PYG{p}{)}    \PYG{c+c1}{\PYGZsh{} Returns False}
\PYG{p}{(}\PYG{n}{has\PYGZus{}email\PYGZus{}key} \PYG{o}{\PYGZam{}} \PYG{o}{\PYGZti{}} \PYG{n}{has\PYGZus{}city\PYGZus{}key}\PYG{p}{)}\PYG{p}{(}\PYG{n}{d}\PYG{p}{)}  \PYG{c+c1}{\PYGZsh{} Returns True}
\PYG{p}{(}\PYG{n}{has\PYGZus{}email\PYGZus{}key} \PYG{o}{\textbar{}} \PYG{n}{has\PYGZus{}city\PYGZus{}key}\PYG{p}{)}\PYG{p}{(}\PYG{n}{d}\PYG{p}{)}    \PYG{c+c1}{\PYGZsh{} Returns True}
\end{sphinxVerbatim}


\subsection{Traversal objects}
\label{\detokenize{treehorn:traversal-objects}}
\sphinxcode{\sphinxupquote{MeetsCondition}} objects aren’t very useful unless they’re combined with traversals. There are two types of traversal classes: \sphinxcode{\sphinxupquote{GoUp}} and \sphinxcode{\sphinxupquote{GoDown}}. Each takes a \sphinxcode{\sphinxupquote{MeetsCondition}} object as a parameter. For example, if you want to search from the root of the tree for every location that is a dictionary with the \sphinxcode{\sphinxupquote{email}} key, the traversal is:

\fvset{hllines={, ,}}%
\begin{sphinxVerbatim}[commandchars=\\\{\}]
\PYG{n}{find\PYGZus{}email} \PYG{o}{=} \PYG{n}{GoDown}\PYG{p}{(}\PYG{n}{condition}\PYG{o}{=}\PYG{n}{has\PYGZus{}email\PYGZus{}key}\PYG{p}{)}  \PYG{c+c1}{\PYGZsh{} or GoDown(condition=HasKey(\PYGZsq{}email\PYGZsq{}))}
\end{sphinxVerbatim}

Similarly for finding places with a \sphinxcode{\sphinxupquote{city}} key:

\fvset{hllines={, ,}}%
\begin{sphinxVerbatim}[commandchars=\\\{\}]
\PYG{n}{find\PYGZus{}city} \PYG{o}{=} \PYG{n}{GoDown}\PYG{p}{(}\PYG{n}{condition}\PYG{o}{=}\PYG{n}{has\PYGZus{}city\PYGZus{}key}\PYG{p}{)}  \PYG{c+c1}{\PYGZsh{} or GoDown(condition=HasKey(\PYGZsq{}city\PYGZsq{}))}
\end{sphinxVerbatim}

If you want to retrieve all of \sphinxcode{\sphinxupquote{find\_city}}’s matches, you can use its \sphinxcode{\sphinxupquote{matches}} method, which will yield each match:

\fvset{hllines={, ,}}%
\begin{sphinxVerbatim}[commandchars=\\\{\}]
\PYG{k}{for} \PYG{n}{match} \PYG{o+ow}{in} \PYG{n}{has\PYGZus{}email\PYGZus{}key}\PYG{o}{.}\PYG{n}{matches}\PYG{p}{(}\PYG{n}{api\PYGZus{}response}\PYG{p}{)}\PYG{p}{:}
    \PYG{n+nb}{print}\PYG{p}{(}\PYG{n}{match}\PYG{p}{)}
\end{sphinxVerbatim}

which will yield:

\fvset{hllines={, ,}}%
\begin{sphinxVerbatim}[commandchars=\\\{\}]
\PYG{p}{\PYGZob{}}\PYG{l+s+s1}{\PYGZsq{}}\PYG{l+s+s1}{id}\PYG{l+s+s1}{\PYGZsq{}}\PYG{p}{:} \PYG{l+s+s1}{\PYGZsq{}}\PYG{l+s+s1}{af6de71b}\PYG{l+s+s1}{\PYGZsq{}}\PYG{p}{,} \PYG{l+s+s1}{\PYGZsq{}}\PYG{l+s+s1}{portalId}\PYG{l+s+s1}{\PYGZsq{}}\PYG{p}{:} \PYG{l+m+mi}{537105}\PYG{p}{,} \PYG{l+s+s1}{\PYGZsq{}}\PYG{l+s+s1}{location}\PYG{l+s+s1}{\PYGZsq{}}\PYG{p}{:} \PYG{p}{\PYGZob{}}\PYG{l+s+s1}{\PYGZsq{}}\PYG{l+s+s1}{state}\PYG{l+s+s1}{\PYGZsq{}}\PYG{p}{:} \PYG{l+s+s1}{\PYGZsq{}}\PYG{l+s+s1}{Massachusetts}\PYG{l+s+s1}{\PYGZsq{}}\PYG{p}{,} \PYG{l+s+s1}{\PYGZsq{}}\PYG{l+s+s1}{city}\PYG{l+s+s1}{\PYGZsq{}}\PYG{p}{:} \PYG{l+s+s1}{\PYGZsq{}}\PYG{l+s+s1}{Boston}\PYG{l+s+s1}{\PYGZsq{}}\PYG{p}{,} \PYG{l+s+s1}{\PYGZsq{}}\PYG{l+s+s1}{country}\PYG{l+s+s1}{\PYGZsq{}}\PYG{p}{:} \PYG{l+s+s1}{\PYGZsq{}}\PYG{l+s+s1}{United States}\PYG{l+s+s1}{\PYGZsq{}}\PYG{p}{\PYGZcb{}}\PYG{p}{,} \PYG{l+s+s1}{\PYGZsq{}}\PYG{l+s+s1}{type}\PYG{l+s+s1}{\PYGZsq{}}\PYG{p}{:} \PYG{l+s+s1}{\PYGZsq{}}\PYG{l+s+s1}{OPEN}\PYG{l+s+s1}{\PYGZsq{}}\PYG{p}{,} \PYG{l+s+s1}{\PYGZsq{}}\PYG{l+s+s1}{sentBy}\PYG{l+s+s1}{\PYGZsq{}}\PYG{p}{:} \PYG{p}{\PYGZob{}}\PYG{l+s+s1}{\PYGZsq{}}\PYG{l+s+s1}{id}\PYG{l+s+s1}{\PYGZsq{}}\PYG{p}{:} \PYG{l+s+s1}{\PYGZsq{}}\PYG{l+s+s1}{befa29c9}\PYG{l+s+s1}{\PYGZsq{}}\PYG{p}{,} \PYG{l+s+s1}{\PYGZsq{}}\PYG{l+s+s1}{created}\PYG{l+s+s1}{\PYGZsq{}}\PYG{p}{:} \PYG{l+m+mi}{1550518557458}\PYG{p}{\PYGZcb{}}\PYG{p}{,} \PYG{l+s+s1}{\PYGZsq{}}\PYG{l+s+s1}{appName}\PYG{l+s+s1}{\PYGZsq{}}\PYG{p}{:} \PYG{l+s+s1}{\PYGZsq{}}\PYG{l+s+s1}{mobileapp}\PYG{l+s+s1}{\PYGZsq{}}\PYG{p}{,} \PYG{l+s+s1}{\PYGZsq{}}\PYG{l+s+s1}{duration}\PYG{l+s+s1}{\PYGZsq{}}\PYG{p}{:} \PYG{l+m+mi}{0}\PYG{p}{,}\PYG{l+s+s1}{\PYGZsq{}}\PYG{l+s+s1}{smtpId}\PYG{l+s+s1}{\PYGZsq{}}\PYG{p}{:} \PYG{k+kc}{None}\PYG{p}{,} \PYG{l+s+s1}{\PYGZsq{}}\PYG{l+s+s1}{deviceType}\PYG{l+s+s1}{\PYGZsq{}}\PYG{p}{:} \PYG{l+s+s1}{\PYGZsq{}}\PYG{l+s+s1}{COMPUTER}\PYG{l+s+s1}{\PYGZsq{}}\PYG{p}{,} \PYG{l+s+s1}{\PYGZsq{}}\PYG{l+s+s1}{created}\PYG{l+s+s1}{\PYGZsq{}}\PYG{p}{:} \PYG{l+m+mi}{1550596005797}\PYG{p}{,} \PYG{l+s+s1}{\PYGZsq{}}\PYG{l+s+s1}{email}\PYG{l+s+s1}{\PYGZsq{}}\PYG{p}{:} \PYG{l+s+s1}{\PYGZsq{}}\PYG{l+s+s1}{alice@gmail.com}\PYG{l+s+s1}{\PYGZsq{}}\PYG{p}{,} \PYG{l+s+s1}{\PYGZsq{}}\PYG{l+s+s1}{browser}\PYG{l+s+s1}{\PYGZsq{}}\PYG{p}{:} \PYG{p}{\PYGZob{}}\PYG{l+s+s1}{\PYGZsq{}}\PYG{l+s+s1}{version}\PYG{l+s+s1}{\PYGZsq{}}\PYG{p}{:} \PYG{p}{[}\PYG{p}{]}\PYG{p}{,} \PYG{l+s+s1}{\PYGZsq{}}\PYG{l+s+s1}{name}\PYG{l+s+s1}{\PYGZsq{}}\PYG{p}{:} \PYG{l+s+s1}{\PYGZsq{}}\PYG{l+s+s1}{Google Chrome}\PYG{l+s+s1}{\PYGZsq{}}\PYG{p}{\PYGZcb{}}\PYG{p}{,} \PYG{l+s+s1}{\PYGZsq{}}\PYG{l+s+s1}{filteredEvent}\PYG{l+s+s1}{\PYGZsq{}}\PYG{p}{:} \PYG{k+kc}{False}\PYG{p}{\PYGZcb{}}
\PYG{p}{\PYGZob{}}\PYG{l+s+s1}{\PYGZsq{}}\PYG{l+s+s1}{id}\PYG{l+s+s1}{\PYGZsq{}}\PYG{p}{:} \PYG{l+s+s1}{\PYGZsq{}}\PYG{l+s+s1}{12aadd80}\PYG{l+s+s1}{\PYGZsq{}}\PYG{p}{,} \PYG{l+s+s1}{\PYGZsq{}}\PYG{l+s+s1}{portalId}\PYG{l+s+s1}{\PYGZsq{}}\PYG{p}{:} \PYG{l+m+mi}{537105}\PYG{p}{,} \PYG{l+s+s1}{\PYGZsq{}}\PYG{l+s+s1}{location}\PYG{l+s+s1}{\PYGZsq{}}\PYG{p}{:} \PYG{p}{\PYGZob{}}\PYG{l+s+s1}{\PYGZsq{}}\PYG{l+s+s1}{state}\PYG{l+s+s1}{\PYGZsq{}}\PYG{p}{:} \PYG{l+s+s1}{\PYGZsq{}}\PYG{l+s+s1}{New York}\PYG{l+s+s1}{\PYGZsq{}}\PYG{p}{,} \PYG{l+s+s1}{\PYGZsq{}}\PYG{l+s+s1}{city}\PYG{l+s+s1}{\PYGZsq{}}\PYG{p}{:} \PYG{l+s+s1}{\PYGZsq{}}\PYG{l+s+s1}{New York}\PYG{l+s+s1}{\PYGZsq{}}\PYG{p}{,} \PYG{l+s+s1}{\PYGZsq{}}\PYG{l+s+s1}{country}\PYG{l+s+s1}{\PYGZsq{}}\PYG{p}{:} \PYG{l+s+s1}{\PYGZsq{}}\PYG{l+s+s1}{United States}\PYG{l+s+s1}{\PYGZsq{}}\PYG{p}{\PYGZcb{}}\PYG{p}{,} \PYG{l+s+s1}{\PYGZsq{}}\PYG{l+s+s1}{type}\PYG{l+s+s1}{\PYGZsq{}}\PYG{p}{:} \PYG{l+s+s1}{\PYGZsq{}}\PYG{l+s+s1}{OPEN}\PYG{l+s+s1}{\PYGZsq{}}\PYG{p}{,} \PYG{l+s+s1}{\PYGZsq{}}\PYG{l+s+s1}{sentBy}\PYG{l+s+s1}{\PYGZsq{}}\PYG{p}{:} \PYG{p}{\PYGZob{}}\PYG{l+s+s1}{\PYGZsq{}}\PYG{l+s+s1}{id}\PYG{l+s+s1}{\PYGZsq{}}\PYG{p}{:} \PYG{l+s+s1}{\PYGZsq{}}\PYG{l+s+s1}{2cd1e257}\PYG{l+s+s1}{\PYGZsq{}}\PYG{p}{,} \PYG{l+s+s1}{\PYGZsq{}}\PYG{l+s+s1}{created}\PYG{l+s+s1}{\PYGZsq{}}\PYG{p}{:} \PYG{l+m+mi}{1550581974777}\PYG{p}{\PYGZcb{}}\PYG{p}{,} \PYG{l+s+s1}{\PYGZsq{}}\PYG{l+s+s1}{appName}\PYG{l+s+s1}{\PYGZsq{}}\PYG{p}{:} \PYG{l+s+s1}{\PYGZsq{}}\PYG{l+s+s1}{desktopapp}\PYG{l+s+s1}{\PYGZsq{}}\PYG{p}{,} \PYG{l+s+s1}{\PYGZsq{}}\PYG{l+s+s1}{duration}\PYG{l+s+s1}{\PYGZsq{}}\PYG{p}{:} \PYG{l+m+mi}{0}\PYG{p}{,} \PYG{l+s+s1}{\PYGZsq{}}\PYG{l+s+s1}{smtpId}\PYG{l+s+s1}{\PYGZsq{}}\PYG{p}{:} \PYG{k+kc}{None}\PYG{p}{,} \PYG{l+s+s1}{\PYGZsq{}}\PYG{l+s+s1}{deviceType}\PYG{l+s+s1}{\PYGZsq{}}\PYG{p}{:} \PYG{l+s+s1}{\PYGZsq{}}\PYG{l+s+s1}{COMPUTER}\PYG{l+s+s1}{\PYGZsq{}}\PYG{p}{,} \PYG{l+s+s1}{\PYGZsq{}}\PYG{l+s+s1}{created}\PYG{l+s+s1}{\PYGZsq{}}\PYG{p}{:} \PYG{l+m+mi}{1550596005389}\PYG{p}{,} \PYG{l+s+s1}{\PYGZsq{}}\PYG{l+s+s1}{email}\PYG{l+s+s1}{\PYGZsq{}}\PYG{p}{:} \PYG{l+s+s1}{\PYGZsq{}}\PYG{l+s+s1}{bob@gmail.com}\PYG{l+s+s1}{\PYGZsq{}}\PYG{p}{,} \PYG{l+s+s1}{\PYGZsq{}}\PYG{l+s+s1}{browser}\PYG{l+s+s1}{\PYGZsq{}}\PYG{p}{:} \PYG{p}{\PYGZob{}}\PYG{l+s+s1}{\PYGZsq{}}\PYG{l+s+s1}{version}\PYG{l+s+s1}{\PYGZsq{}}\PYG{p}{:} \PYG{p}{[}\PYG{p}{]}\PYG{p}{,} \PYG{l+s+s1}{\PYGZsq{}}\PYG{l+s+s1}{name}\PYG{l+s+s1}{\PYGZsq{}}\PYG{p}{:} \PYG{l+s+s1}{\PYGZsq{}}\PYG{l+s+s1}{Firefox}\PYG{l+s+s1}{\PYGZsq{}}\PYG{p}{\PYGZcb{}}\PYG{p}{,} \PYG{l+s+s1}{\PYGZsq{}}\PYG{l+s+s1}{filteredEvent}\PYG{l+s+s1}{\PYGZsq{}}\PYG{p}{:} \PYG{k+kc}{False}\PYG{p}{\PYGZcb{}}
\end{sphinxVerbatim}

Examining each of the dictionaries, we see that they do in fact contain an \sphinxcode{\sphinxupquote{email}} key. Note that the traversal does \sphinxstylestrong{not} return the email string itself \textendash{} we asked only for the dictionary containing the key. This is by design, as we will see soon.

Because we want to retrieve not only the email addresses but also the cities, we need another traversal. Each of the two dictionaries containing the \sphinxcode{\sphinxupquote{email}} key also have a subdictionary that contains a \sphinxcode{\sphinxupquote{city}} key. So we need a second traversal to get that subdictionary. In other words, retrieving the data we need is a two-step process:
\begin{enumerate}
\def\theenumi{\arabic{enumi}}
\def\labelenumi{\theenumi .}
\makeatletter\def\p@enumii{\p@enumi \theenumi .}\makeatother
\item {} 
Starting at the root of the tree, we traverse downward until we find a dictionary with the \sphinxcode{\sphinxupquote{email}} key.

\item {} 
From each of those dictionaries, we go down until we find a dictionary with the \sphinxcode{\sphinxupquote{city}} key.

\end{enumerate}

Any non-trivial ETL task involving nested dictionary-like objects will require multi-stage traversals like this one. So Treehorn allows you to chain traversals together using the \sphinxcode{\sphinxupquote{\textgreater{}}} operator:

\fvset{hllines={, ,}}%
\begin{sphinxVerbatim}[commandchars=\\\{\}]
\PYG{n}{chained\PYGZus{}traversal} \PYG{o}{=} \PYG{n}{find\PYGZus{}email} \PYG{o}{\PYGZgt{}} \PYG{n}{find\PYGZus{}city}
\end{sphinxVerbatim}

The \sphinxcode{\sphinxupquote{chained\_traversal}} says, in effect, “Go down into the tree and find every node that has an \sphinxcode{\sphinxupquote{email}} key. Then, from each of those, continue to go down until you find a node that contains a \sphinxcode{\sphinxupquote{city}} key. In pseudo-code:

\fvset{hllines={, ,}}%
\begin{sphinxVerbatim}[commandchars=\\\{\}]
For each node\PYGZus{}1 starting at the root:
    if node\PYGZus{}1 has {}`{}`email{}`{}` key:
        for each node\PYGZus{}2 starting at node\PYGZus{}1:
            if node\PYGZus{}2 has {}`{}`city{}`{}` key:
                return
\end{sphinxVerbatim}

So far, we have set set up multi-stage searches for nodes in a tree that satisfy various conditions. Next, we have to extract the right data from those searches. This is where the \sphinxcode{\sphinxupquote{Label}} and \sphinxcode{\sphinxupquote{Relation}} classes come into play.


\subsection{Labels}
\label{\detokenize{treehorn:labels}}
When nodes are identified that satisfy certain conditions, we will want to label those nodes so that we can extract data from them later. The mechanism for doing this is to use a “label”.

Continuing the example, let’s use the labels “email” and “city” to mark the respective nodes in the two-stage traversal. We do so by adding a label to the traversal chain. Recall that in the previous section, we wrote:

\fvset{hllines={, ,}}%
\begin{sphinxVerbatim}[commandchars=\\\{\}]
\PYG{n}{chained\PYGZus{}traversal} \PYG{o}{=} \PYG{n}{find\PYGZus{}email} \PYG{o}{\PYGZgt{}} \PYG{n}{find\PYGZus{}city}
\end{sphinxVerbatim}

whereas we now have:

\fvset{hllines={, ,}}%
\begin{sphinxVerbatim}[commandchars=\\\{\}]
\PYG{n}{chained\PYGZus{}traversal} \PYG{o}{=} \PYG{n}{find\PYGZus{}email} \PYG{o}{+} \PYG{l+s+s1}{\PYGZsq{}}\PYG{l+s+s1}{email}\PYG{l+s+s1}{\PYGZsq{}} \PYG{o}{\PYGZgt{}} \PYG{n}{find\PYGZus{}city} \PYG{o}{+} \PYG{l+s+s1}{\PYGZsq{}}\PYG{l+s+s1}{city}\PYG{l+s+s1}{\PYGZsq{}}
\end{sphinxVerbatim}

We use \sphinxcode{\sphinxupquote{+}} to add a label, and the label is just a string. Under the hood, Treehorn is instantiating a \sphinxcode{\sphinxupquote{Label}} object, but ordinarily, you shouldn’t have to do that directly.


\subsection{Relations}
\label{\detokenize{treehorn:relations}}
Lastly, we define a \sphinxcode{\sphinxupquote{Relation}} object to extract the data from our search. In this example, we might think of the search as returning data about people who live in a certain city. So we might name the \sphinxcode{\sphinxupquote{Relation}} “FROM\_CITY”.We’ll want to extract the value of the \sphinxcode{\sphinxupquote{email}} key from the node labeled with \sphinxcode{\sphinxupquote{email}}, and similarly with the \sphinxcode{\sphinxupquote{city}} node. This is accomplished by adding a little more syntax:

\fvset{hllines={, ,}}%
\begin{sphinxVerbatim}[commandchars=\\\{\}]
\PYG{n}{Relation}\PYG{p}{(}\PYG{l+s+s1}{\PYGZsq{}}\PYG{l+s+s1}{FROM\PYGZus{}CITY}\PYG{l+s+s1}{\PYGZsq{}}\PYG{p}{)} \PYG{o}{==} \PYG{p}{(}
    \PYG{p}{(}\PYG{n}{find\PYGZus{}email} \PYG{o}{+} \PYG{l+s+s1}{\PYGZsq{}}\PYG{l+s+s1}{email}\PYG{l+s+s1}{\PYGZsq{}}\PYG{p}{)}\PYG{p}{[}\PYG{l+s+s1}{\PYGZsq{}}\PYG{l+s+s1}{email}\PYG{l+s+s1}{\PYGZsq{}}\PYG{p}{]} \PYG{o}{\PYGZgt{}} \PYG{p}{(}\PYG{n}{find\PYGZus{}city} \PYG{o}{+} \PYG{l+s+s1}{\PYGZsq{}}\PYG{l+s+s1}{city}\PYG{l+s+s1}{\PYGZsq{}}\PYG{p}{)}\PYG{p}{[}\PYG{l+s+s1}{\PYGZsq{}}\PYG{l+s+s1}{city}\PYG{l+s+s1}{\PYGZsq{}}\PYG{p}{]}\PYG{p}{)}
\end{sphinxVerbatim}

After executing that statement, Treehorn will create an object named \sphinxcode{\sphinxupquote{FROM\_CITY}}, which can be called on a dictionary to yield the information we want, like so:

\fvset{hllines={, ,}}%
\begin{sphinxVerbatim}[commandchars=\\\{\}]
\PYG{k}{for} \PYG{n}{email\PYGZus{}city} \PYG{o+ow}{in} \PYG{n}{FROM\PYGZus{}CITY}\PYG{p}{(}\PYG{n}{api\PYGZus{}response}\PYG{p}{)}\PYG{p}{:}
    \PYG{n+nb}{print}\PYG{p}{(}\PYG{n}{email\PYGZus{}city}\PYG{p}{)}
\end{sphinxVerbatim}

which will give us:

\fvset{hllines={, ,}}%
\begin{sphinxVerbatim}[commandchars=\\\{\}]
\PYG{p}{\PYGZob{}}\PYG{l+s+s1}{\PYGZsq{}}\PYG{l+s+s1}{city}\PYG{l+s+s1}{\PYGZsq{}}\PYG{p}{:} \PYG{l+s+s1}{\PYGZsq{}}\PYG{l+s+s1}{Boston}\PYG{l+s+s1}{\PYGZsq{}}\PYG{p}{,} \PYG{l+s+s1}{\PYGZsq{}}\PYG{l+s+s1}{email}\PYG{l+s+s1}{\PYGZsq{}}\PYG{p}{:} \PYG{l+s+s1}{\PYGZsq{}}\PYG{l+s+s1}{alice@gmail.com}\PYG{l+s+s1}{\PYGZsq{}}\PYG{p}{\PYGZcb{}}
\PYG{p}{\PYGZob{}}\PYG{l+s+s1}{\PYGZsq{}}\PYG{l+s+s1}{city}\PYG{l+s+s1}{\PYGZsq{}}\PYG{p}{:} \PYG{l+s+s1}{\PYGZsq{}}\PYG{l+s+s1}{New York}\PYG{l+s+s1}{\PYGZsq{}}\PYG{p}{,} \PYG{l+s+s1}{\PYGZsq{}}\PYG{l+s+s1}{email}\PYG{l+s+s1}{\PYGZsq{}}\PYG{p}{:} \PYG{l+s+s1}{\PYGZsq{}}\PYG{l+s+s1}{bob@gmail.com}\PYG{l+s+s1}{\PYGZsq{}}\PYG{p}{\PYGZcb{}}
\end{sphinxVerbatim}

Voila!


\section{Summing up}
\label{\detokenize{treehorn:summing-up}}
Normally, ETL pipelines that extract data from dictionary-like objects involve a lot of loops and hard-coded keypaths. To accomplish the simple task of extracting emails and city names from our sample JSON blob, we’d probably hard-code paths for each specific key and value, and then we’d loop over various levels in the dictionary. This has several disadvantages:
\begin{enumerate}
\def\theenumi{\arabic{enumi}}
\def\labelenumi{\theenumi .}
\makeatletter\def\p@enumii{\p@enumi \theenumi .}\makeatother
\item {} 
It leads to brittle code. If the JSON blob changes structure in even very small ways, the hard-coded paths become obsolete and have to be rewritten.

\item {} 
The code is difficult to understand and debug. Given a whole bunch of nested loops and hard-coded keypaths, it’s very difficult to understand the intent of the code. Errors have to be found by painstakingly stepping through the execution.

\item {} 
It is very difficult to accommodate JSON blobs with variable structure. Some JSON blobs returned from APIs have unpredictable levels of nesting, for example. Therefore, keypaths cannot be hard-coded and recursive searches have to be written, which are inefficient and difficult to debug.

\end{enumerate}

The approach taken by Treehorn alleviates some of this pain. For example, the \sphinxcode{\sphinxupquote{GoDown}} traversal doesn’t care how many levels down in the tree it must search; so it is often able to cope with inconsistent structures (within reason) without any code changes. It’s also much easier to understand. You can tell from glancing at the code that the intention is to search for a dictionary with a key, and then search from there for lower-level dictionaries with another key, and return the results. Treehorn is also more efficient than writing loops and keypaths because all of its evaluations are lazy \textendash{} it doesn’t hold partial results in memory any longer than necessary because everything is yielded by generators.


\chapter{Implementation}
\label{\detokenize{implementation:implementation}}\label{\detokenize{implementation::doc}}
This section describes what’s happening under the hood in a \sphinxcode{\sphinxupquote{NanoStream}}
data pipeline. Most people won’t need to read this section.


\section{The data journey}
\label{\detokenize{implementation:the-data-journey}}
NanoStream pipelines are sets of \sphinxcode{\sphinxupquote{NanoNode}} objects connected by \sphinxcode{\sphinxupquote{NanoStreamQueue}}
objects. Think of each \sphinxcode{\sphinxupquote{NanoNode}} as a vertex in a directed graph, and each
\sphinxcode{\sphinxupquote{NanoStreamQueue}} as a directed edge.

There are two types of \sphinxcode{\sphinxupquote{NanoNode}} objects. A “source” is a \sphinxcode{\sphinxupquote{NanoNode}} that does not accept incoming data from another \sphinxcode{\sphinxupquote{NanoNode}}. A “processor” is any \sphinxcode{\sphinxupquote{NanoNode}} that is not a “source”. Note that there is nothing in the class definition or object that distinguishes between these two \textendash{} the only
difference is that processors have a \sphinxcode{\sphinxupquote{process\_item}} method, and sources have a \sphinxcode{\sphinxupquote{generator}} method. Other than that, they are identical.

The data journey begins with one or more source nodes. When a source node is started (by calling its \sphinxcode{\sphinxupquote{start}} method), a new thread is created and the node’s \sphinxcode{\sphinxupquote{generator}} method is executed inside the thread. As results from the \sphinxcode{\sphinxupquote{generator}} method are yielded, they are placed on each outgoing \sphinxcode{\sphinxupquote{NanoStreamQueue}} to be picked up by one or more processors downstream.

The data from the source’s \sphinxcode{\sphinxupquote{generator}} is handled by the \sphinxcode{\sphinxupquote{NanoStreamQueue}} object. At its heart, the \sphinxcode{\sphinxupquote{NanoStreamQueue}} is simply a class which has a Python \sphinxcode{\sphinxupquote{Queue.queue}} object as an attribute. The reason we don’t simply use Python \sphinxcode{\sphinxupquote{Queue}} objects is because the \sphinxcode{\sphinxupquote{NanoStreamQueue}} contains some logic that’s useful. In particular:
\begin{enumerate}
\def\theenumi{\arabic{enumi}}
\def\labelenumi{\theenumi .}
\makeatletter\def\p@enumii{\p@enumi \theenumi .}\makeatother
\item {} 
It wraps the data into a \sphinxcode{\sphinxupquote{NanoStreamMessage}} object, which also holds useful metadata including a UUID, the ID of the node that generated the data, and a timestamp.

\item {} 
If the \sphinxcode{\sphinxupquote{NanoStreamQueue}} receives data that is simply a \sphinxcode{\sphinxupquote{None}} object, then it is skipped.

\end{enumerate}


\chapter{API Documentation}
\label{\detokenize{api:module-nanostream.node}}\label{\detokenize{api:api-documentation}}\label{\detokenize{api::doc}}\index{nanostream.node (module)}

\section{Node module}
\label{\detokenize{api:node-module}}
The \sphinxcode{\sphinxupquote{node}} module contains the \sphinxcode{\sphinxupquote{NanoNode}} class, which is the foundation
for NanoStream.
\index{AggregateValues (class in nanostream.node)}

\begin{fulllineitems}
\phantomsection\label{\detokenize{api:nanostream.node.AggregateValues}}\pysiglinewithargsret{\sphinxbfcode{\sphinxupquote{class }}\sphinxcode{\sphinxupquote{nanostream.node.}}\sphinxbfcode{\sphinxupquote{AggregateValues}}}{\emph{values=False}, \emph{tail\_path=None}, \emph{**kwargs}}{}
Bases: {\hyperref[\detokenize{api:nanostream.node.NanoNode}]{\sphinxcrossref{\sphinxcode{\sphinxupquote{nanostream.node.NanoNode}}}}}

Does that.
\index{process\_item() (nanostream.node.AggregateValues method)}

\begin{fulllineitems}
\phantomsection\label{\detokenize{api:nanostream.node.AggregateValues.process_item}}\pysiglinewithargsret{\sphinxbfcode{\sphinxupquote{process\_item}}}{}{}
Default no-op for nodes.

\end{fulllineitems}


\end{fulllineitems}

\index{BatchMessages (class in nanostream.node)}

\begin{fulllineitems}
\phantomsection\label{\detokenize{api:nanostream.node.BatchMessages}}\pysiglinewithargsret{\sphinxbfcode{\sphinxupquote{class }}\sphinxcode{\sphinxupquote{nanostream.node.}}\sphinxbfcode{\sphinxupquote{BatchMessages}}}{\emph{batch\_size=None}, \emph{batch\_list=None}, \emph{counter=0}, \emph{timeout=5}, \emph{**kwargs}}{}
Bases: {\hyperref[\detokenize{api:nanostream.node.NanoNode}]{\sphinxcrossref{\sphinxcode{\sphinxupquote{nanostream.node.NanoNode}}}}}
\index{cleanup() (nanostream.node.BatchMessages method)}

\begin{fulllineitems}
\phantomsection\label{\detokenize{api:nanostream.node.BatchMessages.cleanup}}\pysiglinewithargsret{\sphinxbfcode{\sphinxupquote{cleanup}}}{}{}
If there is any cleanup (closing files, shutting down database connections),
necessary when the node is stopped, then the node’s class should provide
a \sphinxcode{\sphinxupquote{cleanup}} method. By default, the method is just a logging statement.

\end{fulllineitems}

\index{process\_item() (nanostream.node.BatchMessages method)}

\begin{fulllineitems}
\phantomsection\label{\detokenize{api:nanostream.node.BatchMessages.process_item}}\pysiglinewithargsret{\sphinxbfcode{\sphinxupquote{process\_item}}}{}{}
Default no-op for nodes.

\end{fulllineitems}


\end{fulllineitems}

\index{CSVReader (class in nanostream.node)}

\begin{fulllineitems}
\phantomsection\label{\detokenize{api:nanostream.node.CSVReader}}\pysiglinewithargsret{\sphinxbfcode{\sphinxupquote{class }}\sphinxcode{\sphinxupquote{nanostream.node.}}\sphinxbfcode{\sphinxupquote{CSVReader}}}{\emph{*args}, \emph{**kwargs}}{}
Bases: {\hyperref[\detokenize{api:nanostream.node.NanoNode}]{\sphinxcrossref{\sphinxcode{\sphinxupquote{nanostream.node.NanoNode}}}}}
\index{process\_item() (nanostream.node.CSVReader method)}

\begin{fulllineitems}
\phantomsection\label{\detokenize{api:nanostream.node.CSVReader.process_item}}\pysiglinewithargsret{\sphinxbfcode{\sphinxupquote{process\_item}}}{}{}
Default no-op for nodes.

\end{fulllineitems}


\end{fulllineitems}

\index{CSVToDictionaryList (class in nanostream.node)}

\begin{fulllineitems}
\phantomsection\label{\detokenize{api:nanostream.node.CSVToDictionaryList}}\pysiglinewithargsret{\sphinxbfcode{\sphinxupquote{class }}\sphinxcode{\sphinxupquote{nanostream.node.}}\sphinxbfcode{\sphinxupquote{CSVToDictionaryList}}}{\emph{**kwargs}}{}
Bases: {\hyperref[\detokenize{api:nanostream.node.NanoNode}]{\sphinxcrossref{\sphinxcode{\sphinxupquote{nanostream.node.NanoNode}}}}}
\index{process\_item() (nanostream.node.CSVToDictionaryList method)}

\begin{fulllineitems}
\phantomsection\label{\detokenize{api:nanostream.node.CSVToDictionaryList.process_item}}\pysiglinewithargsret{\sphinxbfcode{\sphinxupquote{process\_item}}}{}{}
Default no-op for nodes.

\end{fulllineitems}


\end{fulllineitems}

\index{ConstantEmitter (class in nanostream.node)}

\begin{fulllineitems}
\phantomsection\label{\detokenize{api:nanostream.node.ConstantEmitter}}\pysiglinewithargsret{\sphinxbfcode{\sphinxupquote{class }}\sphinxcode{\sphinxupquote{nanostream.node.}}\sphinxbfcode{\sphinxupquote{ConstantEmitter}}}{\emph{thing=None}, \emph{delay=2}, \emph{**kwargs}}{}
Bases: {\hyperref[\detokenize{api:nanostream.node.NanoNode}]{\sphinxcrossref{\sphinxcode{\sphinxupquote{nanostream.node.NanoNode}}}}}

Send a thing every n seconds
\index{generator() (nanostream.node.ConstantEmitter method)}

\begin{fulllineitems}
\phantomsection\label{\detokenize{api:nanostream.node.ConstantEmitter.generator}}\pysiglinewithargsret{\sphinxbfcode{\sphinxupquote{generator}}}{}{}
\end{fulllineitems}


\end{fulllineitems}

\index{CounterOfThings (class in nanostream.node)}

\begin{fulllineitems}
\phantomsection\label{\detokenize{api:nanostream.node.CounterOfThings}}\pysiglinewithargsret{\sphinxbfcode{\sphinxupquote{class }}\sphinxcode{\sphinxupquote{nanostream.node.}}\sphinxbfcode{\sphinxupquote{CounterOfThings}}}{\emph{*args}, \emph{batch=False}, \emph{get\_runtime\_attrs=\textless{}function no\_op\textgreater{}}, \emph{get\_runtime\_attrs\_args=None}, \emph{get\_runtime\_attrs\_kwargs=None}, \emph{runtime\_attrs\_destinations=None}, \emph{input\_mapping=None}, \emph{retain\_input=True}, \emph{throttle=0}, \emph{keep\_alive=True}, \emph{max\_errors=0}, \emph{name=None}, \emph{input\_message\_keypath=None}, \emph{key=None}, \emph{messages\_received\_counter=0}, \emph{prefer\_existing\_value=False}, \emph{messages\_sent\_counter=0}, \emph{post\_process\_function=None}, \emph{post\_process\_keypath=None}, \emph{summary=''}, \emph{post\_process\_function\_kwargs=None}, \emph{output\_key=None}, \emph{**kwargs}}{}
Bases: {\hyperref[\detokenize{api:nanostream.node.NanoNode}]{\sphinxcrossref{\sphinxcode{\sphinxupquote{nanostream.node.NanoNode}}}}}
\index{foo\_\_init\_\_() (nanostream.node.CounterOfThings method)}

\begin{fulllineitems}
\phantomsection\label{\detokenize{api:nanostream.node.CounterOfThings.foo__init__}}\pysiglinewithargsret{\sphinxbfcode{\sphinxupquote{foo\_\_init\_\_}}}{\emph{*args}, \emph{start=0}, \emph{end=None}, \emph{**kwargs}}{}
\end{fulllineitems}

\index{generator() (nanostream.node.CounterOfThings method)}

\begin{fulllineitems}
\phantomsection\label{\detokenize{api:nanostream.node.CounterOfThings.generator}}\pysiglinewithargsret{\sphinxbfcode{\sphinxupquote{generator}}}{}{}
Just start counting integers

\end{fulllineitems}


\end{fulllineitems}

\index{DynamicClassMediator (class in nanostream.node)}

\begin{fulllineitems}
\phantomsection\label{\detokenize{api:nanostream.node.DynamicClassMediator}}\pysiglinewithargsret{\sphinxbfcode{\sphinxupquote{class }}\sphinxcode{\sphinxupquote{nanostream.node.}}\sphinxbfcode{\sphinxupquote{DynamicClassMediator}}}{\emph{*args}, \emph{**kwargs}}{}
Bases: {\hyperref[\detokenize{api:nanostream.node.NanoNode}]{\sphinxcrossref{\sphinxcode{\sphinxupquote{nanostream.node.NanoNode}}}}}
\index{get\_sink() (nanostream.node.DynamicClassMediator method)}

\begin{fulllineitems}
\phantomsection\label{\detokenize{api:nanostream.node.DynamicClassMediator.get_sink}}\pysiglinewithargsret{\sphinxbfcode{\sphinxupquote{get\_sink}}}{}{}
\end{fulllineitems}

\index{get\_source() (nanostream.node.DynamicClassMediator method)}

\begin{fulllineitems}
\phantomsection\label{\detokenize{api:nanostream.node.DynamicClassMediator.get_source}}\pysiglinewithargsret{\sphinxbfcode{\sphinxupquote{get\_source}}}{}{}
\end{fulllineitems}

\index{hi() (nanostream.node.DynamicClassMediator method)}

\begin{fulllineitems}
\phantomsection\label{\detokenize{api:nanostream.node.DynamicClassMediator.hi}}\pysiglinewithargsret{\sphinxbfcode{\sphinxupquote{hi}}}{}{}
\end{fulllineitems}

\index{sink\_list() (nanostream.node.DynamicClassMediator method)}

\begin{fulllineitems}
\phantomsection\label{\detokenize{api:nanostream.node.DynamicClassMediator.sink_list}}\pysiglinewithargsret{\sphinxbfcode{\sphinxupquote{sink\_list}}}{}{}
\end{fulllineitems}

\index{source\_list() (nanostream.node.DynamicClassMediator method)}

\begin{fulllineitems}
\phantomsection\label{\detokenize{api:nanostream.node.DynamicClassMediator.source_list}}\pysiglinewithargsret{\sphinxbfcode{\sphinxupquote{source\_list}}}{}{}
\end{fulllineitems}


\end{fulllineitems}

\index{Filter (class in nanostream.node)}

\begin{fulllineitems}
\phantomsection\label{\detokenize{api:nanostream.node.Filter}}\pysiglinewithargsret{\sphinxbfcode{\sphinxupquote{class }}\sphinxcode{\sphinxupquote{nanostream.node.}}\sphinxbfcode{\sphinxupquote{Filter}}}{\emph{test=None}, \emph{test\_keypath=None}, \emph{value=True}, \emph{*args}, \emph{**kwargs}}{}
Bases: {\hyperref[\detokenize{api:nanostream.node.NanoNode}]{\sphinxcrossref{\sphinxcode{\sphinxupquote{nanostream.node.NanoNode}}}}}

Applies tests to each message and filters out messages that don’t pass
\begin{description}
\item[{Built-in tests:}] \leavevmode
key\_exists
value\_is\_true
value\_is\_not\_none

\item[{Example:}] \leavevmode\begin{description}
\item[{\{‘test’: ‘key\_exists’,}] \leavevmode
‘key’: mykey\}

\end{description}

\end{description}
\index{process\_item() (nanostream.node.Filter method)}

\begin{fulllineitems}
\phantomsection\label{\detokenize{api:nanostream.node.Filter.process_item}}\pysiglinewithargsret{\sphinxbfcode{\sphinxupquote{process\_item}}}{}{}
Default no-op for nodes.

\end{fulllineitems}


\end{fulllineitems}

\index{GetEnvironmentVariables (class in nanostream.node)}

\begin{fulllineitems}
\phantomsection\label{\detokenize{api:nanostream.node.GetEnvironmentVariables}}\pysiglinewithargsret{\sphinxbfcode{\sphinxupquote{class }}\sphinxcode{\sphinxupquote{nanostream.node.}}\sphinxbfcode{\sphinxupquote{GetEnvironmentVariables}}}{\emph{mappings=None}, \emph{environment\_variables=None}, \emph{**kwargs}}{}
Bases: {\hyperref[\detokenize{api:nanostream.node.NanoNode}]{\sphinxcrossref{\sphinxcode{\sphinxupquote{nanostream.node.NanoNode}}}}}
\index{generator() (nanostream.node.GetEnvironmentVariables method)}

\begin{fulllineitems}
\phantomsection\label{\detokenize{api:nanostream.node.GetEnvironmentVariables.generator}}\pysiglinewithargsret{\sphinxbfcode{\sphinxupquote{generator}}}{}{}
\end{fulllineitems}

\index{process\_item() (nanostream.node.GetEnvironmentVariables method)}

\begin{fulllineitems}
\phantomsection\label{\detokenize{api:nanostream.node.GetEnvironmentVariables.process_item}}\pysiglinewithargsret{\sphinxbfcode{\sphinxupquote{process\_item}}}{}{}
Default no-op for nodes.

\end{fulllineitems}


\end{fulllineitems}

\index{InsertData (class in nanostream.node)}

\begin{fulllineitems}
\phantomsection\label{\detokenize{api:nanostream.node.InsertData}}\pysiglinewithargsret{\sphinxbfcode{\sphinxupquote{class }}\sphinxcode{\sphinxupquote{nanostream.node.}}\sphinxbfcode{\sphinxupquote{InsertData}}}{\emph{overwrite=True}, \emph{overwrite\_if\_null=True}, \emph{value\_dict=None}, \emph{**kwargs}}{}
Bases: {\hyperref[\detokenize{api:nanostream.node.NanoNode}]{\sphinxcrossref{\sphinxcode{\sphinxupquote{nanostream.node.NanoNode}}}}}
\index{process\_item() (nanostream.node.InsertData method)}

\begin{fulllineitems}
\phantomsection\label{\detokenize{api:nanostream.node.InsertData.process_item}}\pysiglinewithargsret{\sphinxbfcode{\sphinxupquote{process\_item}}}{}{}
Default no-op for nodes.

\end{fulllineitems}


\end{fulllineitems}

\index{LocalDirectoryWatchdog (class in nanostream.node)}

\begin{fulllineitems}
\phantomsection\label{\detokenize{api:nanostream.node.LocalDirectoryWatchdog}}\pysiglinewithargsret{\sphinxbfcode{\sphinxupquote{class }}\sphinxcode{\sphinxupquote{nanostream.node.}}\sphinxbfcode{\sphinxupquote{LocalDirectoryWatchdog}}}{\emph{directory='.'}, \emph{check\_interval=3}, \emph{**kwargs}}{}
Bases: {\hyperref[\detokenize{api:nanostream.node.NanoNode}]{\sphinxcrossref{\sphinxcode{\sphinxupquote{nanostream.node.NanoNode}}}}}
\index{generator() (nanostream.node.LocalDirectoryWatchdog method)}

\begin{fulllineitems}
\phantomsection\label{\detokenize{api:nanostream.node.LocalDirectoryWatchdog.generator}}\pysiglinewithargsret{\sphinxbfcode{\sphinxupquote{generator}}}{}{}
\end{fulllineitems}


\end{fulllineitems}

\index{LocalFileReader (class in nanostream.node)}

\begin{fulllineitems}
\phantomsection\label{\detokenize{api:nanostream.node.LocalFileReader}}\pysiglinewithargsret{\sphinxbfcode{\sphinxupquote{class }}\sphinxcode{\sphinxupquote{nanostream.node.}}\sphinxbfcode{\sphinxupquote{LocalFileReader}}}{\emph{*args}, \emph{**kwargs}}{}
Bases: {\hyperref[\detokenize{api:nanostream.node.NanoNode}]{\sphinxcrossref{\sphinxcode{\sphinxupquote{nanostream.node.NanoNode}}}}}
\index{process\_item() (nanostream.node.LocalFileReader method)}

\begin{fulllineitems}
\phantomsection\label{\detokenize{api:nanostream.node.LocalFileReader.process_item}}\pysiglinewithargsret{\sphinxbfcode{\sphinxupquote{process\_item}}}{}{}
Default no-op for nodes.

\end{fulllineitems}


\end{fulllineitems}

\index{NanoNode (class in nanostream.node)}

\begin{fulllineitems}
\phantomsection\label{\detokenize{api:nanostream.node.NanoNode}}\pysiglinewithargsret{\sphinxbfcode{\sphinxupquote{class }}\sphinxcode{\sphinxupquote{nanostream.node.}}\sphinxbfcode{\sphinxupquote{NanoNode}}}{\emph{*args}, \emph{batch=False}, \emph{get\_runtime\_attrs=\textless{}function no\_op\textgreater{}}, \emph{get\_runtime\_attrs\_args=None}, \emph{get\_runtime\_attrs\_kwargs=None}, \emph{runtime\_attrs\_destinations=None}, \emph{input\_mapping=None}, \emph{retain\_input=True}, \emph{throttle=0}, \emph{keep\_alive=True}, \emph{max\_errors=0}, \emph{name=None}, \emph{input\_message\_keypath=None}, \emph{key=None}, \emph{messages\_received\_counter=0}, \emph{prefer\_existing\_value=False}, \emph{messages\_sent\_counter=0}, \emph{post\_process\_function=None}, \emph{post\_process\_keypath=None}, \emph{summary=''}, \emph{post\_process\_function\_kwargs=None}, \emph{output\_key=None}, \emph{**kwargs}}{}
Bases: \sphinxcode{\sphinxupquote{object}}

The foundational class of \sphinxtitleref{NanoStream}. This class is inherited by all
nodes in a computation graph.

Order of operations:
1. Child class \sphinxcode{\sphinxupquote{\_\_init\_\_}} function
2. \sphinxcode{\sphinxupquote{NanoNode}} \sphinxcode{\sphinxupquote{\_\_init\_\_}} function
3. \sphinxcode{\sphinxupquote{preflight\_function}} (Specified in initialization params)
4. \sphinxcode{\sphinxupquote{setup}}
5. start

These methods have the following intended uses:
\begin{enumerate}
\def\theenumi{\arabic{enumi}}
\def\labelenumi{\theenumi .}
\makeatletter\def\p@enumii{\p@enumi \theenumi .}\makeatother
\item {} 
\sphinxcode{\sphinxupquote{\_\_init\_\_}} Sets attribute values and calls the \sphinxcode{\sphinxupquote{NanoNode}} \sphinxcode{\sphinxupquote{\_\_init\_\_}}
method.

\item {} 
\sphinxcode{\sphinxupquote{get\_runtime\_attrs}} Sets any attribute values that are to be determined
at runtime, e.g. by checking environment variables or reading values
from a database. The \sphinxcode{\sphinxupquote{get\_runtime\_attrs}} should return a dictionary
of attributes -\textgreater{} values, or else \sphinxcode{\sphinxupquote{None}}.

\item {} 
\sphinxcode{\sphinxupquote{setup}} Sets the state of the \sphinxcode{\sphinxupquote{NanoNode}} and/or creates any attributes
that require information available only at runtime.

\end{enumerate}
\begin{quote}\begin{description}
\item[{Variables}] \leavevmode\begin{itemize}
\item {} 
\sphinxstyleliteralstrong{\sphinxupquote{send\_batch\_markers}} \textendash{} If \sphinxcode{\sphinxupquote{True}}, then a \sphinxcode{\sphinxupquote{BatchStart}} marker will
be sent when a new input is received, and a \sphinxcode{\sphinxupquote{BatchEnd}} will be sent
after the input has been processed. The intention is that a number of
items will be emitted for each input received. For example, we might
emit a table row-by-row for each input.

\item {} 
\sphinxstyleliteralstrong{\sphinxupquote{get\_runtime\_attrs}} \textendash{} A function that returns a dictionary-like object.
The keys and values will be saved to this \sphinxcode{\sphinxupquote{NanoNode}} object’s
attributes. The function is executed one time, upon starting the node.

\item {} 
\sphinxstyleliteralstrong{\sphinxupquote{get\_runtime\_attrs\_args}} \textendash{} A tuple of arguments to be passed to the
\sphinxcode{\sphinxupquote{get\_runtime\_attrs}} function upon starting the node.

\item {} 
\sphinxstyleliteralstrong{\sphinxupquote{get\_runtime\_attrs\_kwargs}} \textendash{} A dictionary of kwargs passed to the
\sphinxcode{\sphinxupquote{get\_runtime\_attrs}} function.

\item {} 
\sphinxstyleliteralstrong{\sphinxupquote{runtime\_attrs\_destinations}} \textendash{} If set, this is a dictionary mapping
the keys returned from the \sphinxcode{\sphinxupquote{get\_runtime\_attrs}} function to the
names of the attributes to which the values will be saved.

\item {} 
\sphinxstyleliteralstrong{\sphinxupquote{throttle}} \textendash{} For each input received, a delay of \sphinxcode{\sphinxupquote{throttle}} seconds
will be added.

\item {} 
\sphinxstyleliteralstrong{\sphinxupquote{keep\_alive}} \textendash{} If \sphinxcode{\sphinxupquote{True}}, keep the node’s thread alive after
everything has been processed.

\item {} 
\sphinxstyleliteralstrong{\sphinxupquote{name}} \textendash{} The name of the node. Defaults to a randomly generated hash.
Note that this hash is not consistent from one run to the next.

\item {} 
\sphinxstyleliteralstrong{\sphinxupquote{input\_mapping}} \textendash{} When the node receives a dictionary-like object,
this dictionary will cause the keys of the dictionary to be remapped
to new keys.

\item {} 
\sphinxstyleliteralstrong{\sphinxupquote{retain\_input}} \textendash{} If \sphinxcode{\sphinxupquote{True}}, then combine the dictionary-like input
with the output. If keys clash, the output value will be kept.

\item {} 
\sphinxstyleliteralstrong{\sphinxupquote{input\_message\_keypath}} \textendash{} Read the value in this keypath as the content
of the incoming message.

\end{itemize}

\end{description}\end{quote}
\index{add\_edge() (nanostream.node.NanoNode method)}

\begin{fulllineitems}
\phantomsection\label{\detokenize{api:nanostream.node.NanoNode.add_edge}}\pysiglinewithargsret{\sphinxbfcode{\sphinxupquote{add\_edge}}}{\emph{target}, \emph{**kwargs}}{}
Create an edge connecting \sphinxtitleref{self} to \sphinxtitleref{target}.

This method instantiates the \sphinxcode{\sphinxupquote{NanoStreamQueue}} object that connects the
nodes. Connecting the nodes together consists in (1) adding the queue to 
the other’s \sphinxcode{\sphinxupquote{input\_queue\_list}} or \sphinxcode{\sphinxupquote{output\_queue\_list}} and (2) setting
the queue’s \sphinxcode{\sphinxupquote{source\_node}} and \sphinxcode{\sphinxupquote{target\_node}} attributes.
\begin{description}
\item[{Args:}] \leavevmode
target (\sphinxcode{\sphinxupquote{NanoNode}}): The node to which \sphinxcode{\sphinxupquote{self}} will be connected.

\item[{Returns:}] \leavevmode
None

\end{description}

\end{fulllineitems}

\index{all\_connected() (nanostream.node.NanoNode method)}

\begin{fulllineitems}
\phantomsection\label{\detokenize{api:nanostream.node.NanoNode.all_connected}}\pysiglinewithargsret{\sphinxbfcode{\sphinxupquote{all\_connected}}}{\emph{seen=None}}{}
Returns all the nodes connected (directly or indirectly) to \sphinxcode{\sphinxupquote{self}}.
This allows us to loop over all the nodes in a pipeline even if we
have a handle on only one. This is used by \sphinxcode{\sphinxupquote{global\_start}}, for 
example.
\begin{description}
\item[{Args:}] \leavevmode\begin{description}
\item[{seen (set): A set of all the nodes that have been identified as}] \leavevmode
connected to \sphinxcode{\sphinxupquote{self}}.

\end{description}

\item[{Returns:}] \leavevmode\begin{description}
\item[{(set of \sphinxcode{\sphinxupquote{NanoNode}}): All the nodes connected to \sphinxcode{\sphinxupquote{self}}. This}] \leavevmode
includes \sphinxcode{\sphinxupquote{self}}.

\end{description}

\end{description}

\end{fulllineitems}

\index{broadcast() (nanostream.node.NanoNode method)}

\begin{fulllineitems}
\phantomsection\label{\detokenize{api:nanostream.node.NanoNode.broadcast}}\pysiglinewithargsret{\sphinxbfcode{\sphinxupquote{broadcast}}}{\emph{broadcast\_message}}{}
Puts the message into all the input queues for all connected nodes.

\end{fulllineitems}

\index{cleanup() (nanostream.node.NanoNode method)}

\begin{fulllineitems}
\phantomsection\label{\detokenize{api:nanostream.node.NanoNode.cleanup}}\pysiglinewithargsret{\sphinxbfcode{\sphinxupquote{cleanup}}}{}{}
If there is any cleanup (closing files, shutting down database connections),
necessary when the node is stopped, then the node’s class should provide
a \sphinxcode{\sphinxupquote{cleanup}} method. By default, the method is just a logging statement.

\end{fulllineitems}

\index{draw\_pipeline() (nanostream.node.NanoNode method)}

\begin{fulllineitems}
\phantomsection\label{\detokenize{api:nanostream.node.NanoNode.draw_pipeline}}\pysiglinewithargsret{\sphinxbfcode{\sphinxupquote{draw\_pipeline}}}{}{}
Draw the pipeline structure using graphviz.

\end{fulllineitems}

\index{global\_start() (nanostream.node.NanoNode method)}

\begin{fulllineitems}
\phantomsection\label{\detokenize{api:nanostream.node.NanoNode.global_start}}\pysiglinewithargsret{\sphinxbfcode{\sphinxupquote{global\_start}}}{\emph{datadog=False}, \emph{prometheus=False}, \emph{pipeline\_name=None}}{}
Starts every node connected to \sphinxcode{\sphinxupquote{self}}. Mainly, it:
\begin{enumerate}
\def\theenumi{\arabic{enumi}}
\def\labelenumi{\theenumi .}
\makeatletter\def\p@enumii{\p@enumi \theenumi .}\makeatother
\item {} 
calls \sphinxcode{\sphinxupquote{start()}} on each node

\item {} 
sets some global variables

\item {} 
optionally starts some experimental code for monitoring

\end{enumerate}

\end{fulllineitems}

\index{input\_queue\_size (nanostream.node.NanoNode attribute)}

\begin{fulllineitems}
\phantomsection\label{\detokenize{api:nanostream.node.NanoNode.input_queue_size}}\pysigline{\sphinxbfcode{\sphinxupquote{input\_queue\_size}}}
Return the total number of items in all of the queues that are inputs
to this node.

\end{fulllineitems}

\index{is\_sink (nanostream.node.NanoNode attribute)}

\begin{fulllineitems}
\phantomsection\label{\detokenize{api:nanostream.node.NanoNode.is_sink}}\pysigline{\sphinxbfcode{\sphinxupquote{is\_sink}}}
Tests whether the node is a sink or not, i.e. whether there are no
outputs from the node.
\begin{description}
\item[{Returns:}] \leavevmode
(bool): \sphinxcode{\sphinxupquote{True}} if the node has no output nodes, \sphinxcode{\sphinxupquote{False}} otherwise.

\end{description}

\end{fulllineitems}

\index{is\_source (nanostream.node.NanoNode attribute)}

\begin{fulllineitems}
\phantomsection\label{\detokenize{api:nanostream.node.NanoNode.is_source}}\pysigline{\sphinxbfcode{\sphinxupquote{is\_source}}}
Tests whether the node is a source or not, i.e. whether there are no
inputs to the node.
\begin{description}
\item[{Returns:}] \leavevmode
(bool): \sphinxcode{\sphinxupquote{True}} if the node has no inputs, \sphinxcode{\sphinxupquote{False}} otherwise.

\end{description}

\end{fulllineitems}

\index{kill\_pipeline() (nanostream.node.NanoNode method)}

\begin{fulllineitems}
\phantomsection\label{\detokenize{api:nanostream.node.NanoNode.kill_pipeline}}\pysiglinewithargsret{\sphinxbfcode{\sphinxupquote{kill\_pipeline}}}{}{}
\end{fulllineitems}

\index{log\_info() (nanostream.node.NanoNode method)}

\begin{fulllineitems}
\phantomsection\label{\detokenize{api:nanostream.node.NanoNode.log_info}}\pysiglinewithargsret{\sphinxbfcode{\sphinxupquote{log\_info}}}{\emph{message=''}}{}
\end{fulllineitems}

\index{logjam (nanostream.node.NanoNode attribute)}

\begin{fulllineitems}
\phantomsection\label{\detokenize{api:nanostream.node.NanoNode.logjam}}\pysigline{\sphinxbfcode{\sphinxupquote{logjam}}}
Returns the logjam score, which measures the degree to which the
node is holding up progress in downstream nodes.

We’re defining a logjam as a
node whose input queue is full, but whose output queue(s) is not.
More specifically, we poll each node in the \sphinxcode{\sphinxupquote{monitor\_thread}},
and increment a counter if the node is a logjam at that time. This
property returns the percentage of samples in which the node is a
logjam. Our intention is that if this score exceeds a threshold,
the user is alerted, or the load is rebalanced somehow (not yet
implemented).
\begin{description}
\item[{Returns:}] \leavevmode
(float): Logjam score

\end{description}

\end{fulllineitems}

\index{process\_item() (nanostream.node.NanoNode method)}

\begin{fulllineitems}
\phantomsection\label{\detokenize{api:nanostream.node.NanoNode.process_item}}\pysiglinewithargsret{\sphinxbfcode{\sphinxupquote{process\_item}}}{\emph{*args}, \emph{**kwargs}}{}
Default no-op for nodes.

\end{fulllineitems}

\index{processor\_bak() (nanostream.node.NanoNode method)}

\begin{fulllineitems}
\phantomsection\label{\detokenize{api:nanostream.node.NanoNode.processor_bak}}\pysiglinewithargsret{\sphinxbfcode{\sphinxupquote{processor\_bak}}}{}{}
This calls the user’s \sphinxcode{\sphinxupquote{process\_item}} with just the message content,
and then returns the full message.

I think this is deprecated, which is why it’s been renamed to \sphinxcode{\sphinxupquote{processor\_bak}}.

\end{fulllineitems}

\index{setup() (nanostream.node.NanoNode method)}

\begin{fulllineitems}
\phantomsection\label{\detokenize{api:nanostream.node.NanoNode.setup}}\pysiglinewithargsret{\sphinxbfcode{\sphinxupquote{setup}}}{}{}
For classes that require initialization at runtime, which can’t be done
when the class’s \sphinxcode{\sphinxupquote{\_\_init\_\_}} function is called. The \sphinxcode{\sphinxupquote{NanoNode}} base
class’s setup function is just a logging call.

It should be unusual to have to make use of \sphinxcode{\sphinxupquote{setup}} because in practice,
initialization can be done in the \sphinxcode{\sphinxupquote{\_\_init\_\_}} function.

\end{fulllineitems}

\index{start() (nanostream.node.NanoNode method)}

\begin{fulllineitems}
\phantomsection\label{\detokenize{api:nanostream.node.NanoNode.start}}\pysiglinewithargsret{\sphinxbfcode{\sphinxupquote{start}}}{}{}
Starts the node. This is called by \sphinxcode{\sphinxupquote{NanoNode.global\_start()}}.

The node’s main loop is contained in this method. The main loop does
the following:
\begin{enumerate}
\def\theenumi{\arabic{enumi}}
\def\labelenumi{\theenumi .}
\makeatletter\def\p@enumii{\p@enumi \theenumi .}\makeatother
\item {} 
records the timestamp to the node’s \sphinxcode{\sphinxupquote{started\_at}} attribute.

\item {} 
calls \sphinxcode{\sphinxupquote{get\_runtime\_attrs}} (TODO: check if we can deprecate this)

\item {} 
calls the \sphinxcode{\sphinxupquote{setup}} method for the class (which is a no-op by default)

\item {} 
if the node is a source, then successively yield all the results of
the node’s \sphinxcode{\sphinxupquote{generator}} method, then exit.

\item {} 
if the node is not a source, then loop over the input queues, getting
the next message. Note that when the message is pulled from the queue,
the \sphinxcode{\sphinxupquote{NanoStreamQueue}} yields it as a dictionary.

\item {} 
gets either the content of the entire message if the node has no \sphinxcode{\sphinxupquote{key}}
attribute, or the value of \sphinxcode{\sphinxupquote{message{[}self.key{]}}}.

\item {} 
remaps the message content if a \sphinxcode{\sphinxupquote{remapping}} dictionary has been
given in the node’s configuration

\item {} 
calls the node’s \sphinxcode{\sphinxupquote{process\_item}} method, yielding back the results.
(Note that a single input message may cause the node to yield zero,
one, or more than one output message.)

\item {} 
places the results into each of the node’s output queues.

\end{enumerate}

\end{fulllineitems}

\index{stream() (nanostream.node.NanoNode method)}

\begin{fulllineitems}
\phantomsection\label{\detokenize{api:nanostream.node.NanoNode.stream}}\pysiglinewithargsret{\sphinxbfcode{\sphinxupquote{stream}}}{}{}
Called in each \sphinxcode{\sphinxupquote{NanoNode}} thread.

\end{fulllineitems}

\index{terminate\_pipeline() (nanostream.node.NanoNode method)}

\begin{fulllineitems}
\phantomsection\label{\detokenize{api:nanostream.node.NanoNode.terminate_pipeline}}\pysiglinewithargsret{\sphinxbfcode{\sphinxupquote{terminate\_pipeline}}}{\emph{error=False}}{}
This method can be called on any node in a pipeline, and it will cause
all of the nodes to terminate if they haven’t stopped already.
\begin{description}
\item[{Args:}] \leavevmode
error (bool): Not yet implemented.

\end{description}

\end{fulllineitems}

\index{thread\_monitor() (nanostream.node.NanoNode method)}

\begin{fulllineitems}
\phantomsection\label{\detokenize{api:nanostream.node.NanoNode.thread_monitor}}\pysiglinewithargsret{\sphinxbfcode{\sphinxupquote{thread\_monitor}}}{}{}
This function loops over all of the threads in the pipeline, checking
that they are either \sphinxcode{\sphinxupquote{finished}} or \sphinxcode{\sphinxupquote{running}}. If any have had an
abnormal exit, terminate the entire pipeline.

\end{fulllineitems}

\index{time\_running (nanostream.node.NanoNode attribute)}

\begin{fulllineitems}
\phantomsection\label{\detokenize{api:nanostream.node.NanoNode.time_running}}\pysigline{\sphinxbfcode{\sphinxupquote{time\_running}}}
Return the number of wall-clock seconds elapsed since the node was
started.

\end{fulllineitems}


\end{fulllineitems}

\index{NothingToSeeHere (class in nanostream.node)}

\begin{fulllineitems}
\phantomsection\label{\detokenize{api:nanostream.node.NothingToSeeHere}}\pysigline{\sphinxbfcode{\sphinxupquote{class }}\sphinxcode{\sphinxupquote{nanostream.node.}}\sphinxbfcode{\sphinxupquote{NothingToSeeHere}}}
Bases: \sphinxcode{\sphinxupquote{object}}

Vacuous class used as a no-op message type.

\end{fulllineitems}

\index{PrinterOfThings (class in nanostream.node)}

\begin{fulllineitems}
\phantomsection\label{\detokenize{api:nanostream.node.PrinterOfThings}}\pysiglinewithargsret{\sphinxbfcode{\sphinxupquote{class }}\sphinxcode{\sphinxupquote{nanostream.node.}}\sphinxbfcode{\sphinxupquote{PrinterOfThings}}}{\emph{*args}, \emph{**kwargs}}{}
Bases: {\hyperref[\detokenize{api:nanostream.node.NanoNode}]{\sphinxcrossref{\sphinxcode{\sphinxupquote{nanostream.node.NanoNode}}}}}
\index{process\_item() (nanostream.node.PrinterOfThings method)}

\begin{fulllineitems}
\phantomsection\label{\detokenize{api:nanostream.node.PrinterOfThings.process_item}}\pysiglinewithargsret{\sphinxbfcode{\sphinxupquote{process\_item}}}{}{}
Default no-op for nodes.

\end{fulllineitems}


\end{fulllineitems}

\index{RandomSample (class in nanostream.node)}

\begin{fulllineitems}
\phantomsection\label{\detokenize{api:nanostream.node.RandomSample}}\pysiglinewithargsret{\sphinxbfcode{\sphinxupquote{class }}\sphinxcode{\sphinxupquote{nanostream.node.}}\sphinxbfcode{\sphinxupquote{RandomSample}}}{\emph{sample=0.1}}{}
Bases: {\hyperref[\detokenize{api:nanostream.node.NanoNode}]{\sphinxcrossref{\sphinxcode{\sphinxupquote{nanostream.node.NanoNode}}}}}

Lets through only a random sample of incoming messages. Might be useful
for testing, or when only approximate results are necessary.
\index{process\_item() (nanostream.node.RandomSample method)}

\begin{fulllineitems}
\phantomsection\label{\detokenize{api:nanostream.node.RandomSample.process_item}}\pysiglinewithargsret{\sphinxbfcode{\sphinxupquote{process\_item}}}{}{}
Default no-op for nodes.

\end{fulllineitems}


\end{fulllineitems}

\index{Remapper (class in nanostream.node)}

\begin{fulllineitems}
\phantomsection\label{\detokenize{api:nanostream.node.Remapper}}\pysiglinewithargsret{\sphinxbfcode{\sphinxupquote{class }}\sphinxcode{\sphinxupquote{nanostream.node.}}\sphinxbfcode{\sphinxupquote{Remapper}}}{\emph{mapping=None}, \emph{**kwargs}}{}
Bases: {\hyperref[\detokenize{api:nanostream.node.NanoNode}]{\sphinxcrossref{\sphinxcode{\sphinxupquote{nanostream.node.NanoNode}}}}}
\index{process\_item() (nanostream.node.Remapper method)}

\begin{fulllineitems}
\phantomsection\label{\detokenize{api:nanostream.node.Remapper.process_item}}\pysiglinewithargsret{\sphinxbfcode{\sphinxupquote{process\_item}}}{}{}
Default no-op for nodes.

\end{fulllineitems}


\end{fulllineitems}

\index{SequenceEmitter (class in nanostream.node)}

\begin{fulllineitems}
\phantomsection\label{\detokenize{api:nanostream.node.SequenceEmitter}}\pysiglinewithargsret{\sphinxbfcode{\sphinxupquote{class }}\sphinxcode{\sphinxupquote{nanostream.node.}}\sphinxbfcode{\sphinxupquote{SequenceEmitter}}}{\emph{sequence}, \emph{*args}, \emph{max\_sequences=1}, \emph{**kwargs}}{}
Bases: {\hyperref[\detokenize{api:nanostream.node.NanoNode}]{\sphinxcrossref{\sphinxcode{\sphinxupquote{nanostream.node.NanoNode}}}}}

Emits \sphinxcode{\sphinxupquote{sequence}} \sphinxcode{\sphinxupquote{max\_sequences}} times, or forever if
\sphinxcode{\sphinxupquote{max\_sequences}} is \sphinxcode{\sphinxupquote{None}}.
\index{generator() (nanostream.node.SequenceEmitter method)}

\begin{fulllineitems}
\phantomsection\label{\detokenize{api:nanostream.node.SequenceEmitter.generator}}\pysiglinewithargsret{\sphinxbfcode{\sphinxupquote{generator}}}{}{}
Emit the sequence \sphinxcode{\sphinxupquote{max\_sequences}} times.

\end{fulllineitems}

\index{process\_item() (nanostream.node.SequenceEmitter method)}

\begin{fulllineitems}
\phantomsection\label{\detokenize{api:nanostream.node.SequenceEmitter.process_item}}\pysiglinewithargsret{\sphinxbfcode{\sphinxupquote{process\_item}}}{}{}
Emit the sequence \sphinxcode{\sphinxupquote{max\_sequences}} times.

\end{fulllineitems}


\end{fulllineitems}

\index{Serializer (class in nanostream.node)}

\begin{fulllineitems}
\phantomsection\label{\detokenize{api:nanostream.node.Serializer}}\pysiglinewithargsret{\sphinxbfcode{\sphinxupquote{class }}\sphinxcode{\sphinxupquote{nanostream.node.}}\sphinxbfcode{\sphinxupquote{Serializer}}}{\emph{values=False}, \emph{*args}, \emph{**kwargs}}{}
Bases: {\hyperref[\detokenize{api:nanostream.node.NanoNode}]{\sphinxcrossref{\sphinxcode{\sphinxupquote{nanostream.node.NanoNode}}}}}

Takes an iterable thing as input, and successively yields its items.
\index{process\_item() (nanostream.node.Serializer method)}

\begin{fulllineitems}
\phantomsection\label{\detokenize{api:nanostream.node.Serializer.process_item}}\pysiglinewithargsret{\sphinxbfcode{\sphinxupquote{process\_item}}}{}{}
Default no-op for nodes.

\end{fulllineitems}


\end{fulllineitems}

\index{SimpleTransforms (class in nanostream.node)}

\begin{fulllineitems}
\phantomsection\label{\detokenize{api:nanostream.node.SimpleTransforms}}\pysiglinewithargsret{\sphinxbfcode{\sphinxupquote{class }}\sphinxcode{\sphinxupquote{nanostream.node.}}\sphinxbfcode{\sphinxupquote{SimpleTransforms}}}{\emph{missing\_keypath\_action='ignore'}, \emph{starting\_path=None}, \emph{transform\_mapping=None}, \emph{target\_value=None}, \emph{keypath=None}, \emph{**kwargs}}{}
Bases: {\hyperref[\detokenize{api:nanostream.node.NanoNode}]{\sphinxcrossref{\sphinxcode{\sphinxupquote{nanostream.node.NanoNode}}}}}
\index{process\_item() (nanostream.node.SimpleTransforms method)}

\begin{fulllineitems}
\phantomsection\label{\detokenize{api:nanostream.node.SimpleTransforms.process_item}}\pysiglinewithargsret{\sphinxbfcode{\sphinxupquote{process\_item}}}{}{}
Default no-op for nodes.

\end{fulllineitems}


\end{fulllineitems}

\index{StreamMySQLTable (class in nanostream.node)}

\begin{fulllineitems}
\phantomsection\label{\detokenize{api:nanostream.node.StreamMySQLTable}}\pysiglinewithargsret{\sphinxbfcode{\sphinxupquote{class }}\sphinxcode{\sphinxupquote{nanostream.node.}}\sphinxbfcode{\sphinxupquote{StreamMySQLTable}}}{\emph{*args}, \emph{host='localhost'}, \emph{user=None}, \emph{table=None}, \emph{password=None}, \emph{database=None}, \emph{port=3306}, \emph{to\_row\_obj=False}, \emph{send\_batch\_markers=True}, \emph{**kwargs}}{}
Bases: {\hyperref[\detokenize{api:nanostream.node.NanoNode}]{\sphinxcrossref{\sphinxcode{\sphinxupquote{nanostream.node.NanoNode}}}}}
\index{generator() (nanostream.node.StreamMySQLTable method)}

\begin{fulllineitems}
\phantomsection\label{\detokenize{api:nanostream.node.StreamMySQLTable.generator}}\pysiglinewithargsret{\sphinxbfcode{\sphinxupquote{generator}}}{}{}
\end{fulllineitems}

\index{get\_schema() (nanostream.node.StreamMySQLTable method)}

\begin{fulllineitems}
\phantomsection\label{\detokenize{api:nanostream.node.StreamMySQLTable.get_schema}}\pysiglinewithargsret{\sphinxbfcode{\sphinxupquote{get\_schema}}}{}{}
\end{fulllineitems}

\index{setup() (nanostream.node.StreamMySQLTable method)}

\begin{fulllineitems}
\phantomsection\label{\detokenize{api:nanostream.node.StreamMySQLTable.setup}}\pysiglinewithargsret{\sphinxbfcode{\sphinxupquote{setup}}}{}{}
For classes that require initialization at runtime, which can’t be done
when the class’s \sphinxcode{\sphinxupquote{\_\_init\_\_}} function is called. The \sphinxcode{\sphinxupquote{NanoNode}} base
class’s setup function is just a logging call.

It should be unusual to have to make use of \sphinxcode{\sphinxupquote{setup}} because in practice,
initialization can be done in the \sphinxcode{\sphinxupquote{\_\_init\_\_}} function.

\end{fulllineitems}


\end{fulllineitems}

\index{StreamingJoin (class in nanostream.node)}

\begin{fulllineitems}
\phantomsection\label{\detokenize{api:nanostream.node.StreamingJoin}}\pysiglinewithargsret{\sphinxbfcode{\sphinxupquote{class }}\sphinxcode{\sphinxupquote{nanostream.node.}}\sphinxbfcode{\sphinxupquote{StreamingJoin}}}{\emph{window=30}, \emph{streams=None}, \emph{*args}, \emph{**kwargs}}{}
Bases: {\hyperref[\detokenize{api:nanostream.node.NanoNode}]{\sphinxcrossref{\sphinxcode{\sphinxupquote{nanostream.node.NanoNode}}}}}

Joins two streams on a key, using exact match only. MVP.
\index{process\_item() (nanostream.node.StreamingJoin method)}

\begin{fulllineitems}
\phantomsection\label{\detokenize{api:nanostream.node.StreamingJoin.process_item}}\pysiglinewithargsret{\sphinxbfcode{\sphinxupquote{process\_item}}}{}{}
\end{fulllineitems}


\end{fulllineitems}

\index{SubstituteRegex (class in nanostream.node)}

\begin{fulllineitems}
\phantomsection\label{\detokenize{api:nanostream.node.SubstituteRegex}}\pysiglinewithargsret{\sphinxbfcode{\sphinxupquote{class }}\sphinxcode{\sphinxupquote{nanostream.node.}}\sphinxbfcode{\sphinxupquote{SubstituteRegex}}}{\emph{match\_regex=None}, \emph{substitute\_string=None}, \emph{*args}, \emph{**kwargs}}{}
Bases: {\hyperref[\detokenize{api:nanostream.node.NanoNode}]{\sphinxcrossref{\sphinxcode{\sphinxupquote{nanostream.node.NanoNode}}}}}
\index{process\_item() (nanostream.node.SubstituteRegex method)}

\begin{fulllineitems}
\phantomsection\label{\detokenize{api:nanostream.node.SubstituteRegex.process_item}}\pysiglinewithargsret{\sphinxbfcode{\sphinxupquote{process\_item}}}{}{}
Default no-op for nodes.

\end{fulllineitems}


\end{fulllineitems}

\index{TimeWindowAccumulator (class in nanostream.node)}

\begin{fulllineitems}
\phantomsection\label{\detokenize{api:nanostream.node.TimeWindowAccumulator}}\pysiglinewithargsret{\sphinxbfcode{\sphinxupquote{class }}\sphinxcode{\sphinxupquote{nanostream.node.}}\sphinxbfcode{\sphinxupquote{TimeWindowAccumulator}}}{\emph{*args}, \emph{**kwargs}}{}
Bases: {\hyperref[\detokenize{api:nanostream.node.NanoNode}]{\sphinxcrossref{\sphinxcode{\sphinxupquote{nanostream.node.NanoNode}}}}}

Every N seconds, put the latest M seconds data on the queue.

\end{fulllineitems}

\index{bcolors (class in nanostream.node)}

\begin{fulllineitems}
\phantomsection\label{\detokenize{api:nanostream.node.bcolors}}\pysigline{\sphinxbfcode{\sphinxupquote{class }}\sphinxcode{\sphinxupquote{nanostream.node.}}\sphinxbfcode{\sphinxupquote{bcolors}}}
Bases: \sphinxcode{\sphinxupquote{object}}

This class holds the values for the various colors that are used in the
tables that monitor the status of the nodes.
\index{BOLD (nanostream.node.bcolors attribute)}

\begin{fulllineitems}
\phantomsection\label{\detokenize{api:nanostream.node.bcolors.BOLD}}\pysigline{\sphinxbfcode{\sphinxupquote{BOLD}}\sphinxbfcode{\sphinxupquote{ = '\textbackslash{}x1b{[}1m'}}}
\end{fulllineitems}

\index{ENDC (nanostream.node.bcolors attribute)}

\begin{fulllineitems}
\phantomsection\label{\detokenize{api:nanostream.node.bcolors.ENDC}}\pysigline{\sphinxbfcode{\sphinxupquote{ENDC}}\sphinxbfcode{\sphinxupquote{ = '\textbackslash{}x1b{[}0m'}}}
\end{fulllineitems}

\index{FAIL (nanostream.node.bcolors attribute)}

\begin{fulllineitems}
\phantomsection\label{\detokenize{api:nanostream.node.bcolors.FAIL}}\pysigline{\sphinxbfcode{\sphinxupquote{FAIL}}\sphinxbfcode{\sphinxupquote{ = '\textbackslash{}x1b{[}91m'}}}
\end{fulllineitems}

\index{HEADER (nanostream.node.bcolors attribute)}

\begin{fulllineitems}
\phantomsection\label{\detokenize{api:nanostream.node.bcolors.HEADER}}\pysigline{\sphinxbfcode{\sphinxupquote{HEADER}}\sphinxbfcode{\sphinxupquote{ = '\textbackslash{}x1b{[}95m'}}}
\end{fulllineitems}

\index{OKBLUE (nanostream.node.bcolors attribute)}

\begin{fulllineitems}
\phantomsection\label{\detokenize{api:nanostream.node.bcolors.OKBLUE}}\pysigline{\sphinxbfcode{\sphinxupquote{OKBLUE}}\sphinxbfcode{\sphinxupquote{ = '\textbackslash{}x1b{[}94m'}}}
\end{fulllineitems}

\index{OKGREEN (nanostream.node.bcolors attribute)}

\begin{fulllineitems}
\phantomsection\label{\detokenize{api:nanostream.node.bcolors.OKGREEN}}\pysigline{\sphinxbfcode{\sphinxupquote{OKGREEN}}\sphinxbfcode{\sphinxupquote{ = '\textbackslash{}x1b{[}92m'}}}
\end{fulllineitems}

\index{UNDERLINE (nanostream.node.bcolors attribute)}

\begin{fulllineitems}
\phantomsection\label{\detokenize{api:nanostream.node.bcolors.UNDERLINE}}\pysigline{\sphinxbfcode{\sphinxupquote{UNDERLINE}}\sphinxbfcode{\sphinxupquote{ = '\textbackslash{}x1b{[}4m'}}}
\end{fulllineitems}

\index{WARNING (nanostream.node.bcolors attribute)}

\begin{fulllineitems}
\phantomsection\label{\detokenize{api:nanostream.node.bcolors.WARNING}}\pysigline{\sphinxbfcode{\sphinxupquote{WARNING}}\sphinxbfcode{\sphinxupquote{ = '\textbackslash{}x1b{[}93m'}}}
\end{fulllineitems}


\end{fulllineitems}

\index{class\_factory() (in module nanostream.node)}

\begin{fulllineitems}
\phantomsection\label{\detokenize{api:nanostream.node.class_factory}}\pysiglinewithargsret{\sphinxcode{\sphinxupquote{nanostream.node.}}\sphinxbfcode{\sphinxupquote{class\_factory}}}{\emph{raw\_config}}{}
\end{fulllineitems}

\index{get\_environment\_variables() (in module nanostream.node)}

\begin{fulllineitems}
\phantomsection\label{\detokenize{api:nanostream.node.get_environment_variables}}\pysiglinewithargsret{\sphinxcode{\sphinxupquote{nanostream.node.}}\sphinxbfcode{\sphinxupquote{get\_environment\_variables}}}{\emph{*args}}{}
Retrieves the environment variables listed in \sphinxcode{\sphinxupquote{*args}}.
\begin{description}
\item[{Args:}] \leavevmode
args (list of str): List of environment variables.

\item[{Returns:}] \leavevmode\begin{description}
\item[{dict: Dictionary of environment variables to values. If the environment}] \leavevmode
variable is not defined, the value is \sphinxcode{\sphinxupquote{None}}.

\end{description}

\end{description}

\end{fulllineitems}

\index{get\_node\_dict() (in module nanostream.node)}

\begin{fulllineitems}
\phantomsection\label{\detokenize{api:nanostream.node.get_node_dict}}\pysiglinewithargsret{\sphinxcode{\sphinxupquote{nanostream.node.}}\sphinxbfcode{\sphinxupquote{get\_node\_dict}}}{\emph{node\_config}}{}
\end{fulllineitems}

\index{kwarg\_remapper() (in module nanostream.node)}

\begin{fulllineitems}
\phantomsection\label{\detokenize{api:nanostream.node.kwarg_remapper}}\pysiglinewithargsret{\sphinxcode{\sphinxupquote{nanostream.node.}}\sphinxbfcode{\sphinxupquote{kwarg\_remapper}}}{\emph{f}, \emph{**kwarg\_mapping}}{}
\end{fulllineitems}

\index{no\_op() (in module nanostream.node)}

\begin{fulllineitems}
\phantomsection\label{\detokenize{api:nanostream.node.no_op}}\pysiglinewithargsret{\sphinxcode{\sphinxupquote{nanostream.node.}}\sphinxbfcode{\sphinxupquote{no\_op}}}{\emph{*args}, \emph{**kwargs}}{}
No-op function to serve as default \sphinxcode{\sphinxupquote{get\_runtime\_attrs}}.

\end{fulllineitems}

\index{template\_class() (in module nanostream.node)}

\begin{fulllineitems}
\phantomsection\label{\detokenize{api:nanostream.node.template_class}}\pysiglinewithargsret{\sphinxcode{\sphinxupquote{nanostream.node.}}\sphinxbfcode{\sphinxupquote{template\_class}}}{\emph{class\_name}, \emph{parent\_class}, \emph{kwargs\_remapping}, \emph{frozen\_arguments\_mapping}}{}
\end{fulllineitems}

\phantomsection\label{\detokenize{api:module-nanostream.civis_nodes}}\index{nanostream.civis\_nodes (module)}

\section{Civis-specific node types}
\label{\detokenize{api:civis-specific-node-types}}
This is where any classes specific to the Civis API live.
\index{CivisSQLExecute (class in nanostream.civis\_nodes)}

\begin{fulllineitems}
\phantomsection\label{\detokenize{api:nanostream.civis_nodes.CivisSQLExecute}}\pysiglinewithargsret{\sphinxbfcode{\sphinxupquote{class }}\sphinxcode{\sphinxupquote{nanostream.civis\_nodes.}}\sphinxbfcode{\sphinxupquote{CivisSQLExecute}}}{\emph{*args}, \emph{sql=None}, \emph{civis\_api\_key=None}, \emph{civis\_api\_key\_env\_var='CIVIS\_API\_KEY'}, \emph{database=None}, \emph{dummy\_run=False}, \emph{query\_dict=None}, \emph{returned\_columns=None}, \emph{**kwargs}}{}
Bases: {\hyperref[\detokenize{api:nanostream.node.NanoNode}]{\sphinxcrossref{\sphinxcode{\sphinxupquote{nanostream.node.NanoNode}}}}}

Execute a SQL statement and return the results.
\index{process\_item() (nanostream.civis\_nodes.CivisSQLExecute method)}

\begin{fulllineitems}
\phantomsection\label{\detokenize{api:nanostream.civis_nodes.CivisSQLExecute.process_item}}\pysiglinewithargsret{\sphinxbfcode{\sphinxupquote{process\_item}}}{}{}
Execute a SQL statement and return the result.

\end{fulllineitems}


\end{fulllineitems}

\index{CivisToCSV (class in nanostream.civis\_nodes)}

\begin{fulllineitems}
\phantomsection\label{\detokenize{api:nanostream.civis_nodes.CivisToCSV}}\pysiglinewithargsret{\sphinxbfcode{\sphinxupquote{class }}\sphinxcode{\sphinxupquote{nanostream.civis\_nodes.}}\sphinxbfcode{\sphinxupquote{CivisToCSV}}}{\emph{*args}, \emph{sql=None}, \emph{civis\_api\_key=None}, \emph{civis\_api\_key\_env\_var='CIVIS\_API\_KEY'}, \emph{database=None}, \emph{dummy\_run=False}, \emph{query\_dict=None}, \emph{returned\_columns=None}, \emph{include\_headers=True}, \emph{delimiter='}, \emph{'}, \emph{**kwargs}}{}
Bases: {\hyperref[\detokenize{api:nanostream.node.NanoNode}]{\sphinxcrossref{\sphinxcode{\sphinxupquote{nanostream.node.NanoNode}}}}}

Execute a SQL statement and return the results via a CSV file.
\index{process\_item() (nanostream.civis\_nodes.CivisToCSV method)}

\begin{fulllineitems}
\phantomsection\label{\detokenize{api:nanostream.civis_nodes.CivisToCSV.process_item}}\pysiglinewithargsret{\sphinxbfcode{\sphinxupquote{process\_item}}}{}{}
Execute a SQL statement and return the result.

\end{fulllineitems}


\end{fulllineitems}

\index{EnsureCivisRedshiftTableExists (class in nanostream.civis\_nodes)}

\begin{fulllineitems}
\phantomsection\label{\detokenize{api:nanostream.civis_nodes.EnsureCivisRedshiftTableExists}}\pysiglinewithargsret{\sphinxbfcode{\sphinxupquote{class }}\sphinxcode{\sphinxupquote{nanostream.civis\_nodes.}}\sphinxbfcode{\sphinxupquote{EnsureCivisRedshiftTableExists}}}{\emph{on\_failure='exit'}, \emph{table=None}, \emph{schema=None}, \emph{columns=None}, \emph{block=True}, \emph{**kwargs}}{}
Bases: {\hyperref[\detokenize{api:nanostream.node.NanoNode}]{\sphinxcrossref{\sphinxcode{\sphinxupquote{nanostream.node.NanoNode}}}}}
\index{generator() (nanostream.civis\_nodes.EnsureCivisRedshiftTableExists method)}

\begin{fulllineitems}
\phantomsection\label{\detokenize{api:nanostream.civis_nodes.EnsureCivisRedshiftTableExists.generator}}\pysiglinewithargsret{\sphinxbfcode{\sphinxupquote{generator}}}{}{}
\end{fulllineitems}

\index{process\_item() (nanostream.civis\_nodes.EnsureCivisRedshiftTableExists method)}

\begin{fulllineitems}
\phantomsection\label{\detokenize{api:nanostream.civis_nodes.EnsureCivisRedshiftTableExists.process_item}}\pysiglinewithargsret{\sphinxbfcode{\sphinxupquote{process\_item}}}{}{}
Default no-op for nodes.

\end{fulllineitems}


\end{fulllineitems}

\index{FindValueInRedshiftColumn (class in nanostream.civis\_nodes)}

\begin{fulllineitems}
\phantomsection\label{\detokenize{api:nanostream.civis_nodes.FindValueInRedshiftColumn}}\pysiglinewithargsret{\sphinxbfcode{\sphinxupquote{class }}\sphinxcode{\sphinxupquote{nanostream.civis\_nodes.}}\sphinxbfcode{\sphinxupquote{FindValueInRedshiftColumn}}}{\emph{on\_failure='exit'}, \emph{table=None}, \emph{database=None}, \emph{schema=None}, \emph{column=None}, \emph{choice='max'}, \emph{**kwargs}}{}
Bases: {\hyperref[\detokenize{api:nanostream.node.NanoNode}]{\sphinxcrossref{\sphinxcode{\sphinxupquote{nanostream.node.NanoNode}}}}}
\index{generator() (nanostream.civis\_nodes.FindValueInRedshiftColumn method)}

\begin{fulllineitems}
\phantomsection\label{\detokenize{api:nanostream.civis_nodes.FindValueInRedshiftColumn.generator}}\pysiglinewithargsret{\sphinxbfcode{\sphinxupquote{generator}}}{}{}
\end{fulllineitems}

\index{process\_item() (nanostream.civis\_nodes.FindValueInRedshiftColumn method)}

\begin{fulllineitems}
\phantomsection\label{\detokenize{api:nanostream.civis_nodes.FindValueInRedshiftColumn.process_item}}\pysiglinewithargsret{\sphinxbfcode{\sphinxupquote{process\_item}}}{}{}
Default no-op for nodes.

\end{fulllineitems}


\end{fulllineitems}

\index{SendToCivis (class in nanostream.civis\_nodes)}

\begin{fulllineitems}
\phantomsection\label{\detokenize{api:nanostream.civis_nodes.SendToCivis}}\pysiglinewithargsret{\sphinxbfcode{\sphinxupquote{class }}\sphinxcode{\sphinxupquote{nanostream.civis\_nodes.}}\sphinxbfcode{\sphinxupquote{SendToCivis}}}{\emph{*args}, \emph{civis\_api\_key=None}, \emph{civis\_api\_key\_env\_var='CIVIS\_API\_KEY'}, \emph{database=None}, \emph{schema=None}, \emph{existing\_table\_rows='append'}, \emph{include\_columns=None}, \emph{dummy\_run=False}, \emph{block=False}, \emph{max\_errors=0}, \emph{table=None}, \emph{columns=None}, \emph{remap=None}, \emph{recorded\_tables=\{\}}, \emph{**kwargs}}{}
Bases: {\hyperref[\detokenize{api:nanostream.node.NanoNode}]{\sphinxcrossref{\sphinxcode{\sphinxupquote{nanostream.node.NanoNode}}}}}
\index{monitor\_futures() (nanostream.civis\_nodes.SendToCivis method)}

\begin{fulllineitems}
\phantomsection\label{\detokenize{api:nanostream.civis_nodes.SendToCivis.monitor_futures}}\pysiglinewithargsret{\sphinxbfcode{\sphinxupquote{monitor\_futures}}}{}{}
\end{fulllineitems}

\index{process\_item() (nanostream.civis\_nodes.SendToCivis method)}

\begin{fulllineitems}
\phantomsection\label{\detokenize{api:nanostream.civis_nodes.SendToCivis.process_item}}\pysiglinewithargsret{\sphinxbfcode{\sphinxupquote{process\_item}}}{}{}
Accept a bunch of dictionaries mapping column names to values.

\end{fulllineitems}

\index{setup() (nanostream.civis\_nodes.SendToCivis method)}

\begin{fulllineitems}
\phantomsection\label{\detokenize{api:nanostream.civis_nodes.SendToCivis.setup}}\pysiglinewithargsret{\sphinxbfcode{\sphinxupquote{setup}}}{}{}
Not sure if we’ll need this. We could get a client and pass it around.

\end{fulllineitems}


\end{fulllineitems}

\phantomsection\label{\detokenize{api:module-nanostream.utils.data_structures}}\index{nanostream.utils.data\_structures (module)}

\section{Data structures module}
\label{\detokenize{api:data-structures-module}}
Data types (e.g. Rows, Records) for ETL.
\index{BOOL (class in nanostream.utils.data\_structures)}

\begin{fulllineitems}
\phantomsection\label{\detokenize{api:nanostream.utils.data_structures.BOOL}}\pysiglinewithargsret{\sphinxbfcode{\sphinxupquote{class }}\sphinxcode{\sphinxupquote{nanostream.utils.data\_structures.}}\sphinxbfcode{\sphinxupquote{BOOL}}}{\emph{value}, \emph{original\_type=None}, \emph{name=None}}{}
Bases: {\hyperref[\detokenize{api:nanostream.utils.data_structures.DataType}]{\sphinxcrossref{\sphinxcode{\sphinxupquote{nanostream.utils.data\_structures.DataType}}}}}, {\hyperref[\detokenize{api:nanostream.utils.data_structures.IntermediateTypeSystem}]{\sphinxcrossref{\sphinxcode{\sphinxupquote{nanostream.utils.data\_structures.IntermediateTypeSystem}}}}}
\index{python\_cast\_function (nanostream.utils.data\_structures.BOOL attribute)}

\begin{fulllineitems}
\phantomsection\label{\detokenize{api:nanostream.utils.data_structures.BOOL.python_cast_function}}\pysigline{\sphinxbfcode{\sphinxupquote{python\_cast\_function}}}
alias of \sphinxcode{\sphinxupquote{builtins.bool}}

\end{fulllineitems}


\end{fulllineitems}

\index{DATETIME (class in nanostream.utils.data\_structures)}

\begin{fulllineitems}
\phantomsection\label{\detokenize{api:nanostream.utils.data_structures.DATETIME}}\pysiglinewithargsret{\sphinxbfcode{\sphinxupquote{class }}\sphinxcode{\sphinxupquote{nanostream.utils.data\_structures.}}\sphinxbfcode{\sphinxupquote{DATETIME}}}{\emph{value}, \emph{original\_type=None}, \emph{name=None}}{}
Bases: {\hyperref[\detokenize{api:nanostream.utils.data_structures.DataType}]{\sphinxcrossref{\sphinxcode{\sphinxupquote{nanostream.utils.data\_structures.DataType}}}}}, {\hyperref[\detokenize{api:nanostream.utils.data_structures.IntermediateTypeSystem}]{\sphinxcrossref{\sphinxcode{\sphinxupquote{nanostream.utils.data\_structures.IntermediateTypeSystem}}}}}
\index{python\_cast\_function() (nanostream.utils.data\_structures.DATETIME method)}

\begin{fulllineitems}
\phantomsection\label{\detokenize{api:nanostream.utils.data_structures.DATETIME.python_cast_function}}\pysiglinewithargsret{\sphinxbfcode{\sphinxupquote{python\_cast\_function}}}{}{}
\end{fulllineitems}


\end{fulllineitems}

\index{DataSourceTypeSystem (class in nanostream.utils.data\_structures)}

\begin{fulllineitems}
\phantomsection\label{\detokenize{api:nanostream.utils.data_structures.DataSourceTypeSystem}}\pysigline{\sphinxbfcode{\sphinxupquote{class }}\sphinxcode{\sphinxupquote{nanostream.utils.data\_structures.}}\sphinxbfcode{\sphinxupquote{DataSourceTypeSystem}}}
Bases: \sphinxcode{\sphinxupquote{object}}

Information about mapping one type system onto another contained in
the children of this class.
\index{convert() (nanostream.utils.data\_structures.DataSourceTypeSystem static method)}

\begin{fulllineitems}
\phantomsection\label{\detokenize{api:nanostream.utils.data_structures.DataSourceTypeSystem.convert}}\pysiglinewithargsret{\sphinxbfcode{\sphinxupquote{static }}\sphinxbfcode{\sphinxupquote{convert}}}{\emph{obj}}{}
Override this method if something more complicated is necessary.

\end{fulllineitems}

\index{type\_mapping() (nanostream.utils.data\_structures.DataSourceTypeSystem static method)}

\begin{fulllineitems}
\phantomsection\label{\detokenize{api:nanostream.utils.data_structures.DataSourceTypeSystem.type_mapping}}\pysiglinewithargsret{\sphinxbfcode{\sphinxupquote{static }}\sphinxbfcode{\sphinxupquote{type\_mapping}}}{\emph{*args}, \emph{**kwargs}}{}
\end{fulllineitems}


\end{fulllineitems}

\index{DataType (class in nanostream.utils.data\_structures)}

\begin{fulllineitems}
\phantomsection\label{\detokenize{api:nanostream.utils.data_structures.DataType}}\pysiglinewithargsret{\sphinxbfcode{\sphinxupquote{class }}\sphinxcode{\sphinxupquote{nanostream.utils.data\_structures.}}\sphinxbfcode{\sphinxupquote{DataType}}}{\emph{value}, \emph{original\_type=None}, \emph{name=None}}{}
Bases: \sphinxcode{\sphinxupquote{object}}

Each \sphinxcode{\sphinxupquote{DataType}} gets a \sphinxcode{\sphinxupquote{python\_cast\_function}}, which is a function.
\index{intermediate\_type (nanostream.utils.data\_structures.DataType attribute)}

\begin{fulllineitems}
\phantomsection\label{\detokenize{api:nanostream.utils.data_structures.DataType.intermediate_type}}\pysigline{\sphinxbfcode{\sphinxupquote{intermediate\_type}}\sphinxbfcode{\sphinxupquote{ = None}}}
\end{fulllineitems}

\index{python\_cast\_function (nanostream.utils.data\_structures.DataType attribute)}

\begin{fulllineitems}
\phantomsection\label{\detokenize{api:nanostream.utils.data_structures.DataType.python_cast_function}}\pysigline{\sphinxbfcode{\sphinxupquote{python\_cast\_function}}\sphinxbfcode{\sphinxupquote{ = None}}}
\end{fulllineitems}

\index{to\_intermediate\_type() (nanostream.utils.data\_structures.DataType method)}

\begin{fulllineitems}
\phantomsection\label{\detokenize{api:nanostream.utils.data_structures.DataType.to_intermediate_type}}\pysiglinewithargsret{\sphinxbfcode{\sphinxupquote{to\_intermediate\_type}}}{}{}
Convert the \sphinxcode{\sphinxupquote{DataType}} to an \sphinxcode{\sphinxupquote{IntermediateDataType}} using its
class’s \sphinxcode{\sphinxupquote{intermediate\_type}} attribute.

\end{fulllineitems}

\index{to\_python() (nanostream.utils.data\_structures.DataType method)}

\begin{fulllineitems}
\phantomsection\label{\detokenize{api:nanostream.utils.data_structures.DataType.to_python}}\pysiglinewithargsret{\sphinxbfcode{\sphinxupquote{to\_python}}}{}{}
\end{fulllineitems}

\index{type\_system (nanostream.utils.data\_structures.DataType attribute)}

\begin{fulllineitems}
\phantomsection\label{\detokenize{api:nanostream.utils.data_structures.DataType.type_system}}\pysigline{\sphinxbfcode{\sphinxupquote{type\_system}}}
Just for convenience to make the type system an attribute.

\end{fulllineitems}


\end{fulllineitems}

\index{FLOAT (class in nanostream.utils.data\_structures)}

\begin{fulllineitems}
\phantomsection\label{\detokenize{api:nanostream.utils.data_structures.FLOAT}}\pysiglinewithargsret{\sphinxbfcode{\sphinxupquote{class }}\sphinxcode{\sphinxupquote{nanostream.utils.data\_structures.}}\sphinxbfcode{\sphinxupquote{FLOAT}}}{\emph{value}, \emph{original\_type=None}, \emph{name=None}}{}
Bases: {\hyperref[\detokenize{api:nanostream.utils.data_structures.DataType}]{\sphinxcrossref{\sphinxcode{\sphinxupquote{nanostream.utils.data\_structures.DataType}}}}}, {\hyperref[\detokenize{api:nanostream.utils.data_structures.IntermediateTypeSystem}]{\sphinxcrossref{\sphinxcode{\sphinxupquote{nanostream.utils.data\_structures.IntermediateTypeSystem}}}}}
\index{python\_cast\_function (nanostream.utils.data\_structures.FLOAT attribute)}

\begin{fulllineitems}
\phantomsection\label{\detokenize{api:nanostream.utils.data_structures.FLOAT.python_cast_function}}\pysigline{\sphinxbfcode{\sphinxupquote{python\_cast\_function}}}
alias of \sphinxcode{\sphinxupquote{builtins.float}}

\end{fulllineitems}


\end{fulllineitems}

\index{INTEGER (class in nanostream.utils.data\_structures)}

\begin{fulllineitems}
\phantomsection\label{\detokenize{api:nanostream.utils.data_structures.INTEGER}}\pysiglinewithargsret{\sphinxbfcode{\sphinxupquote{class }}\sphinxcode{\sphinxupquote{nanostream.utils.data\_structures.}}\sphinxbfcode{\sphinxupquote{INTEGER}}}{\emph{value}, \emph{original\_type=None}, \emph{name=None}}{}
Bases: {\hyperref[\detokenize{api:nanostream.utils.data_structures.DataType}]{\sphinxcrossref{\sphinxcode{\sphinxupquote{nanostream.utils.data\_structures.DataType}}}}}, {\hyperref[\detokenize{api:nanostream.utils.data_structures.IntermediateTypeSystem}]{\sphinxcrossref{\sphinxcode{\sphinxupquote{nanostream.utils.data\_structures.IntermediateTypeSystem}}}}}
\index{python\_cast\_function (nanostream.utils.data\_structures.INTEGER attribute)}

\begin{fulllineitems}
\phantomsection\label{\detokenize{api:nanostream.utils.data_structures.INTEGER.python_cast_function}}\pysigline{\sphinxbfcode{\sphinxupquote{python\_cast\_function}}}
alias of \sphinxcode{\sphinxupquote{builtins.int}}

\end{fulllineitems}


\end{fulllineitems}

\index{IncompatibleTypesException}

\begin{fulllineitems}
\phantomsection\label{\detokenize{api:nanostream.utils.data_structures.IncompatibleTypesException}}\pysigline{\sphinxbfcode{\sphinxupquote{exception }}\sphinxcode{\sphinxupquote{nanostream.utils.data\_structures.}}\sphinxbfcode{\sphinxupquote{IncompatibleTypesException}}}
Bases: \sphinxcode{\sphinxupquote{Exception}}

\end{fulllineitems}

\index{IntermediateTypeSystem (class in nanostream.utils.data\_structures)}

\begin{fulllineitems}
\phantomsection\label{\detokenize{api:nanostream.utils.data_structures.IntermediateTypeSystem}}\pysigline{\sphinxbfcode{\sphinxupquote{class }}\sphinxcode{\sphinxupquote{nanostream.utils.data\_structures.}}\sphinxbfcode{\sphinxupquote{IntermediateTypeSystem}}}
Bases: {\hyperref[\detokenize{api:nanostream.utils.data_structures.DataSourceTypeSystem}]{\sphinxcrossref{\sphinxcode{\sphinxupquote{nanostream.utils.data\_structures.DataSourceTypeSystem}}}}}

Never instantiate this by hand.

\end{fulllineitems}

\index{MYSQL\_BOOL (class in nanostream.utils.data\_structures)}

\begin{fulllineitems}
\phantomsection\label{\detokenize{api:nanostream.utils.data_structures.MYSQL_BOOL}}\pysiglinewithargsret{\sphinxbfcode{\sphinxupquote{class }}\sphinxcode{\sphinxupquote{nanostream.utils.data\_structures.}}\sphinxbfcode{\sphinxupquote{MYSQL\_BOOL}}}{\emph{value}, \emph{original\_type=None}, \emph{name=None}}{}
Bases: {\hyperref[\detokenize{api:nanostream.utils.data_structures.DataType}]{\sphinxcrossref{\sphinxcode{\sphinxupquote{nanostream.utils.data\_structures.DataType}}}}}, {\hyperref[\detokenize{api:nanostream.utils.data_structures.MySQLTypeSystem}]{\sphinxcrossref{\sphinxcode{\sphinxupquote{nanostream.utils.data\_structures.MySQLTypeSystem}}}}}
\index{intermediate\_type (nanostream.utils.data\_structures.MYSQL\_BOOL attribute)}

\begin{fulllineitems}
\phantomsection\label{\detokenize{api:nanostream.utils.data_structures.MYSQL_BOOL.intermediate_type}}\pysigline{\sphinxbfcode{\sphinxupquote{intermediate\_type}}}
alias of {\hyperref[\detokenize{api:nanostream.utils.data_structures.BOOL}]{\sphinxcrossref{\sphinxcode{\sphinxupquote{BOOL}}}}}

\end{fulllineitems}

\index{python\_cast\_function (nanostream.utils.data\_structures.MYSQL\_BOOL attribute)}

\begin{fulllineitems}
\phantomsection\label{\detokenize{api:nanostream.utils.data_structures.MYSQL_BOOL.python_cast_function}}\pysigline{\sphinxbfcode{\sphinxupquote{python\_cast\_function}}}
alias of \sphinxcode{\sphinxupquote{builtins.bool}}

\end{fulllineitems}


\end{fulllineitems}

\index{MYSQL\_DATE (class in nanostream.utils.data\_structures)}

\begin{fulllineitems}
\phantomsection\label{\detokenize{api:nanostream.utils.data_structures.MYSQL_DATE}}\pysiglinewithargsret{\sphinxbfcode{\sphinxupquote{class }}\sphinxcode{\sphinxupquote{nanostream.utils.data\_structures.}}\sphinxbfcode{\sphinxupquote{MYSQL\_DATE}}}{\emph{value}, \emph{original\_type=None}, \emph{name=None}}{}
Bases: {\hyperref[\detokenize{api:nanostream.utils.data_structures.DataType}]{\sphinxcrossref{\sphinxcode{\sphinxupquote{nanostream.utils.data\_structures.DataType}}}}}, {\hyperref[\detokenize{api:nanostream.utils.data_structures.MySQLTypeSystem}]{\sphinxcrossref{\sphinxcode{\sphinxupquote{nanostream.utils.data\_structures.MySQLTypeSystem}}}}}
\index{intermediate\_type (nanostream.utils.data\_structures.MYSQL\_DATE attribute)}

\begin{fulllineitems}
\phantomsection\label{\detokenize{api:nanostream.utils.data_structures.MYSQL_DATE.intermediate_type}}\pysigline{\sphinxbfcode{\sphinxupquote{intermediate\_type}}}
alias of {\hyperref[\detokenize{api:nanostream.utils.data_structures.DATETIME}]{\sphinxcrossref{\sphinxcode{\sphinxupquote{DATETIME}}}}}

\end{fulllineitems}

\index{python\_cast\_function() (nanostream.utils.data\_structures.MYSQL\_DATE method)}

\begin{fulllineitems}
\phantomsection\label{\detokenize{api:nanostream.utils.data_structures.MYSQL_DATE.python_cast_function}}\pysiglinewithargsret{\sphinxbfcode{\sphinxupquote{python\_cast\_function}}}{}{}
\end{fulllineitems}


\end{fulllineitems}

\index{MYSQL\_ENUM (class in nanostream.utils.data\_structures)}

\begin{fulllineitems}
\phantomsection\label{\detokenize{api:nanostream.utils.data_structures.MYSQL_ENUM}}\pysiglinewithargsret{\sphinxbfcode{\sphinxupquote{class }}\sphinxcode{\sphinxupquote{nanostream.utils.data\_structures.}}\sphinxbfcode{\sphinxupquote{MYSQL\_ENUM}}}{\emph{value}, \emph{original\_type=None}, \emph{name=None}}{}
Bases: {\hyperref[\detokenize{api:nanostream.utils.data_structures.DataType}]{\sphinxcrossref{\sphinxcode{\sphinxupquote{nanostream.utils.data\_structures.DataType}}}}}, {\hyperref[\detokenize{api:nanostream.utils.data_structures.MySQLTypeSystem}]{\sphinxcrossref{\sphinxcode{\sphinxupquote{nanostream.utils.data\_structures.MySQLTypeSystem}}}}}
\index{intermediate\_type (nanostream.utils.data\_structures.MYSQL\_ENUM attribute)}

\begin{fulllineitems}
\phantomsection\label{\detokenize{api:nanostream.utils.data_structures.MYSQL_ENUM.intermediate_type}}\pysigline{\sphinxbfcode{\sphinxupquote{intermediate\_type}}}
alias of {\hyperref[\detokenize{api:nanostream.utils.data_structures.STRING}]{\sphinxcrossref{\sphinxcode{\sphinxupquote{STRING}}}}}

\end{fulllineitems}

\index{python\_cast\_function (nanostream.utils.data\_structures.MYSQL\_ENUM attribute)}

\begin{fulllineitems}
\phantomsection\label{\detokenize{api:nanostream.utils.data_structures.MYSQL_ENUM.python_cast_function}}\pysigline{\sphinxbfcode{\sphinxupquote{python\_cast\_function}}}
alias of \sphinxcode{\sphinxupquote{builtins.str}}

\end{fulllineitems}


\end{fulllineitems}

\index{MYSQL\_INTEGER (class in nanostream.utils.data\_structures)}

\begin{fulllineitems}
\phantomsection\label{\detokenize{api:nanostream.utils.data_structures.MYSQL_INTEGER}}\pysigline{\sphinxbfcode{\sphinxupquote{class }}\sphinxcode{\sphinxupquote{nanostream.utils.data\_structures.}}\sphinxbfcode{\sphinxupquote{MYSQL\_INTEGER}}}
Bases: \sphinxcode{\sphinxupquote{type}}

\end{fulllineitems}

\index{MYSQL\_INTEGER0 (class in nanostream.utils.data\_structures)}

\begin{fulllineitems}
\phantomsection\label{\detokenize{api:nanostream.utils.data_structures.MYSQL_INTEGER0}}\pysiglinewithargsret{\sphinxbfcode{\sphinxupquote{class }}\sphinxcode{\sphinxupquote{nanostream.utils.data\_structures.}}\sphinxbfcode{\sphinxupquote{MYSQL\_INTEGER0}}}{\emph{value}, \emph{original\_type=None}, \emph{name=None}}{}
Bases: {\hyperref[\detokenize{api:nanostream.utils.data_structures.MYSQL_INTEGER_BASE}]{\sphinxcrossref{\sphinxcode{\sphinxupquote{nanostream.utils.data\_structures.MYSQL\_INTEGER\_BASE}}}}}
\index{max\_length (nanostream.utils.data\_structures.MYSQL\_INTEGER0 attribute)}

\begin{fulllineitems}
\phantomsection\label{\detokenize{api:nanostream.utils.data_structures.MYSQL_INTEGER0.max_length}}\pysigline{\sphinxbfcode{\sphinxupquote{max\_length}}\sphinxbfcode{\sphinxupquote{ = 0}}}
\end{fulllineitems}


\end{fulllineitems}

\index{MYSQL\_INTEGER1 (class in nanostream.utils.data\_structures)}

\begin{fulllineitems}
\phantomsection\label{\detokenize{api:nanostream.utils.data_structures.MYSQL_INTEGER1}}\pysiglinewithargsret{\sphinxbfcode{\sphinxupquote{class }}\sphinxcode{\sphinxupquote{nanostream.utils.data\_structures.}}\sphinxbfcode{\sphinxupquote{MYSQL\_INTEGER1}}}{\emph{value}, \emph{original\_type=None}, \emph{name=None}}{}
Bases: {\hyperref[\detokenize{api:nanostream.utils.data_structures.MYSQL_INTEGER_BASE}]{\sphinxcrossref{\sphinxcode{\sphinxupquote{nanostream.utils.data\_structures.MYSQL\_INTEGER\_BASE}}}}}
\index{max\_length (nanostream.utils.data\_structures.MYSQL\_INTEGER1 attribute)}

\begin{fulllineitems}
\phantomsection\label{\detokenize{api:nanostream.utils.data_structures.MYSQL_INTEGER1.max_length}}\pysigline{\sphinxbfcode{\sphinxupquote{max\_length}}\sphinxbfcode{\sphinxupquote{ = 1}}}
\end{fulllineitems}


\end{fulllineitems}

\index{MYSQL\_INTEGER10 (class in nanostream.utils.data\_structures)}

\begin{fulllineitems}
\phantomsection\label{\detokenize{api:nanostream.utils.data_structures.MYSQL_INTEGER10}}\pysiglinewithargsret{\sphinxbfcode{\sphinxupquote{class }}\sphinxcode{\sphinxupquote{nanostream.utils.data\_structures.}}\sphinxbfcode{\sphinxupquote{MYSQL\_INTEGER10}}}{\emph{value}, \emph{original\_type=None}, \emph{name=None}}{}
Bases: {\hyperref[\detokenize{api:nanostream.utils.data_structures.MYSQL_INTEGER_BASE}]{\sphinxcrossref{\sphinxcode{\sphinxupquote{nanostream.utils.data\_structures.MYSQL\_INTEGER\_BASE}}}}}
\index{max\_length (nanostream.utils.data\_structures.MYSQL\_INTEGER10 attribute)}

\begin{fulllineitems}
\phantomsection\label{\detokenize{api:nanostream.utils.data_structures.MYSQL_INTEGER10.max_length}}\pysigline{\sphinxbfcode{\sphinxupquote{max\_length}}\sphinxbfcode{\sphinxupquote{ = 10}}}
\end{fulllineitems}


\end{fulllineitems}

\index{MYSQL\_INTEGER1024 (class in nanostream.utils.data\_structures)}

\begin{fulllineitems}
\phantomsection\label{\detokenize{api:nanostream.utils.data_structures.MYSQL_INTEGER1024}}\pysiglinewithargsret{\sphinxbfcode{\sphinxupquote{class }}\sphinxcode{\sphinxupquote{nanostream.utils.data\_structures.}}\sphinxbfcode{\sphinxupquote{MYSQL\_INTEGER1024}}}{\emph{value}, \emph{original\_type=None}, \emph{name=None}}{}
Bases: {\hyperref[\detokenize{api:nanostream.utils.data_structures.MYSQL_INTEGER_BASE}]{\sphinxcrossref{\sphinxcode{\sphinxupquote{nanostream.utils.data\_structures.MYSQL\_INTEGER\_BASE}}}}}
\index{max\_length (nanostream.utils.data\_structures.MYSQL\_INTEGER1024 attribute)}

\begin{fulllineitems}
\phantomsection\label{\detokenize{api:nanostream.utils.data_structures.MYSQL_INTEGER1024.max_length}}\pysigline{\sphinxbfcode{\sphinxupquote{max\_length}}\sphinxbfcode{\sphinxupquote{ = 1024}}}
\end{fulllineitems}


\end{fulllineitems}

\index{MYSQL\_INTEGER11 (class in nanostream.utils.data\_structures)}

\begin{fulllineitems}
\phantomsection\label{\detokenize{api:nanostream.utils.data_structures.MYSQL_INTEGER11}}\pysiglinewithargsret{\sphinxbfcode{\sphinxupquote{class }}\sphinxcode{\sphinxupquote{nanostream.utils.data\_structures.}}\sphinxbfcode{\sphinxupquote{MYSQL\_INTEGER11}}}{\emph{value}, \emph{original\_type=None}, \emph{name=None}}{}
Bases: {\hyperref[\detokenize{api:nanostream.utils.data_structures.MYSQL_INTEGER_BASE}]{\sphinxcrossref{\sphinxcode{\sphinxupquote{nanostream.utils.data\_structures.MYSQL\_INTEGER\_BASE}}}}}
\index{max\_length (nanostream.utils.data\_structures.MYSQL\_INTEGER11 attribute)}

\begin{fulllineitems}
\phantomsection\label{\detokenize{api:nanostream.utils.data_structures.MYSQL_INTEGER11.max_length}}\pysigline{\sphinxbfcode{\sphinxupquote{max\_length}}\sphinxbfcode{\sphinxupquote{ = 11}}}
\end{fulllineitems}


\end{fulllineitems}

\index{MYSQL\_INTEGER12 (class in nanostream.utils.data\_structures)}

\begin{fulllineitems}
\phantomsection\label{\detokenize{api:nanostream.utils.data_structures.MYSQL_INTEGER12}}\pysiglinewithargsret{\sphinxbfcode{\sphinxupquote{class }}\sphinxcode{\sphinxupquote{nanostream.utils.data\_structures.}}\sphinxbfcode{\sphinxupquote{MYSQL\_INTEGER12}}}{\emph{value}, \emph{original\_type=None}, \emph{name=None}}{}
Bases: {\hyperref[\detokenize{api:nanostream.utils.data_structures.MYSQL_INTEGER_BASE}]{\sphinxcrossref{\sphinxcode{\sphinxupquote{nanostream.utils.data\_structures.MYSQL\_INTEGER\_BASE}}}}}
\index{max\_length (nanostream.utils.data\_structures.MYSQL\_INTEGER12 attribute)}

\begin{fulllineitems}
\phantomsection\label{\detokenize{api:nanostream.utils.data_structures.MYSQL_INTEGER12.max_length}}\pysigline{\sphinxbfcode{\sphinxupquote{max\_length}}\sphinxbfcode{\sphinxupquote{ = 12}}}
\end{fulllineitems}


\end{fulllineitems}

\index{MYSQL\_INTEGER128 (class in nanostream.utils.data\_structures)}

\begin{fulllineitems}
\phantomsection\label{\detokenize{api:nanostream.utils.data_structures.MYSQL_INTEGER128}}\pysiglinewithargsret{\sphinxbfcode{\sphinxupquote{class }}\sphinxcode{\sphinxupquote{nanostream.utils.data\_structures.}}\sphinxbfcode{\sphinxupquote{MYSQL\_INTEGER128}}}{\emph{value}, \emph{original\_type=None}, \emph{name=None}}{}
Bases: {\hyperref[\detokenize{api:nanostream.utils.data_structures.MYSQL_INTEGER_BASE}]{\sphinxcrossref{\sphinxcode{\sphinxupquote{nanostream.utils.data\_structures.MYSQL\_INTEGER\_BASE}}}}}
\index{max\_length (nanostream.utils.data\_structures.MYSQL\_INTEGER128 attribute)}

\begin{fulllineitems}
\phantomsection\label{\detokenize{api:nanostream.utils.data_structures.MYSQL_INTEGER128.max_length}}\pysigline{\sphinxbfcode{\sphinxupquote{max\_length}}\sphinxbfcode{\sphinxupquote{ = 128}}}
\end{fulllineitems}


\end{fulllineitems}

\index{MYSQL\_INTEGER13 (class in nanostream.utils.data\_structures)}

\begin{fulllineitems}
\phantomsection\label{\detokenize{api:nanostream.utils.data_structures.MYSQL_INTEGER13}}\pysiglinewithargsret{\sphinxbfcode{\sphinxupquote{class }}\sphinxcode{\sphinxupquote{nanostream.utils.data\_structures.}}\sphinxbfcode{\sphinxupquote{MYSQL\_INTEGER13}}}{\emph{value}, \emph{original\_type=None}, \emph{name=None}}{}
Bases: {\hyperref[\detokenize{api:nanostream.utils.data_structures.MYSQL_INTEGER_BASE}]{\sphinxcrossref{\sphinxcode{\sphinxupquote{nanostream.utils.data\_structures.MYSQL\_INTEGER\_BASE}}}}}
\index{max\_length (nanostream.utils.data\_structures.MYSQL\_INTEGER13 attribute)}

\begin{fulllineitems}
\phantomsection\label{\detokenize{api:nanostream.utils.data_structures.MYSQL_INTEGER13.max_length}}\pysigline{\sphinxbfcode{\sphinxupquote{max\_length}}\sphinxbfcode{\sphinxupquote{ = 13}}}
\end{fulllineitems}


\end{fulllineitems}

\index{MYSQL\_INTEGER14 (class in nanostream.utils.data\_structures)}

\begin{fulllineitems}
\phantomsection\label{\detokenize{api:nanostream.utils.data_structures.MYSQL_INTEGER14}}\pysiglinewithargsret{\sphinxbfcode{\sphinxupquote{class }}\sphinxcode{\sphinxupquote{nanostream.utils.data\_structures.}}\sphinxbfcode{\sphinxupquote{MYSQL\_INTEGER14}}}{\emph{value}, \emph{original\_type=None}, \emph{name=None}}{}
Bases: {\hyperref[\detokenize{api:nanostream.utils.data_structures.MYSQL_INTEGER_BASE}]{\sphinxcrossref{\sphinxcode{\sphinxupquote{nanostream.utils.data\_structures.MYSQL\_INTEGER\_BASE}}}}}
\index{max\_length (nanostream.utils.data\_structures.MYSQL\_INTEGER14 attribute)}

\begin{fulllineitems}
\phantomsection\label{\detokenize{api:nanostream.utils.data_structures.MYSQL_INTEGER14.max_length}}\pysigline{\sphinxbfcode{\sphinxupquote{max\_length}}\sphinxbfcode{\sphinxupquote{ = 14}}}
\end{fulllineitems}


\end{fulllineitems}

\index{MYSQL\_INTEGER15 (class in nanostream.utils.data\_structures)}

\begin{fulllineitems}
\phantomsection\label{\detokenize{api:nanostream.utils.data_structures.MYSQL_INTEGER15}}\pysiglinewithargsret{\sphinxbfcode{\sphinxupquote{class }}\sphinxcode{\sphinxupquote{nanostream.utils.data\_structures.}}\sphinxbfcode{\sphinxupquote{MYSQL\_INTEGER15}}}{\emph{value}, \emph{original\_type=None}, \emph{name=None}}{}
Bases: {\hyperref[\detokenize{api:nanostream.utils.data_structures.MYSQL_INTEGER_BASE}]{\sphinxcrossref{\sphinxcode{\sphinxupquote{nanostream.utils.data\_structures.MYSQL\_INTEGER\_BASE}}}}}
\index{max\_length (nanostream.utils.data\_structures.MYSQL\_INTEGER15 attribute)}

\begin{fulllineitems}
\phantomsection\label{\detokenize{api:nanostream.utils.data_structures.MYSQL_INTEGER15.max_length}}\pysigline{\sphinxbfcode{\sphinxupquote{max\_length}}\sphinxbfcode{\sphinxupquote{ = 15}}}
\end{fulllineitems}


\end{fulllineitems}

\index{MYSQL\_INTEGER16 (class in nanostream.utils.data\_structures)}

\begin{fulllineitems}
\phantomsection\label{\detokenize{api:nanostream.utils.data_structures.MYSQL_INTEGER16}}\pysiglinewithargsret{\sphinxbfcode{\sphinxupquote{class }}\sphinxcode{\sphinxupquote{nanostream.utils.data\_structures.}}\sphinxbfcode{\sphinxupquote{MYSQL\_INTEGER16}}}{\emph{value}, \emph{original\_type=None}, \emph{name=None}}{}
Bases: {\hyperref[\detokenize{api:nanostream.utils.data_structures.MYSQL_INTEGER_BASE}]{\sphinxcrossref{\sphinxcode{\sphinxupquote{nanostream.utils.data\_structures.MYSQL\_INTEGER\_BASE}}}}}
\index{max\_length (nanostream.utils.data\_structures.MYSQL\_INTEGER16 attribute)}

\begin{fulllineitems}
\phantomsection\label{\detokenize{api:nanostream.utils.data_structures.MYSQL_INTEGER16.max_length}}\pysigline{\sphinxbfcode{\sphinxupquote{max\_length}}\sphinxbfcode{\sphinxupquote{ = 16}}}
\end{fulllineitems}


\end{fulllineitems}

\index{MYSQL\_INTEGER16384 (class in nanostream.utils.data\_structures)}

\begin{fulllineitems}
\phantomsection\label{\detokenize{api:nanostream.utils.data_structures.MYSQL_INTEGER16384}}\pysiglinewithargsret{\sphinxbfcode{\sphinxupquote{class }}\sphinxcode{\sphinxupquote{nanostream.utils.data\_structures.}}\sphinxbfcode{\sphinxupquote{MYSQL\_INTEGER16384}}}{\emph{value}, \emph{original\_type=None}, \emph{name=None}}{}
Bases: {\hyperref[\detokenize{api:nanostream.utils.data_structures.MYSQL_INTEGER_BASE}]{\sphinxcrossref{\sphinxcode{\sphinxupquote{nanostream.utils.data\_structures.MYSQL\_INTEGER\_BASE}}}}}
\index{max\_length (nanostream.utils.data\_structures.MYSQL\_INTEGER16384 attribute)}

\begin{fulllineitems}
\phantomsection\label{\detokenize{api:nanostream.utils.data_structures.MYSQL_INTEGER16384.max_length}}\pysigline{\sphinxbfcode{\sphinxupquote{max\_length}}\sphinxbfcode{\sphinxupquote{ = 16384}}}
\end{fulllineitems}


\end{fulllineitems}

\index{MYSQL\_INTEGER17 (class in nanostream.utils.data\_structures)}

\begin{fulllineitems}
\phantomsection\label{\detokenize{api:nanostream.utils.data_structures.MYSQL_INTEGER17}}\pysiglinewithargsret{\sphinxbfcode{\sphinxupquote{class }}\sphinxcode{\sphinxupquote{nanostream.utils.data\_structures.}}\sphinxbfcode{\sphinxupquote{MYSQL\_INTEGER17}}}{\emph{value}, \emph{original\_type=None}, \emph{name=None}}{}
Bases: {\hyperref[\detokenize{api:nanostream.utils.data_structures.MYSQL_INTEGER_BASE}]{\sphinxcrossref{\sphinxcode{\sphinxupquote{nanostream.utils.data\_structures.MYSQL\_INTEGER\_BASE}}}}}
\index{max\_length (nanostream.utils.data\_structures.MYSQL\_INTEGER17 attribute)}

\begin{fulllineitems}
\phantomsection\label{\detokenize{api:nanostream.utils.data_structures.MYSQL_INTEGER17.max_length}}\pysigline{\sphinxbfcode{\sphinxupquote{max\_length}}\sphinxbfcode{\sphinxupquote{ = 17}}}
\end{fulllineitems}


\end{fulllineitems}

\index{MYSQL\_INTEGER18 (class in nanostream.utils.data\_structures)}

\begin{fulllineitems}
\phantomsection\label{\detokenize{api:nanostream.utils.data_structures.MYSQL_INTEGER18}}\pysiglinewithargsret{\sphinxbfcode{\sphinxupquote{class }}\sphinxcode{\sphinxupquote{nanostream.utils.data\_structures.}}\sphinxbfcode{\sphinxupquote{MYSQL\_INTEGER18}}}{\emph{value}, \emph{original\_type=None}, \emph{name=None}}{}
Bases: {\hyperref[\detokenize{api:nanostream.utils.data_structures.MYSQL_INTEGER_BASE}]{\sphinxcrossref{\sphinxcode{\sphinxupquote{nanostream.utils.data\_structures.MYSQL\_INTEGER\_BASE}}}}}
\index{max\_length (nanostream.utils.data\_structures.MYSQL\_INTEGER18 attribute)}

\begin{fulllineitems}
\phantomsection\label{\detokenize{api:nanostream.utils.data_structures.MYSQL_INTEGER18.max_length}}\pysigline{\sphinxbfcode{\sphinxupquote{max\_length}}\sphinxbfcode{\sphinxupquote{ = 18}}}
\end{fulllineitems}


\end{fulllineitems}

\index{MYSQL\_INTEGER19 (class in nanostream.utils.data\_structures)}

\begin{fulllineitems}
\phantomsection\label{\detokenize{api:nanostream.utils.data_structures.MYSQL_INTEGER19}}\pysiglinewithargsret{\sphinxbfcode{\sphinxupquote{class }}\sphinxcode{\sphinxupquote{nanostream.utils.data\_structures.}}\sphinxbfcode{\sphinxupquote{MYSQL\_INTEGER19}}}{\emph{value}, \emph{original\_type=None}, \emph{name=None}}{}
Bases: {\hyperref[\detokenize{api:nanostream.utils.data_structures.MYSQL_INTEGER_BASE}]{\sphinxcrossref{\sphinxcode{\sphinxupquote{nanostream.utils.data\_structures.MYSQL\_INTEGER\_BASE}}}}}
\index{max\_length (nanostream.utils.data\_structures.MYSQL\_INTEGER19 attribute)}

\begin{fulllineitems}
\phantomsection\label{\detokenize{api:nanostream.utils.data_structures.MYSQL_INTEGER19.max_length}}\pysigline{\sphinxbfcode{\sphinxupquote{max\_length}}\sphinxbfcode{\sphinxupquote{ = 19}}}
\end{fulllineitems}


\end{fulllineitems}

\index{MYSQL\_INTEGER2 (class in nanostream.utils.data\_structures)}

\begin{fulllineitems}
\phantomsection\label{\detokenize{api:nanostream.utils.data_structures.MYSQL_INTEGER2}}\pysiglinewithargsret{\sphinxbfcode{\sphinxupquote{class }}\sphinxcode{\sphinxupquote{nanostream.utils.data\_structures.}}\sphinxbfcode{\sphinxupquote{MYSQL\_INTEGER2}}}{\emph{value}, \emph{original\_type=None}, \emph{name=None}}{}
Bases: {\hyperref[\detokenize{api:nanostream.utils.data_structures.MYSQL_INTEGER_BASE}]{\sphinxcrossref{\sphinxcode{\sphinxupquote{nanostream.utils.data\_structures.MYSQL\_INTEGER\_BASE}}}}}
\index{max\_length (nanostream.utils.data\_structures.MYSQL\_INTEGER2 attribute)}

\begin{fulllineitems}
\phantomsection\label{\detokenize{api:nanostream.utils.data_structures.MYSQL_INTEGER2.max_length}}\pysigline{\sphinxbfcode{\sphinxupquote{max\_length}}\sphinxbfcode{\sphinxupquote{ = 2}}}
\end{fulllineitems}


\end{fulllineitems}

\index{MYSQL\_INTEGER20 (class in nanostream.utils.data\_structures)}

\begin{fulllineitems}
\phantomsection\label{\detokenize{api:nanostream.utils.data_structures.MYSQL_INTEGER20}}\pysiglinewithargsret{\sphinxbfcode{\sphinxupquote{class }}\sphinxcode{\sphinxupquote{nanostream.utils.data\_structures.}}\sphinxbfcode{\sphinxupquote{MYSQL\_INTEGER20}}}{\emph{value}, \emph{original\_type=None}, \emph{name=None}}{}
Bases: {\hyperref[\detokenize{api:nanostream.utils.data_structures.MYSQL_INTEGER_BASE}]{\sphinxcrossref{\sphinxcode{\sphinxupquote{nanostream.utils.data\_structures.MYSQL\_INTEGER\_BASE}}}}}
\index{max\_length (nanostream.utils.data\_structures.MYSQL\_INTEGER20 attribute)}

\begin{fulllineitems}
\phantomsection\label{\detokenize{api:nanostream.utils.data_structures.MYSQL_INTEGER20.max_length}}\pysigline{\sphinxbfcode{\sphinxupquote{max\_length}}\sphinxbfcode{\sphinxupquote{ = 20}}}
\end{fulllineitems}


\end{fulllineitems}

\index{MYSQL\_INTEGER2048 (class in nanostream.utils.data\_structures)}

\begin{fulllineitems}
\phantomsection\label{\detokenize{api:nanostream.utils.data_structures.MYSQL_INTEGER2048}}\pysiglinewithargsret{\sphinxbfcode{\sphinxupquote{class }}\sphinxcode{\sphinxupquote{nanostream.utils.data\_structures.}}\sphinxbfcode{\sphinxupquote{MYSQL\_INTEGER2048}}}{\emph{value}, \emph{original\_type=None}, \emph{name=None}}{}
Bases: {\hyperref[\detokenize{api:nanostream.utils.data_structures.MYSQL_INTEGER_BASE}]{\sphinxcrossref{\sphinxcode{\sphinxupquote{nanostream.utils.data\_structures.MYSQL\_INTEGER\_BASE}}}}}
\index{max\_length (nanostream.utils.data\_structures.MYSQL\_INTEGER2048 attribute)}

\begin{fulllineitems}
\phantomsection\label{\detokenize{api:nanostream.utils.data_structures.MYSQL_INTEGER2048.max_length}}\pysigline{\sphinxbfcode{\sphinxupquote{max\_length}}\sphinxbfcode{\sphinxupquote{ = 2048}}}
\end{fulllineitems}


\end{fulllineitems}

\index{MYSQL\_INTEGER21 (class in nanostream.utils.data\_structures)}

\begin{fulllineitems}
\phantomsection\label{\detokenize{api:nanostream.utils.data_structures.MYSQL_INTEGER21}}\pysiglinewithargsret{\sphinxbfcode{\sphinxupquote{class }}\sphinxcode{\sphinxupquote{nanostream.utils.data\_structures.}}\sphinxbfcode{\sphinxupquote{MYSQL\_INTEGER21}}}{\emph{value}, \emph{original\_type=None}, \emph{name=None}}{}
Bases: {\hyperref[\detokenize{api:nanostream.utils.data_structures.MYSQL_INTEGER_BASE}]{\sphinxcrossref{\sphinxcode{\sphinxupquote{nanostream.utils.data\_structures.MYSQL\_INTEGER\_BASE}}}}}
\index{max\_length (nanostream.utils.data\_structures.MYSQL\_INTEGER21 attribute)}

\begin{fulllineitems}
\phantomsection\label{\detokenize{api:nanostream.utils.data_structures.MYSQL_INTEGER21.max_length}}\pysigline{\sphinxbfcode{\sphinxupquote{max\_length}}\sphinxbfcode{\sphinxupquote{ = 21}}}
\end{fulllineitems}


\end{fulllineitems}

\index{MYSQL\_INTEGER22 (class in nanostream.utils.data\_structures)}

\begin{fulllineitems}
\phantomsection\label{\detokenize{api:nanostream.utils.data_structures.MYSQL_INTEGER22}}\pysiglinewithargsret{\sphinxbfcode{\sphinxupquote{class }}\sphinxcode{\sphinxupquote{nanostream.utils.data\_structures.}}\sphinxbfcode{\sphinxupquote{MYSQL\_INTEGER22}}}{\emph{value}, \emph{original\_type=None}, \emph{name=None}}{}
Bases: {\hyperref[\detokenize{api:nanostream.utils.data_structures.MYSQL_INTEGER_BASE}]{\sphinxcrossref{\sphinxcode{\sphinxupquote{nanostream.utils.data\_structures.MYSQL\_INTEGER\_BASE}}}}}
\index{max\_length (nanostream.utils.data\_structures.MYSQL\_INTEGER22 attribute)}

\begin{fulllineitems}
\phantomsection\label{\detokenize{api:nanostream.utils.data_structures.MYSQL_INTEGER22.max_length}}\pysigline{\sphinxbfcode{\sphinxupquote{max\_length}}\sphinxbfcode{\sphinxupquote{ = 22}}}
\end{fulllineitems}


\end{fulllineitems}

\index{MYSQL\_INTEGER23 (class in nanostream.utils.data\_structures)}

\begin{fulllineitems}
\phantomsection\label{\detokenize{api:nanostream.utils.data_structures.MYSQL_INTEGER23}}\pysiglinewithargsret{\sphinxbfcode{\sphinxupquote{class }}\sphinxcode{\sphinxupquote{nanostream.utils.data\_structures.}}\sphinxbfcode{\sphinxupquote{MYSQL\_INTEGER23}}}{\emph{value}, \emph{original\_type=None}, \emph{name=None}}{}
Bases: {\hyperref[\detokenize{api:nanostream.utils.data_structures.MYSQL_INTEGER_BASE}]{\sphinxcrossref{\sphinxcode{\sphinxupquote{nanostream.utils.data\_structures.MYSQL\_INTEGER\_BASE}}}}}
\index{max\_length (nanostream.utils.data\_structures.MYSQL\_INTEGER23 attribute)}

\begin{fulllineitems}
\phantomsection\label{\detokenize{api:nanostream.utils.data_structures.MYSQL_INTEGER23.max_length}}\pysigline{\sphinxbfcode{\sphinxupquote{max\_length}}\sphinxbfcode{\sphinxupquote{ = 23}}}
\end{fulllineitems}


\end{fulllineitems}

\index{MYSQL\_INTEGER24 (class in nanostream.utils.data\_structures)}

\begin{fulllineitems}
\phantomsection\label{\detokenize{api:nanostream.utils.data_structures.MYSQL_INTEGER24}}\pysiglinewithargsret{\sphinxbfcode{\sphinxupquote{class }}\sphinxcode{\sphinxupquote{nanostream.utils.data\_structures.}}\sphinxbfcode{\sphinxupquote{MYSQL\_INTEGER24}}}{\emph{value}, \emph{original\_type=None}, \emph{name=None}}{}
Bases: {\hyperref[\detokenize{api:nanostream.utils.data_structures.MYSQL_INTEGER_BASE}]{\sphinxcrossref{\sphinxcode{\sphinxupquote{nanostream.utils.data\_structures.MYSQL\_INTEGER\_BASE}}}}}
\index{max\_length (nanostream.utils.data\_structures.MYSQL\_INTEGER24 attribute)}

\begin{fulllineitems}
\phantomsection\label{\detokenize{api:nanostream.utils.data_structures.MYSQL_INTEGER24.max_length}}\pysigline{\sphinxbfcode{\sphinxupquote{max\_length}}\sphinxbfcode{\sphinxupquote{ = 24}}}
\end{fulllineitems}


\end{fulllineitems}

\index{MYSQL\_INTEGER25 (class in nanostream.utils.data\_structures)}

\begin{fulllineitems}
\phantomsection\label{\detokenize{api:nanostream.utils.data_structures.MYSQL_INTEGER25}}\pysiglinewithargsret{\sphinxbfcode{\sphinxupquote{class }}\sphinxcode{\sphinxupquote{nanostream.utils.data\_structures.}}\sphinxbfcode{\sphinxupquote{MYSQL\_INTEGER25}}}{\emph{value}, \emph{original\_type=None}, \emph{name=None}}{}
Bases: {\hyperref[\detokenize{api:nanostream.utils.data_structures.MYSQL_INTEGER_BASE}]{\sphinxcrossref{\sphinxcode{\sphinxupquote{nanostream.utils.data\_structures.MYSQL\_INTEGER\_BASE}}}}}
\index{max\_length (nanostream.utils.data\_structures.MYSQL\_INTEGER25 attribute)}

\begin{fulllineitems}
\phantomsection\label{\detokenize{api:nanostream.utils.data_structures.MYSQL_INTEGER25.max_length}}\pysigline{\sphinxbfcode{\sphinxupquote{max\_length}}\sphinxbfcode{\sphinxupquote{ = 25}}}
\end{fulllineitems}


\end{fulllineitems}

\index{MYSQL\_INTEGER256 (class in nanostream.utils.data\_structures)}

\begin{fulllineitems}
\phantomsection\label{\detokenize{api:nanostream.utils.data_structures.MYSQL_INTEGER256}}\pysiglinewithargsret{\sphinxbfcode{\sphinxupquote{class }}\sphinxcode{\sphinxupquote{nanostream.utils.data\_structures.}}\sphinxbfcode{\sphinxupquote{MYSQL\_INTEGER256}}}{\emph{value}, \emph{original\_type=None}, \emph{name=None}}{}
Bases: {\hyperref[\detokenize{api:nanostream.utils.data_structures.MYSQL_INTEGER_BASE}]{\sphinxcrossref{\sphinxcode{\sphinxupquote{nanostream.utils.data\_structures.MYSQL\_INTEGER\_BASE}}}}}
\index{max\_length (nanostream.utils.data\_structures.MYSQL\_INTEGER256 attribute)}

\begin{fulllineitems}
\phantomsection\label{\detokenize{api:nanostream.utils.data_structures.MYSQL_INTEGER256.max_length}}\pysigline{\sphinxbfcode{\sphinxupquote{max\_length}}\sphinxbfcode{\sphinxupquote{ = 256}}}
\end{fulllineitems}


\end{fulllineitems}

\index{MYSQL\_INTEGER26 (class in nanostream.utils.data\_structures)}

\begin{fulllineitems}
\phantomsection\label{\detokenize{api:nanostream.utils.data_structures.MYSQL_INTEGER26}}\pysiglinewithargsret{\sphinxbfcode{\sphinxupquote{class }}\sphinxcode{\sphinxupquote{nanostream.utils.data\_structures.}}\sphinxbfcode{\sphinxupquote{MYSQL\_INTEGER26}}}{\emph{value}, \emph{original\_type=None}, \emph{name=None}}{}
Bases: {\hyperref[\detokenize{api:nanostream.utils.data_structures.MYSQL_INTEGER_BASE}]{\sphinxcrossref{\sphinxcode{\sphinxupquote{nanostream.utils.data\_structures.MYSQL\_INTEGER\_BASE}}}}}
\index{max\_length (nanostream.utils.data\_structures.MYSQL\_INTEGER26 attribute)}

\begin{fulllineitems}
\phantomsection\label{\detokenize{api:nanostream.utils.data_structures.MYSQL_INTEGER26.max_length}}\pysigline{\sphinxbfcode{\sphinxupquote{max\_length}}\sphinxbfcode{\sphinxupquote{ = 26}}}
\end{fulllineitems}


\end{fulllineitems}

\index{MYSQL\_INTEGER27 (class in nanostream.utils.data\_structures)}

\begin{fulllineitems}
\phantomsection\label{\detokenize{api:nanostream.utils.data_structures.MYSQL_INTEGER27}}\pysiglinewithargsret{\sphinxbfcode{\sphinxupquote{class }}\sphinxcode{\sphinxupquote{nanostream.utils.data\_structures.}}\sphinxbfcode{\sphinxupquote{MYSQL\_INTEGER27}}}{\emph{value}, \emph{original\_type=None}, \emph{name=None}}{}
Bases: {\hyperref[\detokenize{api:nanostream.utils.data_structures.MYSQL_INTEGER_BASE}]{\sphinxcrossref{\sphinxcode{\sphinxupquote{nanostream.utils.data\_structures.MYSQL\_INTEGER\_BASE}}}}}
\index{max\_length (nanostream.utils.data\_structures.MYSQL\_INTEGER27 attribute)}

\begin{fulllineitems}
\phantomsection\label{\detokenize{api:nanostream.utils.data_structures.MYSQL_INTEGER27.max_length}}\pysigline{\sphinxbfcode{\sphinxupquote{max\_length}}\sphinxbfcode{\sphinxupquote{ = 27}}}
\end{fulllineitems}


\end{fulllineitems}

\index{MYSQL\_INTEGER28 (class in nanostream.utils.data\_structures)}

\begin{fulllineitems}
\phantomsection\label{\detokenize{api:nanostream.utils.data_structures.MYSQL_INTEGER28}}\pysiglinewithargsret{\sphinxbfcode{\sphinxupquote{class }}\sphinxcode{\sphinxupquote{nanostream.utils.data\_structures.}}\sphinxbfcode{\sphinxupquote{MYSQL\_INTEGER28}}}{\emph{value}, \emph{original\_type=None}, \emph{name=None}}{}
Bases: {\hyperref[\detokenize{api:nanostream.utils.data_structures.MYSQL_INTEGER_BASE}]{\sphinxcrossref{\sphinxcode{\sphinxupquote{nanostream.utils.data\_structures.MYSQL\_INTEGER\_BASE}}}}}
\index{max\_length (nanostream.utils.data\_structures.MYSQL\_INTEGER28 attribute)}

\begin{fulllineitems}
\phantomsection\label{\detokenize{api:nanostream.utils.data_structures.MYSQL_INTEGER28.max_length}}\pysigline{\sphinxbfcode{\sphinxupquote{max\_length}}\sphinxbfcode{\sphinxupquote{ = 28}}}
\end{fulllineitems}


\end{fulllineitems}

\index{MYSQL\_INTEGER29 (class in nanostream.utils.data\_structures)}

\begin{fulllineitems}
\phantomsection\label{\detokenize{api:nanostream.utils.data_structures.MYSQL_INTEGER29}}\pysiglinewithargsret{\sphinxbfcode{\sphinxupquote{class }}\sphinxcode{\sphinxupquote{nanostream.utils.data\_structures.}}\sphinxbfcode{\sphinxupquote{MYSQL\_INTEGER29}}}{\emph{value}, \emph{original\_type=None}, \emph{name=None}}{}
Bases: {\hyperref[\detokenize{api:nanostream.utils.data_structures.MYSQL_INTEGER_BASE}]{\sphinxcrossref{\sphinxcode{\sphinxupquote{nanostream.utils.data\_structures.MYSQL\_INTEGER\_BASE}}}}}
\index{max\_length (nanostream.utils.data\_structures.MYSQL\_INTEGER29 attribute)}

\begin{fulllineitems}
\phantomsection\label{\detokenize{api:nanostream.utils.data_structures.MYSQL_INTEGER29.max_length}}\pysigline{\sphinxbfcode{\sphinxupquote{max\_length}}\sphinxbfcode{\sphinxupquote{ = 29}}}
\end{fulllineitems}


\end{fulllineitems}

\index{MYSQL\_INTEGER3 (class in nanostream.utils.data\_structures)}

\begin{fulllineitems}
\phantomsection\label{\detokenize{api:nanostream.utils.data_structures.MYSQL_INTEGER3}}\pysiglinewithargsret{\sphinxbfcode{\sphinxupquote{class }}\sphinxcode{\sphinxupquote{nanostream.utils.data\_structures.}}\sphinxbfcode{\sphinxupquote{MYSQL\_INTEGER3}}}{\emph{value}, \emph{original\_type=None}, \emph{name=None}}{}
Bases: {\hyperref[\detokenize{api:nanostream.utils.data_structures.MYSQL_INTEGER_BASE}]{\sphinxcrossref{\sphinxcode{\sphinxupquote{nanostream.utils.data\_structures.MYSQL\_INTEGER\_BASE}}}}}
\index{max\_length (nanostream.utils.data\_structures.MYSQL\_INTEGER3 attribute)}

\begin{fulllineitems}
\phantomsection\label{\detokenize{api:nanostream.utils.data_structures.MYSQL_INTEGER3.max_length}}\pysigline{\sphinxbfcode{\sphinxupquote{max\_length}}\sphinxbfcode{\sphinxupquote{ = 3}}}
\end{fulllineitems}


\end{fulllineitems}

\index{MYSQL\_INTEGER30 (class in nanostream.utils.data\_structures)}

\begin{fulllineitems}
\phantomsection\label{\detokenize{api:nanostream.utils.data_structures.MYSQL_INTEGER30}}\pysiglinewithargsret{\sphinxbfcode{\sphinxupquote{class }}\sphinxcode{\sphinxupquote{nanostream.utils.data\_structures.}}\sphinxbfcode{\sphinxupquote{MYSQL\_INTEGER30}}}{\emph{value}, \emph{original\_type=None}, \emph{name=None}}{}
Bases: {\hyperref[\detokenize{api:nanostream.utils.data_structures.MYSQL_INTEGER_BASE}]{\sphinxcrossref{\sphinxcode{\sphinxupquote{nanostream.utils.data\_structures.MYSQL\_INTEGER\_BASE}}}}}
\index{max\_length (nanostream.utils.data\_structures.MYSQL\_INTEGER30 attribute)}

\begin{fulllineitems}
\phantomsection\label{\detokenize{api:nanostream.utils.data_structures.MYSQL_INTEGER30.max_length}}\pysigline{\sphinxbfcode{\sphinxupquote{max\_length}}\sphinxbfcode{\sphinxupquote{ = 30}}}
\end{fulllineitems}


\end{fulllineitems}

\index{MYSQL\_INTEGER31 (class in nanostream.utils.data\_structures)}

\begin{fulllineitems}
\phantomsection\label{\detokenize{api:nanostream.utils.data_structures.MYSQL_INTEGER31}}\pysiglinewithargsret{\sphinxbfcode{\sphinxupquote{class }}\sphinxcode{\sphinxupquote{nanostream.utils.data\_structures.}}\sphinxbfcode{\sphinxupquote{MYSQL\_INTEGER31}}}{\emph{value}, \emph{original\_type=None}, \emph{name=None}}{}
Bases: {\hyperref[\detokenize{api:nanostream.utils.data_structures.MYSQL_INTEGER_BASE}]{\sphinxcrossref{\sphinxcode{\sphinxupquote{nanostream.utils.data\_structures.MYSQL\_INTEGER\_BASE}}}}}
\index{max\_length (nanostream.utils.data\_structures.MYSQL\_INTEGER31 attribute)}

\begin{fulllineitems}
\phantomsection\label{\detokenize{api:nanostream.utils.data_structures.MYSQL_INTEGER31.max_length}}\pysigline{\sphinxbfcode{\sphinxupquote{max\_length}}\sphinxbfcode{\sphinxupquote{ = 31}}}
\end{fulllineitems}


\end{fulllineitems}

\index{MYSQL\_INTEGER32 (class in nanostream.utils.data\_structures)}

\begin{fulllineitems}
\phantomsection\label{\detokenize{api:nanostream.utils.data_structures.MYSQL_INTEGER32}}\pysiglinewithargsret{\sphinxbfcode{\sphinxupquote{class }}\sphinxcode{\sphinxupquote{nanostream.utils.data\_structures.}}\sphinxbfcode{\sphinxupquote{MYSQL\_INTEGER32}}}{\emph{value}, \emph{original\_type=None}, \emph{name=None}}{}
Bases: {\hyperref[\detokenize{api:nanostream.utils.data_structures.MYSQL_INTEGER_BASE}]{\sphinxcrossref{\sphinxcode{\sphinxupquote{nanostream.utils.data\_structures.MYSQL\_INTEGER\_BASE}}}}}
\index{max\_length (nanostream.utils.data\_structures.MYSQL\_INTEGER32 attribute)}

\begin{fulllineitems}
\phantomsection\label{\detokenize{api:nanostream.utils.data_structures.MYSQL_INTEGER32.max_length}}\pysigline{\sphinxbfcode{\sphinxupquote{max\_length}}\sphinxbfcode{\sphinxupquote{ = 32}}}
\end{fulllineitems}


\end{fulllineitems}

\index{MYSQL\_INTEGER32768 (class in nanostream.utils.data\_structures)}

\begin{fulllineitems}
\phantomsection\label{\detokenize{api:nanostream.utils.data_structures.MYSQL_INTEGER32768}}\pysiglinewithargsret{\sphinxbfcode{\sphinxupquote{class }}\sphinxcode{\sphinxupquote{nanostream.utils.data\_structures.}}\sphinxbfcode{\sphinxupquote{MYSQL\_INTEGER32768}}}{\emph{value}, \emph{original\_type=None}, \emph{name=None}}{}
Bases: {\hyperref[\detokenize{api:nanostream.utils.data_structures.MYSQL_INTEGER_BASE}]{\sphinxcrossref{\sphinxcode{\sphinxupquote{nanostream.utils.data\_structures.MYSQL\_INTEGER\_BASE}}}}}
\index{max\_length (nanostream.utils.data\_structures.MYSQL\_INTEGER32768 attribute)}

\begin{fulllineitems}
\phantomsection\label{\detokenize{api:nanostream.utils.data_structures.MYSQL_INTEGER32768.max_length}}\pysigline{\sphinxbfcode{\sphinxupquote{max\_length}}\sphinxbfcode{\sphinxupquote{ = 32768}}}
\end{fulllineitems}


\end{fulllineitems}

\index{MYSQL\_INTEGER4 (class in nanostream.utils.data\_structures)}

\begin{fulllineitems}
\phantomsection\label{\detokenize{api:nanostream.utils.data_structures.MYSQL_INTEGER4}}\pysiglinewithargsret{\sphinxbfcode{\sphinxupquote{class }}\sphinxcode{\sphinxupquote{nanostream.utils.data\_structures.}}\sphinxbfcode{\sphinxupquote{MYSQL\_INTEGER4}}}{\emph{value}, \emph{original\_type=None}, \emph{name=None}}{}
Bases: {\hyperref[\detokenize{api:nanostream.utils.data_structures.MYSQL_INTEGER_BASE}]{\sphinxcrossref{\sphinxcode{\sphinxupquote{nanostream.utils.data\_structures.MYSQL\_INTEGER\_BASE}}}}}
\index{max\_length (nanostream.utils.data\_structures.MYSQL\_INTEGER4 attribute)}

\begin{fulllineitems}
\phantomsection\label{\detokenize{api:nanostream.utils.data_structures.MYSQL_INTEGER4.max_length}}\pysigline{\sphinxbfcode{\sphinxupquote{max\_length}}\sphinxbfcode{\sphinxupquote{ = 4}}}
\end{fulllineitems}


\end{fulllineitems}

\index{MYSQL\_INTEGER4096 (class in nanostream.utils.data\_structures)}

\begin{fulllineitems}
\phantomsection\label{\detokenize{api:nanostream.utils.data_structures.MYSQL_INTEGER4096}}\pysiglinewithargsret{\sphinxbfcode{\sphinxupquote{class }}\sphinxcode{\sphinxupquote{nanostream.utils.data\_structures.}}\sphinxbfcode{\sphinxupquote{MYSQL\_INTEGER4096}}}{\emph{value}, \emph{original\_type=None}, \emph{name=None}}{}
Bases: {\hyperref[\detokenize{api:nanostream.utils.data_structures.MYSQL_INTEGER_BASE}]{\sphinxcrossref{\sphinxcode{\sphinxupquote{nanostream.utils.data\_structures.MYSQL\_INTEGER\_BASE}}}}}
\index{max\_length (nanostream.utils.data\_structures.MYSQL\_INTEGER4096 attribute)}

\begin{fulllineitems}
\phantomsection\label{\detokenize{api:nanostream.utils.data_structures.MYSQL_INTEGER4096.max_length}}\pysigline{\sphinxbfcode{\sphinxupquote{max\_length}}\sphinxbfcode{\sphinxupquote{ = 4096}}}
\end{fulllineitems}


\end{fulllineitems}

\index{MYSQL\_INTEGER5 (class in nanostream.utils.data\_structures)}

\begin{fulllineitems}
\phantomsection\label{\detokenize{api:nanostream.utils.data_structures.MYSQL_INTEGER5}}\pysiglinewithargsret{\sphinxbfcode{\sphinxupquote{class }}\sphinxcode{\sphinxupquote{nanostream.utils.data\_structures.}}\sphinxbfcode{\sphinxupquote{MYSQL\_INTEGER5}}}{\emph{value}, \emph{original\_type=None}, \emph{name=None}}{}
Bases: {\hyperref[\detokenize{api:nanostream.utils.data_structures.MYSQL_INTEGER_BASE}]{\sphinxcrossref{\sphinxcode{\sphinxupquote{nanostream.utils.data\_structures.MYSQL\_INTEGER\_BASE}}}}}
\index{max\_length (nanostream.utils.data\_structures.MYSQL\_INTEGER5 attribute)}

\begin{fulllineitems}
\phantomsection\label{\detokenize{api:nanostream.utils.data_structures.MYSQL_INTEGER5.max_length}}\pysigline{\sphinxbfcode{\sphinxupquote{max\_length}}\sphinxbfcode{\sphinxupquote{ = 5}}}
\end{fulllineitems}


\end{fulllineitems}

\index{MYSQL\_INTEGER512 (class in nanostream.utils.data\_structures)}

\begin{fulllineitems}
\phantomsection\label{\detokenize{api:nanostream.utils.data_structures.MYSQL_INTEGER512}}\pysiglinewithargsret{\sphinxbfcode{\sphinxupquote{class }}\sphinxcode{\sphinxupquote{nanostream.utils.data\_structures.}}\sphinxbfcode{\sphinxupquote{MYSQL\_INTEGER512}}}{\emph{value}, \emph{original\_type=None}, \emph{name=None}}{}
Bases: {\hyperref[\detokenize{api:nanostream.utils.data_structures.MYSQL_INTEGER_BASE}]{\sphinxcrossref{\sphinxcode{\sphinxupquote{nanostream.utils.data\_structures.MYSQL\_INTEGER\_BASE}}}}}
\index{max\_length (nanostream.utils.data\_structures.MYSQL\_INTEGER512 attribute)}

\begin{fulllineitems}
\phantomsection\label{\detokenize{api:nanostream.utils.data_structures.MYSQL_INTEGER512.max_length}}\pysigline{\sphinxbfcode{\sphinxupquote{max\_length}}\sphinxbfcode{\sphinxupquote{ = 512}}}
\end{fulllineitems}


\end{fulllineitems}

\index{MYSQL\_INTEGER6 (class in nanostream.utils.data\_structures)}

\begin{fulllineitems}
\phantomsection\label{\detokenize{api:nanostream.utils.data_structures.MYSQL_INTEGER6}}\pysiglinewithargsret{\sphinxbfcode{\sphinxupquote{class }}\sphinxcode{\sphinxupquote{nanostream.utils.data\_structures.}}\sphinxbfcode{\sphinxupquote{MYSQL\_INTEGER6}}}{\emph{value}, \emph{original\_type=None}, \emph{name=None}}{}
Bases: {\hyperref[\detokenize{api:nanostream.utils.data_structures.MYSQL_INTEGER_BASE}]{\sphinxcrossref{\sphinxcode{\sphinxupquote{nanostream.utils.data\_structures.MYSQL\_INTEGER\_BASE}}}}}
\index{max\_length (nanostream.utils.data\_structures.MYSQL\_INTEGER6 attribute)}

\begin{fulllineitems}
\phantomsection\label{\detokenize{api:nanostream.utils.data_structures.MYSQL_INTEGER6.max_length}}\pysigline{\sphinxbfcode{\sphinxupquote{max\_length}}\sphinxbfcode{\sphinxupquote{ = 6}}}
\end{fulllineitems}


\end{fulllineitems}

\index{MYSQL\_INTEGER64 (class in nanostream.utils.data\_structures)}

\begin{fulllineitems}
\phantomsection\label{\detokenize{api:nanostream.utils.data_structures.MYSQL_INTEGER64}}\pysiglinewithargsret{\sphinxbfcode{\sphinxupquote{class }}\sphinxcode{\sphinxupquote{nanostream.utils.data\_structures.}}\sphinxbfcode{\sphinxupquote{MYSQL\_INTEGER64}}}{\emph{value}, \emph{original\_type=None}, \emph{name=None}}{}
Bases: {\hyperref[\detokenize{api:nanostream.utils.data_structures.MYSQL_INTEGER_BASE}]{\sphinxcrossref{\sphinxcode{\sphinxupquote{nanostream.utils.data\_structures.MYSQL\_INTEGER\_BASE}}}}}
\index{max\_length (nanostream.utils.data\_structures.MYSQL\_INTEGER64 attribute)}

\begin{fulllineitems}
\phantomsection\label{\detokenize{api:nanostream.utils.data_structures.MYSQL_INTEGER64.max_length}}\pysigline{\sphinxbfcode{\sphinxupquote{max\_length}}\sphinxbfcode{\sphinxupquote{ = 64}}}
\end{fulllineitems}


\end{fulllineitems}

\index{MYSQL\_INTEGER7 (class in nanostream.utils.data\_structures)}

\begin{fulllineitems}
\phantomsection\label{\detokenize{api:nanostream.utils.data_structures.MYSQL_INTEGER7}}\pysiglinewithargsret{\sphinxbfcode{\sphinxupquote{class }}\sphinxcode{\sphinxupquote{nanostream.utils.data\_structures.}}\sphinxbfcode{\sphinxupquote{MYSQL\_INTEGER7}}}{\emph{value}, \emph{original\_type=None}, \emph{name=None}}{}
Bases: {\hyperref[\detokenize{api:nanostream.utils.data_structures.MYSQL_INTEGER_BASE}]{\sphinxcrossref{\sphinxcode{\sphinxupquote{nanostream.utils.data\_structures.MYSQL\_INTEGER\_BASE}}}}}
\index{max\_length (nanostream.utils.data\_structures.MYSQL\_INTEGER7 attribute)}

\begin{fulllineitems}
\phantomsection\label{\detokenize{api:nanostream.utils.data_structures.MYSQL_INTEGER7.max_length}}\pysigline{\sphinxbfcode{\sphinxupquote{max\_length}}\sphinxbfcode{\sphinxupquote{ = 7}}}
\end{fulllineitems}


\end{fulllineitems}

\index{MYSQL\_INTEGER8 (class in nanostream.utils.data\_structures)}

\begin{fulllineitems}
\phantomsection\label{\detokenize{api:nanostream.utils.data_structures.MYSQL_INTEGER8}}\pysiglinewithargsret{\sphinxbfcode{\sphinxupquote{class }}\sphinxcode{\sphinxupquote{nanostream.utils.data\_structures.}}\sphinxbfcode{\sphinxupquote{MYSQL\_INTEGER8}}}{\emph{value}, \emph{original\_type=None}, \emph{name=None}}{}
Bases: {\hyperref[\detokenize{api:nanostream.utils.data_structures.MYSQL_INTEGER_BASE}]{\sphinxcrossref{\sphinxcode{\sphinxupquote{nanostream.utils.data\_structures.MYSQL\_INTEGER\_BASE}}}}}
\index{max\_length (nanostream.utils.data\_structures.MYSQL\_INTEGER8 attribute)}

\begin{fulllineitems}
\phantomsection\label{\detokenize{api:nanostream.utils.data_structures.MYSQL_INTEGER8.max_length}}\pysigline{\sphinxbfcode{\sphinxupquote{max\_length}}\sphinxbfcode{\sphinxupquote{ = 8}}}
\end{fulllineitems}


\end{fulllineitems}

\index{MYSQL\_INTEGER8192 (class in nanostream.utils.data\_structures)}

\begin{fulllineitems}
\phantomsection\label{\detokenize{api:nanostream.utils.data_structures.MYSQL_INTEGER8192}}\pysiglinewithargsret{\sphinxbfcode{\sphinxupquote{class }}\sphinxcode{\sphinxupquote{nanostream.utils.data\_structures.}}\sphinxbfcode{\sphinxupquote{MYSQL\_INTEGER8192}}}{\emph{value}, \emph{original\_type=None}, \emph{name=None}}{}
Bases: {\hyperref[\detokenize{api:nanostream.utils.data_structures.MYSQL_INTEGER_BASE}]{\sphinxcrossref{\sphinxcode{\sphinxupquote{nanostream.utils.data\_structures.MYSQL\_INTEGER\_BASE}}}}}
\index{max\_length (nanostream.utils.data\_structures.MYSQL\_INTEGER8192 attribute)}

\begin{fulllineitems}
\phantomsection\label{\detokenize{api:nanostream.utils.data_structures.MYSQL_INTEGER8192.max_length}}\pysigline{\sphinxbfcode{\sphinxupquote{max\_length}}\sphinxbfcode{\sphinxupquote{ = 8192}}}
\end{fulllineitems}


\end{fulllineitems}

\index{MYSQL\_INTEGER9 (class in nanostream.utils.data\_structures)}

\begin{fulllineitems}
\phantomsection\label{\detokenize{api:nanostream.utils.data_structures.MYSQL_INTEGER9}}\pysiglinewithargsret{\sphinxbfcode{\sphinxupquote{class }}\sphinxcode{\sphinxupquote{nanostream.utils.data\_structures.}}\sphinxbfcode{\sphinxupquote{MYSQL\_INTEGER9}}}{\emph{value}, \emph{original\_type=None}, \emph{name=None}}{}
Bases: {\hyperref[\detokenize{api:nanostream.utils.data_structures.MYSQL_INTEGER_BASE}]{\sphinxcrossref{\sphinxcode{\sphinxupquote{nanostream.utils.data\_structures.MYSQL\_INTEGER\_BASE}}}}}
\index{max\_length (nanostream.utils.data\_structures.MYSQL\_INTEGER9 attribute)}

\begin{fulllineitems}
\phantomsection\label{\detokenize{api:nanostream.utils.data_structures.MYSQL_INTEGER9.max_length}}\pysigline{\sphinxbfcode{\sphinxupquote{max\_length}}\sphinxbfcode{\sphinxupquote{ = 9}}}
\end{fulllineitems}


\end{fulllineitems}

\index{MYSQL\_INTEGER\_BASE (class in nanostream.utils.data\_structures)}

\begin{fulllineitems}
\phantomsection\label{\detokenize{api:nanostream.utils.data_structures.MYSQL_INTEGER_BASE}}\pysiglinewithargsret{\sphinxbfcode{\sphinxupquote{class }}\sphinxcode{\sphinxupquote{nanostream.utils.data\_structures.}}\sphinxbfcode{\sphinxupquote{MYSQL\_INTEGER\_BASE}}}{\emph{value}, \emph{original\_type=None}, \emph{name=None}}{}
Bases: {\hyperref[\detokenize{api:nanostream.utils.data_structures.DataType}]{\sphinxcrossref{\sphinxcode{\sphinxupquote{nanostream.utils.data\_structures.DataType}}}}}, {\hyperref[\detokenize{api:nanostream.utils.data_structures.MySQLTypeSystem}]{\sphinxcrossref{\sphinxcode{\sphinxupquote{nanostream.utils.data\_structures.MySQLTypeSystem}}}}}
\index{intermediate\_type (nanostream.utils.data\_structures.MYSQL\_INTEGER\_BASE attribute)}

\begin{fulllineitems}
\phantomsection\label{\detokenize{api:nanostream.utils.data_structures.MYSQL_INTEGER_BASE.intermediate_type}}\pysigline{\sphinxbfcode{\sphinxupquote{intermediate\_type}}}
alias of {\hyperref[\detokenize{api:nanostream.utils.data_structures.INTEGER}]{\sphinxcrossref{\sphinxcode{\sphinxupquote{INTEGER}}}}}

\end{fulllineitems}

\index{python\_cast\_function (nanostream.utils.data\_structures.MYSQL\_INTEGER\_BASE attribute)}

\begin{fulllineitems}
\phantomsection\label{\detokenize{api:nanostream.utils.data_structures.MYSQL_INTEGER_BASE.python_cast_function}}\pysigline{\sphinxbfcode{\sphinxupquote{python\_cast\_function}}}
alias of \sphinxcode{\sphinxupquote{builtins.int}}

\end{fulllineitems}


\end{fulllineitems}

\index{MYSQL\_VARCHAR (class in nanostream.utils.data\_structures)}

\begin{fulllineitems}
\phantomsection\label{\detokenize{api:nanostream.utils.data_structures.MYSQL_VARCHAR}}\pysigline{\sphinxbfcode{\sphinxupquote{class }}\sphinxcode{\sphinxupquote{nanostream.utils.data\_structures.}}\sphinxbfcode{\sphinxupquote{MYSQL\_VARCHAR}}}
Bases: \sphinxcode{\sphinxupquote{type}}

\end{fulllineitems}

\index{MYSQL\_VARCHAR0 (class in nanostream.utils.data\_structures)}

\begin{fulllineitems}
\phantomsection\label{\detokenize{api:nanostream.utils.data_structures.MYSQL_VARCHAR0}}\pysiglinewithargsret{\sphinxbfcode{\sphinxupquote{class }}\sphinxcode{\sphinxupquote{nanostream.utils.data\_structures.}}\sphinxbfcode{\sphinxupquote{MYSQL\_VARCHAR0}}}{\emph{value}, \emph{original\_type=None}, \emph{name=None}}{}
Bases: {\hyperref[\detokenize{api:nanostream.utils.data_structures.MYSQL_VARCHAR_BASE}]{\sphinxcrossref{\sphinxcode{\sphinxupquote{nanostream.utils.data\_structures.MYSQL\_VARCHAR\_BASE}}}}}
\index{max\_length (nanostream.utils.data\_structures.MYSQL\_VARCHAR0 attribute)}

\begin{fulllineitems}
\phantomsection\label{\detokenize{api:nanostream.utils.data_structures.MYSQL_VARCHAR0.max_length}}\pysigline{\sphinxbfcode{\sphinxupquote{max\_length}}\sphinxbfcode{\sphinxupquote{ = 0}}}
\end{fulllineitems}


\end{fulllineitems}

\index{MYSQL\_VARCHAR1 (class in nanostream.utils.data\_structures)}

\begin{fulllineitems}
\phantomsection\label{\detokenize{api:nanostream.utils.data_structures.MYSQL_VARCHAR1}}\pysiglinewithargsret{\sphinxbfcode{\sphinxupquote{class }}\sphinxcode{\sphinxupquote{nanostream.utils.data\_structures.}}\sphinxbfcode{\sphinxupquote{MYSQL\_VARCHAR1}}}{\emph{value}, \emph{original\_type=None}, \emph{name=None}}{}
Bases: {\hyperref[\detokenize{api:nanostream.utils.data_structures.MYSQL_VARCHAR_BASE}]{\sphinxcrossref{\sphinxcode{\sphinxupquote{nanostream.utils.data\_structures.MYSQL\_VARCHAR\_BASE}}}}}
\index{max\_length (nanostream.utils.data\_structures.MYSQL\_VARCHAR1 attribute)}

\begin{fulllineitems}
\phantomsection\label{\detokenize{api:nanostream.utils.data_structures.MYSQL_VARCHAR1.max_length}}\pysigline{\sphinxbfcode{\sphinxupquote{max\_length}}\sphinxbfcode{\sphinxupquote{ = 1}}}
\end{fulllineitems}


\end{fulllineitems}

\index{MYSQL\_VARCHAR10 (class in nanostream.utils.data\_structures)}

\begin{fulllineitems}
\phantomsection\label{\detokenize{api:nanostream.utils.data_structures.MYSQL_VARCHAR10}}\pysiglinewithargsret{\sphinxbfcode{\sphinxupquote{class }}\sphinxcode{\sphinxupquote{nanostream.utils.data\_structures.}}\sphinxbfcode{\sphinxupquote{MYSQL\_VARCHAR10}}}{\emph{value}, \emph{original\_type=None}, \emph{name=None}}{}
Bases: {\hyperref[\detokenize{api:nanostream.utils.data_structures.MYSQL_VARCHAR_BASE}]{\sphinxcrossref{\sphinxcode{\sphinxupquote{nanostream.utils.data\_structures.MYSQL\_VARCHAR\_BASE}}}}}
\index{max\_length (nanostream.utils.data\_structures.MYSQL\_VARCHAR10 attribute)}

\begin{fulllineitems}
\phantomsection\label{\detokenize{api:nanostream.utils.data_structures.MYSQL_VARCHAR10.max_length}}\pysigline{\sphinxbfcode{\sphinxupquote{max\_length}}\sphinxbfcode{\sphinxupquote{ = 10}}}
\end{fulllineitems}


\end{fulllineitems}

\index{MYSQL\_VARCHAR1024 (class in nanostream.utils.data\_structures)}

\begin{fulllineitems}
\phantomsection\label{\detokenize{api:nanostream.utils.data_structures.MYSQL_VARCHAR1024}}\pysiglinewithargsret{\sphinxbfcode{\sphinxupquote{class }}\sphinxcode{\sphinxupquote{nanostream.utils.data\_structures.}}\sphinxbfcode{\sphinxupquote{MYSQL\_VARCHAR1024}}}{\emph{value}, \emph{original\_type=None}, \emph{name=None}}{}
Bases: {\hyperref[\detokenize{api:nanostream.utils.data_structures.MYSQL_VARCHAR_BASE}]{\sphinxcrossref{\sphinxcode{\sphinxupquote{nanostream.utils.data\_structures.MYSQL\_VARCHAR\_BASE}}}}}
\index{max\_length (nanostream.utils.data\_structures.MYSQL\_VARCHAR1024 attribute)}

\begin{fulllineitems}
\phantomsection\label{\detokenize{api:nanostream.utils.data_structures.MYSQL_VARCHAR1024.max_length}}\pysigline{\sphinxbfcode{\sphinxupquote{max\_length}}\sphinxbfcode{\sphinxupquote{ = 1024}}}
\end{fulllineitems}


\end{fulllineitems}

\index{MYSQL\_VARCHAR11 (class in nanostream.utils.data\_structures)}

\begin{fulllineitems}
\phantomsection\label{\detokenize{api:nanostream.utils.data_structures.MYSQL_VARCHAR11}}\pysiglinewithargsret{\sphinxbfcode{\sphinxupquote{class }}\sphinxcode{\sphinxupquote{nanostream.utils.data\_structures.}}\sphinxbfcode{\sphinxupquote{MYSQL\_VARCHAR11}}}{\emph{value}, \emph{original\_type=None}, \emph{name=None}}{}
Bases: {\hyperref[\detokenize{api:nanostream.utils.data_structures.MYSQL_VARCHAR_BASE}]{\sphinxcrossref{\sphinxcode{\sphinxupquote{nanostream.utils.data\_structures.MYSQL\_VARCHAR\_BASE}}}}}
\index{max\_length (nanostream.utils.data\_structures.MYSQL\_VARCHAR11 attribute)}

\begin{fulllineitems}
\phantomsection\label{\detokenize{api:nanostream.utils.data_structures.MYSQL_VARCHAR11.max_length}}\pysigline{\sphinxbfcode{\sphinxupquote{max\_length}}\sphinxbfcode{\sphinxupquote{ = 11}}}
\end{fulllineitems}


\end{fulllineitems}

\index{MYSQL\_VARCHAR12 (class in nanostream.utils.data\_structures)}

\begin{fulllineitems}
\phantomsection\label{\detokenize{api:nanostream.utils.data_structures.MYSQL_VARCHAR12}}\pysiglinewithargsret{\sphinxbfcode{\sphinxupquote{class }}\sphinxcode{\sphinxupquote{nanostream.utils.data\_structures.}}\sphinxbfcode{\sphinxupquote{MYSQL\_VARCHAR12}}}{\emph{value}, \emph{original\_type=None}, \emph{name=None}}{}
Bases: {\hyperref[\detokenize{api:nanostream.utils.data_structures.MYSQL_VARCHAR_BASE}]{\sphinxcrossref{\sphinxcode{\sphinxupquote{nanostream.utils.data\_structures.MYSQL\_VARCHAR\_BASE}}}}}
\index{max\_length (nanostream.utils.data\_structures.MYSQL\_VARCHAR12 attribute)}

\begin{fulllineitems}
\phantomsection\label{\detokenize{api:nanostream.utils.data_structures.MYSQL_VARCHAR12.max_length}}\pysigline{\sphinxbfcode{\sphinxupquote{max\_length}}\sphinxbfcode{\sphinxupquote{ = 12}}}
\end{fulllineitems}


\end{fulllineitems}

\index{MYSQL\_VARCHAR128 (class in nanostream.utils.data\_structures)}

\begin{fulllineitems}
\phantomsection\label{\detokenize{api:nanostream.utils.data_structures.MYSQL_VARCHAR128}}\pysiglinewithargsret{\sphinxbfcode{\sphinxupquote{class }}\sphinxcode{\sphinxupquote{nanostream.utils.data\_structures.}}\sphinxbfcode{\sphinxupquote{MYSQL\_VARCHAR128}}}{\emph{value}, \emph{original\_type=None}, \emph{name=None}}{}
Bases: {\hyperref[\detokenize{api:nanostream.utils.data_structures.MYSQL_VARCHAR_BASE}]{\sphinxcrossref{\sphinxcode{\sphinxupquote{nanostream.utils.data\_structures.MYSQL\_VARCHAR\_BASE}}}}}
\index{max\_length (nanostream.utils.data\_structures.MYSQL\_VARCHAR128 attribute)}

\begin{fulllineitems}
\phantomsection\label{\detokenize{api:nanostream.utils.data_structures.MYSQL_VARCHAR128.max_length}}\pysigline{\sphinxbfcode{\sphinxupquote{max\_length}}\sphinxbfcode{\sphinxupquote{ = 128}}}
\end{fulllineitems}


\end{fulllineitems}

\index{MYSQL\_VARCHAR13 (class in nanostream.utils.data\_structures)}

\begin{fulllineitems}
\phantomsection\label{\detokenize{api:nanostream.utils.data_structures.MYSQL_VARCHAR13}}\pysiglinewithargsret{\sphinxbfcode{\sphinxupquote{class }}\sphinxcode{\sphinxupquote{nanostream.utils.data\_structures.}}\sphinxbfcode{\sphinxupquote{MYSQL\_VARCHAR13}}}{\emph{value}, \emph{original\_type=None}, \emph{name=None}}{}
Bases: {\hyperref[\detokenize{api:nanostream.utils.data_structures.MYSQL_VARCHAR_BASE}]{\sphinxcrossref{\sphinxcode{\sphinxupquote{nanostream.utils.data\_structures.MYSQL\_VARCHAR\_BASE}}}}}
\index{max\_length (nanostream.utils.data\_structures.MYSQL\_VARCHAR13 attribute)}

\begin{fulllineitems}
\phantomsection\label{\detokenize{api:nanostream.utils.data_structures.MYSQL_VARCHAR13.max_length}}\pysigline{\sphinxbfcode{\sphinxupquote{max\_length}}\sphinxbfcode{\sphinxupquote{ = 13}}}
\end{fulllineitems}


\end{fulllineitems}

\index{MYSQL\_VARCHAR14 (class in nanostream.utils.data\_structures)}

\begin{fulllineitems}
\phantomsection\label{\detokenize{api:nanostream.utils.data_structures.MYSQL_VARCHAR14}}\pysiglinewithargsret{\sphinxbfcode{\sphinxupquote{class }}\sphinxcode{\sphinxupquote{nanostream.utils.data\_structures.}}\sphinxbfcode{\sphinxupquote{MYSQL\_VARCHAR14}}}{\emph{value}, \emph{original\_type=None}, \emph{name=None}}{}
Bases: {\hyperref[\detokenize{api:nanostream.utils.data_structures.MYSQL_VARCHAR_BASE}]{\sphinxcrossref{\sphinxcode{\sphinxupquote{nanostream.utils.data\_structures.MYSQL\_VARCHAR\_BASE}}}}}
\index{max\_length (nanostream.utils.data\_structures.MYSQL\_VARCHAR14 attribute)}

\begin{fulllineitems}
\phantomsection\label{\detokenize{api:nanostream.utils.data_structures.MYSQL_VARCHAR14.max_length}}\pysigline{\sphinxbfcode{\sphinxupquote{max\_length}}\sphinxbfcode{\sphinxupquote{ = 14}}}
\end{fulllineitems}


\end{fulllineitems}

\index{MYSQL\_VARCHAR15 (class in nanostream.utils.data\_structures)}

\begin{fulllineitems}
\phantomsection\label{\detokenize{api:nanostream.utils.data_structures.MYSQL_VARCHAR15}}\pysiglinewithargsret{\sphinxbfcode{\sphinxupquote{class }}\sphinxcode{\sphinxupquote{nanostream.utils.data\_structures.}}\sphinxbfcode{\sphinxupquote{MYSQL\_VARCHAR15}}}{\emph{value}, \emph{original\_type=None}, \emph{name=None}}{}
Bases: {\hyperref[\detokenize{api:nanostream.utils.data_structures.MYSQL_VARCHAR_BASE}]{\sphinxcrossref{\sphinxcode{\sphinxupquote{nanostream.utils.data\_structures.MYSQL\_VARCHAR\_BASE}}}}}
\index{max\_length (nanostream.utils.data\_structures.MYSQL\_VARCHAR15 attribute)}

\begin{fulllineitems}
\phantomsection\label{\detokenize{api:nanostream.utils.data_structures.MYSQL_VARCHAR15.max_length}}\pysigline{\sphinxbfcode{\sphinxupquote{max\_length}}\sphinxbfcode{\sphinxupquote{ = 15}}}
\end{fulllineitems}


\end{fulllineitems}

\index{MYSQL\_VARCHAR16 (class in nanostream.utils.data\_structures)}

\begin{fulllineitems}
\phantomsection\label{\detokenize{api:nanostream.utils.data_structures.MYSQL_VARCHAR16}}\pysiglinewithargsret{\sphinxbfcode{\sphinxupquote{class }}\sphinxcode{\sphinxupquote{nanostream.utils.data\_structures.}}\sphinxbfcode{\sphinxupquote{MYSQL\_VARCHAR16}}}{\emph{value}, \emph{original\_type=None}, \emph{name=None}}{}
Bases: {\hyperref[\detokenize{api:nanostream.utils.data_structures.MYSQL_VARCHAR_BASE}]{\sphinxcrossref{\sphinxcode{\sphinxupquote{nanostream.utils.data\_structures.MYSQL\_VARCHAR\_BASE}}}}}
\index{max\_length (nanostream.utils.data\_structures.MYSQL\_VARCHAR16 attribute)}

\begin{fulllineitems}
\phantomsection\label{\detokenize{api:nanostream.utils.data_structures.MYSQL_VARCHAR16.max_length}}\pysigline{\sphinxbfcode{\sphinxupquote{max\_length}}\sphinxbfcode{\sphinxupquote{ = 16}}}
\end{fulllineitems}


\end{fulllineitems}

\index{MYSQL\_VARCHAR16384 (class in nanostream.utils.data\_structures)}

\begin{fulllineitems}
\phantomsection\label{\detokenize{api:nanostream.utils.data_structures.MYSQL_VARCHAR16384}}\pysiglinewithargsret{\sphinxbfcode{\sphinxupquote{class }}\sphinxcode{\sphinxupquote{nanostream.utils.data\_structures.}}\sphinxbfcode{\sphinxupquote{MYSQL\_VARCHAR16384}}}{\emph{value}, \emph{original\_type=None}, \emph{name=None}}{}
Bases: {\hyperref[\detokenize{api:nanostream.utils.data_structures.MYSQL_VARCHAR_BASE}]{\sphinxcrossref{\sphinxcode{\sphinxupquote{nanostream.utils.data\_structures.MYSQL\_VARCHAR\_BASE}}}}}
\index{max\_length (nanostream.utils.data\_structures.MYSQL\_VARCHAR16384 attribute)}

\begin{fulllineitems}
\phantomsection\label{\detokenize{api:nanostream.utils.data_structures.MYSQL_VARCHAR16384.max_length}}\pysigline{\sphinxbfcode{\sphinxupquote{max\_length}}\sphinxbfcode{\sphinxupquote{ = 16384}}}
\end{fulllineitems}


\end{fulllineitems}

\index{MYSQL\_VARCHAR17 (class in nanostream.utils.data\_structures)}

\begin{fulllineitems}
\phantomsection\label{\detokenize{api:nanostream.utils.data_structures.MYSQL_VARCHAR17}}\pysiglinewithargsret{\sphinxbfcode{\sphinxupquote{class }}\sphinxcode{\sphinxupquote{nanostream.utils.data\_structures.}}\sphinxbfcode{\sphinxupquote{MYSQL\_VARCHAR17}}}{\emph{value}, \emph{original\_type=None}, \emph{name=None}}{}
Bases: {\hyperref[\detokenize{api:nanostream.utils.data_structures.MYSQL_VARCHAR_BASE}]{\sphinxcrossref{\sphinxcode{\sphinxupquote{nanostream.utils.data\_structures.MYSQL\_VARCHAR\_BASE}}}}}
\index{max\_length (nanostream.utils.data\_structures.MYSQL\_VARCHAR17 attribute)}

\begin{fulllineitems}
\phantomsection\label{\detokenize{api:nanostream.utils.data_structures.MYSQL_VARCHAR17.max_length}}\pysigline{\sphinxbfcode{\sphinxupquote{max\_length}}\sphinxbfcode{\sphinxupquote{ = 17}}}
\end{fulllineitems}


\end{fulllineitems}

\index{MYSQL\_VARCHAR18 (class in nanostream.utils.data\_structures)}

\begin{fulllineitems}
\phantomsection\label{\detokenize{api:nanostream.utils.data_structures.MYSQL_VARCHAR18}}\pysiglinewithargsret{\sphinxbfcode{\sphinxupquote{class }}\sphinxcode{\sphinxupquote{nanostream.utils.data\_structures.}}\sphinxbfcode{\sphinxupquote{MYSQL\_VARCHAR18}}}{\emph{value}, \emph{original\_type=None}, \emph{name=None}}{}
Bases: {\hyperref[\detokenize{api:nanostream.utils.data_structures.MYSQL_VARCHAR_BASE}]{\sphinxcrossref{\sphinxcode{\sphinxupquote{nanostream.utils.data\_structures.MYSQL\_VARCHAR\_BASE}}}}}
\index{max\_length (nanostream.utils.data\_structures.MYSQL\_VARCHAR18 attribute)}

\begin{fulllineitems}
\phantomsection\label{\detokenize{api:nanostream.utils.data_structures.MYSQL_VARCHAR18.max_length}}\pysigline{\sphinxbfcode{\sphinxupquote{max\_length}}\sphinxbfcode{\sphinxupquote{ = 18}}}
\end{fulllineitems}


\end{fulllineitems}

\index{MYSQL\_VARCHAR19 (class in nanostream.utils.data\_structures)}

\begin{fulllineitems}
\phantomsection\label{\detokenize{api:nanostream.utils.data_structures.MYSQL_VARCHAR19}}\pysiglinewithargsret{\sphinxbfcode{\sphinxupquote{class }}\sphinxcode{\sphinxupquote{nanostream.utils.data\_structures.}}\sphinxbfcode{\sphinxupquote{MYSQL\_VARCHAR19}}}{\emph{value}, \emph{original\_type=None}, \emph{name=None}}{}
Bases: {\hyperref[\detokenize{api:nanostream.utils.data_structures.MYSQL_VARCHAR_BASE}]{\sphinxcrossref{\sphinxcode{\sphinxupquote{nanostream.utils.data\_structures.MYSQL\_VARCHAR\_BASE}}}}}
\index{max\_length (nanostream.utils.data\_structures.MYSQL\_VARCHAR19 attribute)}

\begin{fulllineitems}
\phantomsection\label{\detokenize{api:nanostream.utils.data_structures.MYSQL_VARCHAR19.max_length}}\pysigline{\sphinxbfcode{\sphinxupquote{max\_length}}\sphinxbfcode{\sphinxupquote{ = 19}}}
\end{fulllineitems}


\end{fulllineitems}

\index{MYSQL\_VARCHAR2 (class in nanostream.utils.data\_structures)}

\begin{fulllineitems}
\phantomsection\label{\detokenize{api:nanostream.utils.data_structures.MYSQL_VARCHAR2}}\pysiglinewithargsret{\sphinxbfcode{\sphinxupquote{class }}\sphinxcode{\sphinxupquote{nanostream.utils.data\_structures.}}\sphinxbfcode{\sphinxupquote{MYSQL\_VARCHAR2}}}{\emph{value}, \emph{original\_type=None}, \emph{name=None}}{}
Bases: {\hyperref[\detokenize{api:nanostream.utils.data_structures.MYSQL_VARCHAR_BASE}]{\sphinxcrossref{\sphinxcode{\sphinxupquote{nanostream.utils.data\_structures.MYSQL\_VARCHAR\_BASE}}}}}
\index{max\_length (nanostream.utils.data\_structures.MYSQL\_VARCHAR2 attribute)}

\begin{fulllineitems}
\phantomsection\label{\detokenize{api:nanostream.utils.data_structures.MYSQL_VARCHAR2.max_length}}\pysigline{\sphinxbfcode{\sphinxupquote{max\_length}}\sphinxbfcode{\sphinxupquote{ = 2}}}
\end{fulllineitems}


\end{fulllineitems}

\index{MYSQL\_VARCHAR20 (class in nanostream.utils.data\_structures)}

\begin{fulllineitems}
\phantomsection\label{\detokenize{api:nanostream.utils.data_structures.MYSQL_VARCHAR20}}\pysiglinewithargsret{\sphinxbfcode{\sphinxupquote{class }}\sphinxcode{\sphinxupquote{nanostream.utils.data\_structures.}}\sphinxbfcode{\sphinxupquote{MYSQL\_VARCHAR20}}}{\emph{value}, \emph{original\_type=None}, \emph{name=None}}{}
Bases: {\hyperref[\detokenize{api:nanostream.utils.data_structures.MYSQL_VARCHAR_BASE}]{\sphinxcrossref{\sphinxcode{\sphinxupquote{nanostream.utils.data\_structures.MYSQL\_VARCHAR\_BASE}}}}}
\index{max\_length (nanostream.utils.data\_structures.MYSQL\_VARCHAR20 attribute)}

\begin{fulllineitems}
\phantomsection\label{\detokenize{api:nanostream.utils.data_structures.MYSQL_VARCHAR20.max_length}}\pysigline{\sphinxbfcode{\sphinxupquote{max\_length}}\sphinxbfcode{\sphinxupquote{ = 20}}}
\end{fulllineitems}


\end{fulllineitems}

\index{MYSQL\_VARCHAR2048 (class in nanostream.utils.data\_structures)}

\begin{fulllineitems}
\phantomsection\label{\detokenize{api:nanostream.utils.data_structures.MYSQL_VARCHAR2048}}\pysiglinewithargsret{\sphinxbfcode{\sphinxupquote{class }}\sphinxcode{\sphinxupquote{nanostream.utils.data\_structures.}}\sphinxbfcode{\sphinxupquote{MYSQL\_VARCHAR2048}}}{\emph{value}, \emph{original\_type=None}, \emph{name=None}}{}
Bases: {\hyperref[\detokenize{api:nanostream.utils.data_structures.MYSQL_VARCHAR_BASE}]{\sphinxcrossref{\sphinxcode{\sphinxupquote{nanostream.utils.data\_structures.MYSQL\_VARCHAR\_BASE}}}}}
\index{max\_length (nanostream.utils.data\_structures.MYSQL\_VARCHAR2048 attribute)}

\begin{fulllineitems}
\phantomsection\label{\detokenize{api:nanostream.utils.data_structures.MYSQL_VARCHAR2048.max_length}}\pysigline{\sphinxbfcode{\sphinxupquote{max\_length}}\sphinxbfcode{\sphinxupquote{ = 2048}}}
\end{fulllineitems}


\end{fulllineitems}

\index{MYSQL\_VARCHAR21 (class in nanostream.utils.data\_structures)}

\begin{fulllineitems}
\phantomsection\label{\detokenize{api:nanostream.utils.data_structures.MYSQL_VARCHAR21}}\pysiglinewithargsret{\sphinxbfcode{\sphinxupquote{class }}\sphinxcode{\sphinxupquote{nanostream.utils.data\_structures.}}\sphinxbfcode{\sphinxupquote{MYSQL\_VARCHAR21}}}{\emph{value}, \emph{original\_type=None}, \emph{name=None}}{}
Bases: {\hyperref[\detokenize{api:nanostream.utils.data_structures.MYSQL_VARCHAR_BASE}]{\sphinxcrossref{\sphinxcode{\sphinxupquote{nanostream.utils.data\_structures.MYSQL\_VARCHAR\_BASE}}}}}
\index{max\_length (nanostream.utils.data\_structures.MYSQL\_VARCHAR21 attribute)}

\begin{fulllineitems}
\phantomsection\label{\detokenize{api:nanostream.utils.data_structures.MYSQL_VARCHAR21.max_length}}\pysigline{\sphinxbfcode{\sphinxupquote{max\_length}}\sphinxbfcode{\sphinxupquote{ = 21}}}
\end{fulllineitems}


\end{fulllineitems}

\index{MYSQL\_VARCHAR22 (class in nanostream.utils.data\_structures)}

\begin{fulllineitems}
\phantomsection\label{\detokenize{api:nanostream.utils.data_structures.MYSQL_VARCHAR22}}\pysiglinewithargsret{\sphinxbfcode{\sphinxupquote{class }}\sphinxcode{\sphinxupquote{nanostream.utils.data\_structures.}}\sphinxbfcode{\sphinxupquote{MYSQL\_VARCHAR22}}}{\emph{value}, \emph{original\_type=None}, \emph{name=None}}{}
Bases: {\hyperref[\detokenize{api:nanostream.utils.data_structures.MYSQL_VARCHAR_BASE}]{\sphinxcrossref{\sphinxcode{\sphinxupquote{nanostream.utils.data\_structures.MYSQL\_VARCHAR\_BASE}}}}}
\index{max\_length (nanostream.utils.data\_structures.MYSQL\_VARCHAR22 attribute)}

\begin{fulllineitems}
\phantomsection\label{\detokenize{api:nanostream.utils.data_structures.MYSQL_VARCHAR22.max_length}}\pysigline{\sphinxbfcode{\sphinxupquote{max\_length}}\sphinxbfcode{\sphinxupquote{ = 22}}}
\end{fulllineitems}


\end{fulllineitems}

\index{MYSQL\_VARCHAR23 (class in nanostream.utils.data\_structures)}

\begin{fulllineitems}
\phantomsection\label{\detokenize{api:nanostream.utils.data_structures.MYSQL_VARCHAR23}}\pysiglinewithargsret{\sphinxbfcode{\sphinxupquote{class }}\sphinxcode{\sphinxupquote{nanostream.utils.data\_structures.}}\sphinxbfcode{\sphinxupquote{MYSQL\_VARCHAR23}}}{\emph{value}, \emph{original\_type=None}, \emph{name=None}}{}
Bases: {\hyperref[\detokenize{api:nanostream.utils.data_structures.MYSQL_VARCHAR_BASE}]{\sphinxcrossref{\sphinxcode{\sphinxupquote{nanostream.utils.data\_structures.MYSQL\_VARCHAR\_BASE}}}}}
\index{max\_length (nanostream.utils.data\_structures.MYSQL\_VARCHAR23 attribute)}

\begin{fulllineitems}
\phantomsection\label{\detokenize{api:nanostream.utils.data_structures.MYSQL_VARCHAR23.max_length}}\pysigline{\sphinxbfcode{\sphinxupquote{max\_length}}\sphinxbfcode{\sphinxupquote{ = 23}}}
\end{fulllineitems}


\end{fulllineitems}

\index{MYSQL\_VARCHAR24 (class in nanostream.utils.data\_structures)}

\begin{fulllineitems}
\phantomsection\label{\detokenize{api:nanostream.utils.data_structures.MYSQL_VARCHAR24}}\pysiglinewithargsret{\sphinxbfcode{\sphinxupquote{class }}\sphinxcode{\sphinxupquote{nanostream.utils.data\_structures.}}\sphinxbfcode{\sphinxupquote{MYSQL\_VARCHAR24}}}{\emph{value}, \emph{original\_type=None}, \emph{name=None}}{}
Bases: {\hyperref[\detokenize{api:nanostream.utils.data_structures.MYSQL_VARCHAR_BASE}]{\sphinxcrossref{\sphinxcode{\sphinxupquote{nanostream.utils.data\_structures.MYSQL\_VARCHAR\_BASE}}}}}
\index{max\_length (nanostream.utils.data\_structures.MYSQL\_VARCHAR24 attribute)}

\begin{fulllineitems}
\phantomsection\label{\detokenize{api:nanostream.utils.data_structures.MYSQL_VARCHAR24.max_length}}\pysigline{\sphinxbfcode{\sphinxupquote{max\_length}}\sphinxbfcode{\sphinxupquote{ = 24}}}
\end{fulllineitems}


\end{fulllineitems}

\index{MYSQL\_VARCHAR25 (class in nanostream.utils.data\_structures)}

\begin{fulllineitems}
\phantomsection\label{\detokenize{api:nanostream.utils.data_structures.MYSQL_VARCHAR25}}\pysiglinewithargsret{\sphinxbfcode{\sphinxupquote{class }}\sphinxcode{\sphinxupquote{nanostream.utils.data\_structures.}}\sphinxbfcode{\sphinxupquote{MYSQL\_VARCHAR25}}}{\emph{value}, \emph{original\_type=None}, \emph{name=None}}{}
Bases: {\hyperref[\detokenize{api:nanostream.utils.data_structures.MYSQL_VARCHAR_BASE}]{\sphinxcrossref{\sphinxcode{\sphinxupquote{nanostream.utils.data\_structures.MYSQL\_VARCHAR\_BASE}}}}}
\index{max\_length (nanostream.utils.data\_structures.MYSQL\_VARCHAR25 attribute)}

\begin{fulllineitems}
\phantomsection\label{\detokenize{api:nanostream.utils.data_structures.MYSQL_VARCHAR25.max_length}}\pysigline{\sphinxbfcode{\sphinxupquote{max\_length}}\sphinxbfcode{\sphinxupquote{ = 25}}}
\end{fulllineitems}


\end{fulllineitems}

\index{MYSQL\_VARCHAR256 (class in nanostream.utils.data\_structures)}

\begin{fulllineitems}
\phantomsection\label{\detokenize{api:nanostream.utils.data_structures.MYSQL_VARCHAR256}}\pysiglinewithargsret{\sphinxbfcode{\sphinxupquote{class }}\sphinxcode{\sphinxupquote{nanostream.utils.data\_structures.}}\sphinxbfcode{\sphinxupquote{MYSQL\_VARCHAR256}}}{\emph{value}, \emph{original\_type=None}, \emph{name=None}}{}
Bases: {\hyperref[\detokenize{api:nanostream.utils.data_structures.MYSQL_VARCHAR_BASE}]{\sphinxcrossref{\sphinxcode{\sphinxupquote{nanostream.utils.data\_structures.MYSQL\_VARCHAR\_BASE}}}}}
\index{max\_length (nanostream.utils.data\_structures.MYSQL\_VARCHAR256 attribute)}

\begin{fulllineitems}
\phantomsection\label{\detokenize{api:nanostream.utils.data_structures.MYSQL_VARCHAR256.max_length}}\pysigline{\sphinxbfcode{\sphinxupquote{max\_length}}\sphinxbfcode{\sphinxupquote{ = 256}}}
\end{fulllineitems}


\end{fulllineitems}

\index{MYSQL\_VARCHAR26 (class in nanostream.utils.data\_structures)}

\begin{fulllineitems}
\phantomsection\label{\detokenize{api:nanostream.utils.data_structures.MYSQL_VARCHAR26}}\pysiglinewithargsret{\sphinxbfcode{\sphinxupquote{class }}\sphinxcode{\sphinxupquote{nanostream.utils.data\_structures.}}\sphinxbfcode{\sphinxupquote{MYSQL\_VARCHAR26}}}{\emph{value}, \emph{original\_type=None}, \emph{name=None}}{}
Bases: {\hyperref[\detokenize{api:nanostream.utils.data_structures.MYSQL_VARCHAR_BASE}]{\sphinxcrossref{\sphinxcode{\sphinxupquote{nanostream.utils.data\_structures.MYSQL\_VARCHAR\_BASE}}}}}
\index{max\_length (nanostream.utils.data\_structures.MYSQL\_VARCHAR26 attribute)}

\begin{fulllineitems}
\phantomsection\label{\detokenize{api:nanostream.utils.data_structures.MYSQL_VARCHAR26.max_length}}\pysigline{\sphinxbfcode{\sphinxupquote{max\_length}}\sphinxbfcode{\sphinxupquote{ = 26}}}
\end{fulllineitems}


\end{fulllineitems}

\index{MYSQL\_VARCHAR27 (class in nanostream.utils.data\_structures)}

\begin{fulllineitems}
\phantomsection\label{\detokenize{api:nanostream.utils.data_structures.MYSQL_VARCHAR27}}\pysiglinewithargsret{\sphinxbfcode{\sphinxupquote{class }}\sphinxcode{\sphinxupquote{nanostream.utils.data\_structures.}}\sphinxbfcode{\sphinxupquote{MYSQL\_VARCHAR27}}}{\emph{value}, \emph{original\_type=None}, \emph{name=None}}{}
Bases: {\hyperref[\detokenize{api:nanostream.utils.data_structures.MYSQL_VARCHAR_BASE}]{\sphinxcrossref{\sphinxcode{\sphinxupquote{nanostream.utils.data\_structures.MYSQL\_VARCHAR\_BASE}}}}}
\index{max\_length (nanostream.utils.data\_structures.MYSQL\_VARCHAR27 attribute)}

\begin{fulllineitems}
\phantomsection\label{\detokenize{api:nanostream.utils.data_structures.MYSQL_VARCHAR27.max_length}}\pysigline{\sphinxbfcode{\sphinxupquote{max\_length}}\sphinxbfcode{\sphinxupquote{ = 27}}}
\end{fulllineitems}


\end{fulllineitems}

\index{MYSQL\_VARCHAR28 (class in nanostream.utils.data\_structures)}

\begin{fulllineitems}
\phantomsection\label{\detokenize{api:nanostream.utils.data_structures.MYSQL_VARCHAR28}}\pysiglinewithargsret{\sphinxbfcode{\sphinxupquote{class }}\sphinxcode{\sphinxupquote{nanostream.utils.data\_structures.}}\sphinxbfcode{\sphinxupquote{MYSQL\_VARCHAR28}}}{\emph{value}, \emph{original\_type=None}, \emph{name=None}}{}
Bases: {\hyperref[\detokenize{api:nanostream.utils.data_structures.MYSQL_VARCHAR_BASE}]{\sphinxcrossref{\sphinxcode{\sphinxupquote{nanostream.utils.data\_structures.MYSQL\_VARCHAR\_BASE}}}}}
\index{max\_length (nanostream.utils.data\_structures.MYSQL\_VARCHAR28 attribute)}

\begin{fulllineitems}
\phantomsection\label{\detokenize{api:nanostream.utils.data_structures.MYSQL_VARCHAR28.max_length}}\pysigline{\sphinxbfcode{\sphinxupquote{max\_length}}\sphinxbfcode{\sphinxupquote{ = 28}}}
\end{fulllineitems}


\end{fulllineitems}

\index{MYSQL\_VARCHAR29 (class in nanostream.utils.data\_structures)}

\begin{fulllineitems}
\phantomsection\label{\detokenize{api:nanostream.utils.data_structures.MYSQL_VARCHAR29}}\pysiglinewithargsret{\sphinxbfcode{\sphinxupquote{class }}\sphinxcode{\sphinxupquote{nanostream.utils.data\_structures.}}\sphinxbfcode{\sphinxupquote{MYSQL\_VARCHAR29}}}{\emph{value}, \emph{original\_type=None}, \emph{name=None}}{}
Bases: {\hyperref[\detokenize{api:nanostream.utils.data_structures.MYSQL_VARCHAR_BASE}]{\sphinxcrossref{\sphinxcode{\sphinxupquote{nanostream.utils.data\_structures.MYSQL\_VARCHAR\_BASE}}}}}
\index{max\_length (nanostream.utils.data\_structures.MYSQL\_VARCHAR29 attribute)}

\begin{fulllineitems}
\phantomsection\label{\detokenize{api:nanostream.utils.data_structures.MYSQL_VARCHAR29.max_length}}\pysigline{\sphinxbfcode{\sphinxupquote{max\_length}}\sphinxbfcode{\sphinxupquote{ = 29}}}
\end{fulllineitems}


\end{fulllineitems}

\index{MYSQL\_VARCHAR3 (class in nanostream.utils.data\_structures)}

\begin{fulllineitems}
\phantomsection\label{\detokenize{api:nanostream.utils.data_structures.MYSQL_VARCHAR3}}\pysiglinewithargsret{\sphinxbfcode{\sphinxupquote{class }}\sphinxcode{\sphinxupquote{nanostream.utils.data\_structures.}}\sphinxbfcode{\sphinxupquote{MYSQL\_VARCHAR3}}}{\emph{value}, \emph{original\_type=None}, \emph{name=None}}{}
Bases: {\hyperref[\detokenize{api:nanostream.utils.data_structures.MYSQL_VARCHAR_BASE}]{\sphinxcrossref{\sphinxcode{\sphinxupquote{nanostream.utils.data\_structures.MYSQL\_VARCHAR\_BASE}}}}}
\index{max\_length (nanostream.utils.data\_structures.MYSQL\_VARCHAR3 attribute)}

\begin{fulllineitems}
\phantomsection\label{\detokenize{api:nanostream.utils.data_structures.MYSQL_VARCHAR3.max_length}}\pysigline{\sphinxbfcode{\sphinxupquote{max\_length}}\sphinxbfcode{\sphinxupquote{ = 3}}}
\end{fulllineitems}


\end{fulllineitems}

\index{MYSQL\_VARCHAR30 (class in nanostream.utils.data\_structures)}

\begin{fulllineitems}
\phantomsection\label{\detokenize{api:nanostream.utils.data_structures.MYSQL_VARCHAR30}}\pysiglinewithargsret{\sphinxbfcode{\sphinxupquote{class }}\sphinxcode{\sphinxupquote{nanostream.utils.data\_structures.}}\sphinxbfcode{\sphinxupquote{MYSQL\_VARCHAR30}}}{\emph{value}, \emph{original\_type=None}, \emph{name=None}}{}
Bases: {\hyperref[\detokenize{api:nanostream.utils.data_structures.MYSQL_VARCHAR_BASE}]{\sphinxcrossref{\sphinxcode{\sphinxupquote{nanostream.utils.data\_structures.MYSQL\_VARCHAR\_BASE}}}}}
\index{max\_length (nanostream.utils.data\_structures.MYSQL\_VARCHAR30 attribute)}

\begin{fulllineitems}
\phantomsection\label{\detokenize{api:nanostream.utils.data_structures.MYSQL_VARCHAR30.max_length}}\pysigline{\sphinxbfcode{\sphinxupquote{max\_length}}\sphinxbfcode{\sphinxupquote{ = 30}}}
\end{fulllineitems}


\end{fulllineitems}

\index{MYSQL\_VARCHAR31 (class in nanostream.utils.data\_structures)}

\begin{fulllineitems}
\phantomsection\label{\detokenize{api:nanostream.utils.data_structures.MYSQL_VARCHAR31}}\pysiglinewithargsret{\sphinxbfcode{\sphinxupquote{class }}\sphinxcode{\sphinxupquote{nanostream.utils.data\_structures.}}\sphinxbfcode{\sphinxupquote{MYSQL\_VARCHAR31}}}{\emph{value}, \emph{original\_type=None}, \emph{name=None}}{}
Bases: {\hyperref[\detokenize{api:nanostream.utils.data_structures.MYSQL_VARCHAR_BASE}]{\sphinxcrossref{\sphinxcode{\sphinxupquote{nanostream.utils.data\_structures.MYSQL\_VARCHAR\_BASE}}}}}
\index{max\_length (nanostream.utils.data\_structures.MYSQL\_VARCHAR31 attribute)}

\begin{fulllineitems}
\phantomsection\label{\detokenize{api:nanostream.utils.data_structures.MYSQL_VARCHAR31.max_length}}\pysigline{\sphinxbfcode{\sphinxupquote{max\_length}}\sphinxbfcode{\sphinxupquote{ = 31}}}
\end{fulllineitems}


\end{fulllineitems}

\index{MYSQL\_VARCHAR32 (class in nanostream.utils.data\_structures)}

\begin{fulllineitems}
\phantomsection\label{\detokenize{api:nanostream.utils.data_structures.MYSQL_VARCHAR32}}\pysiglinewithargsret{\sphinxbfcode{\sphinxupquote{class }}\sphinxcode{\sphinxupquote{nanostream.utils.data\_structures.}}\sphinxbfcode{\sphinxupquote{MYSQL\_VARCHAR32}}}{\emph{value}, \emph{original\_type=None}, \emph{name=None}}{}
Bases: {\hyperref[\detokenize{api:nanostream.utils.data_structures.MYSQL_VARCHAR_BASE}]{\sphinxcrossref{\sphinxcode{\sphinxupquote{nanostream.utils.data\_structures.MYSQL\_VARCHAR\_BASE}}}}}
\index{max\_length (nanostream.utils.data\_structures.MYSQL\_VARCHAR32 attribute)}

\begin{fulllineitems}
\phantomsection\label{\detokenize{api:nanostream.utils.data_structures.MYSQL_VARCHAR32.max_length}}\pysigline{\sphinxbfcode{\sphinxupquote{max\_length}}\sphinxbfcode{\sphinxupquote{ = 32}}}
\end{fulllineitems}


\end{fulllineitems}

\index{MYSQL\_VARCHAR32768 (class in nanostream.utils.data\_structures)}

\begin{fulllineitems}
\phantomsection\label{\detokenize{api:nanostream.utils.data_structures.MYSQL_VARCHAR32768}}\pysiglinewithargsret{\sphinxbfcode{\sphinxupquote{class }}\sphinxcode{\sphinxupquote{nanostream.utils.data\_structures.}}\sphinxbfcode{\sphinxupquote{MYSQL\_VARCHAR32768}}}{\emph{value}, \emph{original\_type=None}, \emph{name=None}}{}
Bases: {\hyperref[\detokenize{api:nanostream.utils.data_structures.MYSQL_VARCHAR_BASE}]{\sphinxcrossref{\sphinxcode{\sphinxupquote{nanostream.utils.data\_structures.MYSQL\_VARCHAR\_BASE}}}}}
\index{max\_length (nanostream.utils.data\_structures.MYSQL\_VARCHAR32768 attribute)}

\begin{fulllineitems}
\phantomsection\label{\detokenize{api:nanostream.utils.data_structures.MYSQL_VARCHAR32768.max_length}}\pysigline{\sphinxbfcode{\sphinxupquote{max\_length}}\sphinxbfcode{\sphinxupquote{ = 32768}}}
\end{fulllineitems}


\end{fulllineitems}

\index{MYSQL\_VARCHAR4 (class in nanostream.utils.data\_structures)}

\begin{fulllineitems}
\phantomsection\label{\detokenize{api:nanostream.utils.data_structures.MYSQL_VARCHAR4}}\pysiglinewithargsret{\sphinxbfcode{\sphinxupquote{class }}\sphinxcode{\sphinxupquote{nanostream.utils.data\_structures.}}\sphinxbfcode{\sphinxupquote{MYSQL\_VARCHAR4}}}{\emph{value}, \emph{original\_type=None}, \emph{name=None}}{}
Bases: {\hyperref[\detokenize{api:nanostream.utils.data_structures.MYSQL_VARCHAR_BASE}]{\sphinxcrossref{\sphinxcode{\sphinxupquote{nanostream.utils.data\_structures.MYSQL\_VARCHAR\_BASE}}}}}
\index{max\_length (nanostream.utils.data\_structures.MYSQL\_VARCHAR4 attribute)}

\begin{fulllineitems}
\phantomsection\label{\detokenize{api:nanostream.utils.data_structures.MYSQL_VARCHAR4.max_length}}\pysigline{\sphinxbfcode{\sphinxupquote{max\_length}}\sphinxbfcode{\sphinxupquote{ = 4}}}
\end{fulllineitems}


\end{fulllineitems}

\index{MYSQL\_VARCHAR4096 (class in nanostream.utils.data\_structures)}

\begin{fulllineitems}
\phantomsection\label{\detokenize{api:nanostream.utils.data_structures.MYSQL_VARCHAR4096}}\pysiglinewithargsret{\sphinxbfcode{\sphinxupquote{class }}\sphinxcode{\sphinxupquote{nanostream.utils.data\_structures.}}\sphinxbfcode{\sphinxupquote{MYSQL\_VARCHAR4096}}}{\emph{value}, \emph{original\_type=None}, \emph{name=None}}{}
Bases: {\hyperref[\detokenize{api:nanostream.utils.data_structures.MYSQL_VARCHAR_BASE}]{\sphinxcrossref{\sphinxcode{\sphinxupquote{nanostream.utils.data\_structures.MYSQL\_VARCHAR\_BASE}}}}}
\index{max\_length (nanostream.utils.data\_structures.MYSQL\_VARCHAR4096 attribute)}

\begin{fulllineitems}
\phantomsection\label{\detokenize{api:nanostream.utils.data_structures.MYSQL_VARCHAR4096.max_length}}\pysigline{\sphinxbfcode{\sphinxupquote{max\_length}}\sphinxbfcode{\sphinxupquote{ = 4096}}}
\end{fulllineitems}


\end{fulllineitems}

\index{MYSQL\_VARCHAR5 (class in nanostream.utils.data\_structures)}

\begin{fulllineitems}
\phantomsection\label{\detokenize{api:nanostream.utils.data_structures.MYSQL_VARCHAR5}}\pysiglinewithargsret{\sphinxbfcode{\sphinxupquote{class }}\sphinxcode{\sphinxupquote{nanostream.utils.data\_structures.}}\sphinxbfcode{\sphinxupquote{MYSQL\_VARCHAR5}}}{\emph{value}, \emph{original\_type=None}, \emph{name=None}}{}
Bases: {\hyperref[\detokenize{api:nanostream.utils.data_structures.MYSQL_VARCHAR_BASE}]{\sphinxcrossref{\sphinxcode{\sphinxupquote{nanostream.utils.data\_structures.MYSQL\_VARCHAR\_BASE}}}}}
\index{max\_length (nanostream.utils.data\_structures.MYSQL\_VARCHAR5 attribute)}

\begin{fulllineitems}
\phantomsection\label{\detokenize{api:nanostream.utils.data_structures.MYSQL_VARCHAR5.max_length}}\pysigline{\sphinxbfcode{\sphinxupquote{max\_length}}\sphinxbfcode{\sphinxupquote{ = 5}}}
\end{fulllineitems}


\end{fulllineitems}

\index{MYSQL\_VARCHAR512 (class in nanostream.utils.data\_structures)}

\begin{fulllineitems}
\phantomsection\label{\detokenize{api:nanostream.utils.data_structures.MYSQL_VARCHAR512}}\pysiglinewithargsret{\sphinxbfcode{\sphinxupquote{class }}\sphinxcode{\sphinxupquote{nanostream.utils.data\_structures.}}\sphinxbfcode{\sphinxupquote{MYSQL\_VARCHAR512}}}{\emph{value}, \emph{original\_type=None}, \emph{name=None}}{}
Bases: {\hyperref[\detokenize{api:nanostream.utils.data_structures.MYSQL_VARCHAR_BASE}]{\sphinxcrossref{\sphinxcode{\sphinxupquote{nanostream.utils.data\_structures.MYSQL\_VARCHAR\_BASE}}}}}
\index{max\_length (nanostream.utils.data\_structures.MYSQL\_VARCHAR512 attribute)}

\begin{fulllineitems}
\phantomsection\label{\detokenize{api:nanostream.utils.data_structures.MYSQL_VARCHAR512.max_length}}\pysigline{\sphinxbfcode{\sphinxupquote{max\_length}}\sphinxbfcode{\sphinxupquote{ = 512}}}
\end{fulllineitems}


\end{fulllineitems}

\index{MYSQL\_VARCHAR6 (class in nanostream.utils.data\_structures)}

\begin{fulllineitems}
\phantomsection\label{\detokenize{api:nanostream.utils.data_structures.MYSQL_VARCHAR6}}\pysiglinewithargsret{\sphinxbfcode{\sphinxupquote{class }}\sphinxcode{\sphinxupquote{nanostream.utils.data\_structures.}}\sphinxbfcode{\sphinxupquote{MYSQL\_VARCHAR6}}}{\emph{value}, \emph{original\_type=None}, \emph{name=None}}{}
Bases: {\hyperref[\detokenize{api:nanostream.utils.data_structures.MYSQL_VARCHAR_BASE}]{\sphinxcrossref{\sphinxcode{\sphinxupquote{nanostream.utils.data\_structures.MYSQL\_VARCHAR\_BASE}}}}}
\index{max\_length (nanostream.utils.data\_structures.MYSQL\_VARCHAR6 attribute)}

\begin{fulllineitems}
\phantomsection\label{\detokenize{api:nanostream.utils.data_structures.MYSQL_VARCHAR6.max_length}}\pysigline{\sphinxbfcode{\sphinxupquote{max\_length}}\sphinxbfcode{\sphinxupquote{ = 6}}}
\end{fulllineitems}


\end{fulllineitems}

\index{MYSQL\_VARCHAR64 (class in nanostream.utils.data\_structures)}

\begin{fulllineitems}
\phantomsection\label{\detokenize{api:nanostream.utils.data_structures.MYSQL_VARCHAR64}}\pysiglinewithargsret{\sphinxbfcode{\sphinxupquote{class }}\sphinxcode{\sphinxupquote{nanostream.utils.data\_structures.}}\sphinxbfcode{\sphinxupquote{MYSQL\_VARCHAR64}}}{\emph{value}, \emph{original\_type=None}, \emph{name=None}}{}
Bases: {\hyperref[\detokenize{api:nanostream.utils.data_structures.MYSQL_VARCHAR_BASE}]{\sphinxcrossref{\sphinxcode{\sphinxupquote{nanostream.utils.data\_structures.MYSQL\_VARCHAR\_BASE}}}}}
\index{max\_length (nanostream.utils.data\_structures.MYSQL\_VARCHAR64 attribute)}

\begin{fulllineitems}
\phantomsection\label{\detokenize{api:nanostream.utils.data_structures.MYSQL_VARCHAR64.max_length}}\pysigline{\sphinxbfcode{\sphinxupquote{max\_length}}\sphinxbfcode{\sphinxupquote{ = 64}}}
\end{fulllineitems}


\end{fulllineitems}

\index{MYSQL\_VARCHAR7 (class in nanostream.utils.data\_structures)}

\begin{fulllineitems}
\phantomsection\label{\detokenize{api:nanostream.utils.data_structures.MYSQL_VARCHAR7}}\pysiglinewithargsret{\sphinxbfcode{\sphinxupquote{class }}\sphinxcode{\sphinxupquote{nanostream.utils.data\_structures.}}\sphinxbfcode{\sphinxupquote{MYSQL\_VARCHAR7}}}{\emph{value}, \emph{original\_type=None}, \emph{name=None}}{}
Bases: {\hyperref[\detokenize{api:nanostream.utils.data_structures.MYSQL_VARCHAR_BASE}]{\sphinxcrossref{\sphinxcode{\sphinxupquote{nanostream.utils.data\_structures.MYSQL\_VARCHAR\_BASE}}}}}
\index{max\_length (nanostream.utils.data\_structures.MYSQL\_VARCHAR7 attribute)}

\begin{fulllineitems}
\phantomsection\label{\detokenize{api:nanostream.utils.data_structures.MYSQL_VARCHAR7.max_length}}\pysigline{\sphinxbfcode{\sphinxupquote{max\_length}}\sphinxbfcode{\sphinxupquote{ = 7}}}
\end{fulllineitems}


\end{fulllineitems}

\index{MYSQL\_VARCHAR8 (class in nanostream.utils.data\_structures)}

\begin{fulllineitems}
\phantomsection\label{\detokenize{api:nanostream.utils.data_structures.MYSQL_VARCHAR8}}\pysiglinewithargsret{\sphinxbfcode{\sphinxupquote{class }}\sphinxcode{\sphinxupquote{nanostream.utils.data\_structures.}}\sphinxbfcode{\sphinxupquote{MYSQL\_VARCHAR8}}}{\emph{value}, \emph{original\_type=None}, \emph{name=None}}{}
Bases: {\hyperref[\detokenize{api:nanostream.utils.data_structures.MYSQL_VARCHAR_BASE}]{\sphinxcrossref{\sphinxcode{\sphinxupquote{nanostream.utils.data\_structures.MYSQL\_VARCHAR\_BASE}}}}}
\index{max\_length (nanostream.utils.data\_structures.MYSQL\_VARCHAR8 attribute)}

\begin{fulllineitems}
\phantomsection\label{\detokenize{api:nanostream.utils.data_structures.MYSQL_VARCHAR8.max_length}}\pysigline{\sphinxbfcode{\sphinxupquote{max\_length}}\sphinxbfcode{\sphinxupquote{ = 8}}}
\end{fulllineitems}


\end{fulllineitems}

\index{MYSQL\_VARCHAR8192 (class in nanostream.utils.data\_structures)}

\begin{fulllineitems}
\phantomsection\label{\detokenize{api:nanostream.utils.data_structures.MYSQL_VARCHAR8192}}\pysiglinewithargsret{\sphinxbfcode{\sphinxupquote{class }}\sphinxcode{\sphinxupquote{nanostream.utils.data\_structures.}}\sphinxbfcode{\sphinxupquote{MYSQL\_VARCHAR8192}}}{\emph{value}, \emph{original\_type=None}, \emph{name=None}}{}
Bases: {\hyperref[\detokenize{api:nanostream.utils.data_structures.MYSQL_VARCHAR_BASE}]{\sphinxcrossref{\sphinxcode{\sphinxupquote{nanostream.utils.data\_structures.MYSQL\_VARCHAR\_BASE}}}}}
\index{max\_length (nanostream.utils.data\_structures.MYSQL\_VARCHAR8192 attribute)}

\begin{fulllineitems}
\phantomsection\label{\detokenize{api:nanostream.utils.data_structures.MYSQL_VARCHAR8192.max_length}}\pysigline{\sphinxbfcode{\sphinxupquote{max\_length}}\sphinxbfcode{\sphinxupquote{ = 8192}}}
\end{fulllineitems}


\end{fulllineitems}

\index{MYSQL\_VARCHAR9 (class in nanostream.utils.data\_structures)}

\begin{fulllineitems}
\phantomsection\label{\detokenize{api:nanostream.utils.data_structures.MYSQL_VARCHAR9}}\pysiglinewithargsret{\sphinxbfcode{\sphinxupquote{class }}\sphinxcode{\sphinxupquote{nanostream.utils.data\_structures.}}\sphinxbfcode{\sphinxupquote{MYSQL\_VARCHAR9}}}{\emph{value}, \emph{original\_type=None}, \emph{name=None}}{}
Bases: {\hyperref[\detokenize{api:nanostream.utils.data_structures.MYSQL_VARCHAR_BASE}]{\sphinxcrossref{\sphinxcode{\sphinxupquote{nanostream.utils.data\_structures.MYSQL\_VARCHAR\_BASE}}}}}
\index{max\_length (nanostream.utils.data\_structures.MYSQL\_VARCHAR9 attribute)}

\begin{fulllineitems}
\phantomsection\label{\detokenize{api:nanostream.utils.data_structures.MYSQL_VARCHAR9.max_length}}\pysigline{\sphinxbfcode{\sphinxupquote{max\_length}}\sphinxbfcode{\sphinxupquote{ = 9}}}
\end{fulllineitems}


\end{fulllineitems}

\index{MYSQL\_VARCHAR\_BASE (class in nanostream.utils.data\_structures)}

\begin{fulllineitems}
\phantomsection\label{\detokenize{api:nanostream.utils.data_structures.MYSQL_VARCHAR_BASE}}\pysiglinewithargsret{\sphinxbfcode{\sphinxupquote{class }}\sphinxcode{\sphinxupquote{nanostream.utils.data\_structures.}}\sphinxbfcode{\sphinxupquote{MYSQL\_VARCHAR\_BASE}}}{\emph{value}, \emph{original\_type=None}, \emph{name=None}}{}
Bases: {\hyperref[\detokenize{api:nanostream.utils.data_structures.DataType}]{\sphinxcrossref{\sphinxcode{\sphinxupquote{nanostream.utils.data\_structures.DataType}}}}}, {\hyperref[\detokenize{api:nanostream.utils.data_structures.MySQLTypeSystem}]{\sphinxcrossref{\sphinxcode{\sphinxupquote{nanostream.utils.data\_structures.MySQLTypeSystem}}}}}
\index{intermediate\_type (nanostream.utils.data\_structures.MYSQL\_VARCHAR\_BASE attribute)}

\begin{fulllineitems}
\phantomsection\label{\detokenize{api:nanostream.utils.data_structures.MYSQL_VARCHAR_BASE.intermediate_type}}\pysigline{\sphinxbfcode{\sphinxupquote{intermediate\_type}}}
alias of {\hyperref[\detokenize{api:nanostream.utils.data_structures.STRING}]{\sphinxcrossref{\sphinxcode{\sphinxupquote{STRING}}}}}

\end{fulllineitems}

\index{python\_cast\_function (nanostream.utils.data\_structures.MYSQL\_VARCHAR\_BASE attribute)}

\begin{fulllineitems}
\phantomsection\label{\detokenize{api:nanostream.utils.data_structures.MYSQL_VARCHAR_BASE.python_cast_function}}\pysigline{\sphinxbfcode{\sphinxupquote{python\_cast\_function}}}
alias of \sphinxcode{\sphinxupquote{builtins.str}}

\end{fulllineitems}


\end{fulllineitems}

\index{MySQLTypeSystem (class in nanostream.utils.data\_structures)}

\begin{fulllineitems}
\phantomsection\label{\detokenize{api:nanostream.utils.data_structures.MySQLTypeSystem}}\pysigline{\sphinxbfcode{\sphinxupquote{class }}\sphinxcode{\sphinxupquote{nanostream.utils.data\_structures.}}\sphinxbfcode{\sphinxupquote{MySQLTypeSystem}}}
Bases: {\hyperref[\detokenize{api:nanostream.utils.data_structures.DataSourceTypeSystem}]{\sphinxcrossref{\sphinxcode{\sphinxupquote{nanostream.utils.data\_structures.DataSourceTypeSystem}}}}}

Each \sphinxcode{\sphinxupquote{TypeSystem}} gets a \sphinxcode{\sphinxupquote{type\_mapping}} static method that takes a
string and returns the class in the type system named by that string.
For example, \sphinxcode{\sphinxupquote{int(8)}} in a MySQL schema should return the
\sphinxcode{\sphinxupquote{MYSQL\_INTEGER8}} class.
\index{type\_mapping() (nanostream.utils.data\_structures.MySQLTypeSystem static method)}

\begin{fulllineitems}
\phantomsection\label{\detokenize{api:nanostream.utils.data_structures.MySQLTypeSystem.type_mapping}}\pysiglinewithargsret{\sphinxbfcode{\sphinxupquote{static }}\sphinxbfcode{\sphinxupquote{type\_mapping}}}{\emph{string}}{}
Parses the schema strings from MySQL and returns the appropriate class.

\end{fulllineitems}


\end{fulllineitems}

\index{PrimitiveTypeSystem (class in nanostream.utils.data\_structures)}

\begin{fulllineitems}
\phantomsection\label{\detokenize{api:nanostream.utils.data_structures.PrimitiveTypeSystem}}\pysigline{\sphinxbfcode{\sphinxupquote{class }}\sphinxcode{\sphinxupquote{nanostream.utils.data\_structures.}}\sphinxbfcode{\sphinxupquote{PrimitiveTypeSystem}}}
Bases: {\hyperref[\detokenize{api:nanostream.utils.data_structures.DataSourceTypeSystem}]{\sphinxcrossref{\sphinxcode{\sphinxupquote{nanostream.utils.data\_structures.DataSourceTypeSystem}}}}}

\end{fulllineitems}

\index{PythonTypeSystem (class in nanostream.utils.data\_structures)}

\begin{fulllineitems}
\phantomsection\label{\detokenize{api:nanostream.utils.data_structures.PythonTypeSystem}}\pysigline{\sphinxbfcode{\sphinxupquote{class }}\sphinxcode{\sphinxupquote{nanostream.utils.data\_structures.}}\sphinxbfcode{\sphinxupquote{PythonTypeSystem}}}
Bases: {\hyperref[\detokenize{api:nanostream.utils.data_structures.DataSourceTypeSystem}]{\sphinxcrossref{\sphinxcode{\sphinxupquote{nanostream.utils.data\_structures.DataSourceTypeSystem}}}}}

\end{fulllineitems}

\index{Row (class in nanostream.utils.data\_structures)}

\begin{fulllineitems}
\phantomsection\label{\detokenize{api:nanostream.utils.data_structures.Row}}\pysiglinewithargsret{\sphinxbfcode{\sphinxupquote{class }}\sphinxcode{\sphinxupquote{nanostream.utils.data\_structures.}}\sphinxbfcode{\sphinxupquote{Row}}}{\emph{*records}, \emph{type\_system=None}}{}
Bases: \sphinxcode{\sphinxupquote{object}}

A collection of \sphinxcode{\sphinxupquote{DataType}} objects (typed values). They are dictionaries
mapping the names of the values to the \sphinxcode{\sphinxupquote{DataType}} objects.
\index{concat() (nanostream.utils.data\_structures.Row method)}

\begin{fulllineitems}
\phantomsection\label{\detokenize{api:nanostream.utils.data_structures.Row.concat}}\pysiglinewithargsret{\sphinxbfcode{\sphinxupquote{concat}}}{\emph{other}, \emph{fail\_on\_duplicate=True}}{}
\end{fulllineitems}

\index{from\_dict() (nanostream.utils.data\_structures.Row static method)}

\begin{fulllineitems}
\phantomsection\label{\detokenize{api:nanostream.utils.data_structures.Row.from_dict}}\pysiglinewithargsret{\sphinxbfcode{\sphinxupquote{static }}\sphinxbfcode{\sphinxupquote{from\_dict}}}{\emph{row\_dictionary}, \emph{**kwargs}}{}
Creates a \sphinxcode{\sphinxupquote{Row}} object form a dictionary mapping names to values.

\end{fulllineitems}

\index{is\_empty() (nanostream.utils.data\_structures.Row method)}

\begin{fulllineitems}
\phantomsection\label{\detokenize{api:nanostream.utils.data_structures.Row.is_empty}}\pysiglinewithargsret{\sphinxbfcode{\sphinxupquote{is\_empty}}}{}{}
\end{fulllineitems}

\index{keys() (nanostream.utils.data\_structures.Row method)}

\begin{fulllineitems}
\phantomsection\label{\detokenize{api:nanostream.utils.data_structures.Row.keys}}\pysiglinewithargsret{\sphinxbfcode{\sphinxupquote{keys}}}{}{}
For implementing the mapping protocol.

\end{fulllineitems}


\end{fulllineitems}

\index{STRING (class in nanostream.utils.data\_structures)}

\begin{fulllineitems}
\phantomsection\label{\detokenize{api:nanostream.utils.data_structures.STRING}}\pysiglinewithargsret{\sphinxbfcode{\sphinxupquote{class }}\sphinxcode{\sphinxupquote{nanostream.utils.data\_structures.}}\sphinxbfcode{\sphinxupquote{STRING}}}{\emph{value}, \emph{original\_type=None}, \emph{name=None}}{}
Bases: {\hyperref[\detokenize{api:nanostream.utils.data_structures.DataType}]{\sphinxcrossref{\sphinxcode{\sphinxupquote{nanostream.utils.data\_structures.DataType}}}}}, {\hyperref[\detokenize{api:nanostream.utils.data_structures.IntermediateTypeSystem}]{\sphinxcrossref{\sphinxcode{\sphinxupquote{nanostream.utils.data\_structures.IntermediateTypeSystem}}}}}
\index{python\_cast\_function (nanostream.utils.data\_structures.STRING attribute)}

\begin{fulllineitems}
\phantomsection\label{\detokenize{api:nanostream.utils.data_structures.STRING.python_cast_function}}\pysigline{\sphinxbfcode{\sphinxupquote{python\_cast\_function}}}
alias of \sphinxcode{\sphinxupquote{builtins.str}}

\end{fulllineitems}


\end{fulllineitems}

\index{all\_bases() (in module nanostream.utils.data\_structures)}

\begin{fulllineitems}
\phantomsection\label{\detokenize{api:nanostream.utils.data_structures.all_bases}}\pysiglinewithargsret{\sphinxcode{\sphinxupquote{nanostream.utils.data\_structures.}}\sphinxbfcode{\sphinxupquote{all\_bases}}}{\emph{obj}}{}
Return all the class to which \sphinxcode{\sphinxupquote{obj}} belongs.

\end{fulllineitems}

\index{convert\_to\_type\_system() (in module nanostream.utils.data\_structures)}

\begin{fulllineitems}
\phantomsection\label{\detokenize{api:nanostream.utils.data_structures.convert_to_type_system}}\pysiglinewithargsret{\sphinxcode{\sphinxupquote{nanostream.utils.data\_structures.}}\sphinxbfcode{\sphinxupquote{convert\_to\_type\_system}}}{\emph{obj}, \emph{cls}}{}
\end{fulllineitems}

\index{get\_type\_system() (in module nanostream.utils.data\_structures)}

\begin{fulllineitems}
\phantomsection\label{\detokenize{api:nanostream.utils.data_structures.get_type_system}}\pysiglinewithargsret{\sphinxcode{\sphinxupquote{nanostream.utils.data\_structures.}}\sphinxbfcode{\sphinxupquote{get\_type\_system}}}{\emph{obj}}{}
\end{fulllineitems}

\index{make\_types() (in module nanostream.utils.data\_structures)}

\begin{fulllineitems}
\phantomsection\label{\detokenize{api:nanostream.utils.data_structures.make_types}}\pysiglinewithargsret{\sphinxcode{\sphinxupquote{nanostream.utils.data\_structures.}}\sphinxbfcode{\sphinxupquote{make\_types}}}{}{}
\end{fulllineitems}

\index{mysql\_type() (in module nanostream.utils.data\_structures)}

\begin{fulllineitems}
\phantomsection\label{\detokenize{api:nanostream.utils.data_structures.mysql_type}}\pysiglinewithargsret{\sphinxcode{\sphinxupquote{nanostream.utils.data\_structures.}}\sphinxbfcode{\sphinxupquote{mysql\_type}}}{\emph{string}}{}
Parses the schema strings from MySQL and returns the appropriate class.

\end{fulllineitems}

\index{primitive\_to\_intermediate\_type() (in module nanostream.utils.data\_structures)}

\begin{fulllineitems}
\phantomsection\label{\detokenize{api:nanostream.utils.data_structures.primitive_to_intermediate_type}}\pysiglinewithargsret{\sphinxcode{\sphinxupquote{nanostream.utils.data\_structures.}}\sphinxbfcode{\sphinxupquote{primitive\_to\_intermediate\_type}}}{\emph{thing}, \emph{name=None}}{}
\end{fulllineitems}

\phantomsection\label{\detokenize{api:module-nanostream.node_classes.network_nodes}}\index{nanostream.node\_classes.network\_nodes (module)}

\section{Network nodes module}
\label{\detokenize{api:network-nodes-module}}
Classes that deal with sending and receiving data across the interwebs.
\index{HttpGetRequest (class in nanostream.node\_classes.network\_nodes)}

\begin{fulllineitems}
\phantomsection\label{\detokenize{api:nanostream.node_classes.network_nodes.HttpGetRequest}}\pysiglinewithargsret{\sphinxbfcode{\sphinxupquote{class }}\sphinxcode{\sphinxupquote{nanostream.node\_classes.network\_nodes.}}\sphinxbfcode{\sphinxupquote{HttpGetRequest}}}{\emph{endpoint\_template=None}, \emph{endpoint\_dict=None}, \emph{protocol='http'}, \emph{retries=5}, \emph{json=True}, \emph{**kwargs}}{}
Bases: {\hyperref[\detokenize{api:nanostream.node.NanoNode}]{\sphinxcrossref{\sphinxcode{\sphinxupquote{nanostream.node.NanoNode}}}}}

Node class for making simple GET requests.
\index{process\_item() (nanostream.node\_classes.network\_nodes.HttpGetRequest method)}

\begin{fulllineitems}
\phantomsection\label{\detokenize{api:nanostream.node_classes.network_nodes.HttpGetRequest.process_item}}\pysiglinewithargsret{\sphinxbfcode{\sphinxupquote{process\_item}}}{}{}
The input to this function will be a dictionary-like object with
parameters to be substituted into the endpoint string and a
dictionary with keys and values to be passed in the GET request.

Three use-cases:
1. Endpoint and parameters set initially and never changed.
2. Endpoint and parameters set once at runtime
3. Endpoint and parameters set by upstream messages

\end{fulllineitems}


\end{fulllineitems}

\index{HttpGetRequestPaginator (class in nanostream.node\_classes.network\_nodes)}

\begin{fulllineitems}
\phantomsection\label{\detokenize{api:nanostream.node_classes.network_nodes.HttpGetRequestPaginator}}\pysiglinewithargsret{\sphinxbfcode{\sphinxupquote{class }}\sphinxcode{\sphinxupquote{nanostream.node\_classes.network\_nodes.}}\sphinxbfcode{\sphinxupquote{HttpGetRequestPaginator}}}{\emph{endpoint\_dict=None}, \emph{json=True}, \emph{pagination\_get\_request\_key=None}, \emph{endpoint\_template=None}, \emph{additional\_data\_key=None}, \emph{pagination\_key=None}, \emph{pagination\_template\_key=None}, \emph{default\_offset\_value=''}, \emph{**kwargs}}{}
Bases: {\hyperref[\detokenize{api:nanostream.node.NanoNode}]{\sphinxcrossref{\sphinxcode{\sphinxupquote{nanostream.node.NanoNode}}}}}

Node class for HTTP API requests that require paging through sets of
results.
\index{process\_item() (nanostream.node\_classes.network\_nodes.HttpGetRequestPaginator method)}

\begin{fulllineitems}
\phantomsection\label{\detokenize{api:nanostream.node_classes.network_nodes.HttpGetRequestPaginator.process_item}}\pysiglinewithargsret{\sphinxbfcode{\sphinxupquote{process\_item}}}{}{}
Default no-op for nodes.

\end{fulllineitems}


\end{fulllineitems}

\index{PaginatedHttpGetRequest (class in nanostream.node\_classes.network\_nodes)}

\begin{fulllineitems}
\phantomsection\label{\detokenize{api:nanostream.node_classes.network_nodes.PaginatedHttpGetRequest}}\pysiglinewithargsret{\sphinxbfcode{\sphinxupquote{class }}\sphinxcode{\sphinxupquote{nanostream.node\_classes.network\_nodes.}}\sphinxbfcode{\sphinxupquote{PaginatedHttpGetRequest}}}{\emph{endpoint\_template=None}, \emph{additional\_data\_key=None}, \emph{pagination\_key=None}, \emph{pagination\_get\_request\_key=None}, \emph{protocol='http'}, \emph{retries=5}, \emph{default\_offset\_value=''}, \emph{additional\_data\_test=\textless{}class 'bool'\textgreater{}}}{}
Bases: \sphinxcode{\sphinxupquote{object}}

For handling requests in a semi-general way that require paging through
lists of results and repeatedly making GET requests.
\index{responses() (nanostream.node\_classes.network\_nodes.PaginatedHttpGetRequest method)}

\begin{fulllineitems}
\phantomsection\label{\detokenize{api:nanostream.node_classes.network_nodes.PaginatedHttpGetRequest.responses}}\pysiglinewithargsret{\sphinxbfcode{\sphinxupquote{responses}}}{}{}
Generator. Yields each response until empty.

\end{fulllineitems}


\end{fulllineitems}

\phantomsection\label{\detokenize{api:module-nanostream.message.message}}\index{nanostream.message.message (module)}

\section{NanoStreamMessage module}
\label{\detokenize{api:nanostreammessage-module}}
The \sphinxcode{\sphinxupquote{NanoStreamMesaage}} encapsulates the content of each piece of data,
along with some useful metadata.
\index{NanoStreamMessage (class in nanostream.message.message)}

\begin{fulllineitems}
\phantomsection\label{\detokenize{api:nanostream.message.message.NanoStreamMessage}}\pysiglinewithargsret{\sphinxbfcode{\sphinxupquote{class }}\sphinxcode{\sphinxupquote{nanostream.message.message.}}\sphinxbfcode{\sphinxupquote{NanoStreamMessage}}}{\emph{message\_content}}{}
Bases: \sphinxcode{\sphinxupquote{object}}

A class that contains the message payloads that are queued for
each \sphinxcode{\sphinxupquote{NanoStreamProcessor}}. It holds the messages and lots
of metadata used for logging, monitoring, etc.

\end{fulllineitems}

\phantomsection\label{\detokenize{api:module-nanostream.message.poison_pill}}\index{nanostream.message.poison\_pill (module)}

\section{PoisonPill module}
\label{\detokenize{api:poisonpill-module}}
A simple class that is sent in a message to signal that the node should be
terminated.
\index{PoisonPill (class in nanostream.message.poison\_pill)}

\begin{fulllineitems}
\phantomsection\label{\detokenize{api:nanostream.message.poison_pill.PoisonPill}}\pysigline{\sphinxbfcode{\sphinxupquote{class }}\sphinxcode{\sphinxupquote{nanostream.message.poison\_pill.}}\sphinxbfcode{\sphinxupquote{PoisonPill}}}
Bases: \sphinxcode{\sphinxupquote{object}}

\end{fulllineitems}

\phantomsection\label{\detokenize{api:module-nanostream.message.trigger}}\index{nanostream.message.trigger (module)}

\section{Trigger module}
\label{\detokenize{api:trigger-module}}
A simple class containing no data, which is intended merely as a trigger,
signaling that the downstream node should do something.
\index{Trigger (class in nanostream.message.trigger)}

\begin{fulllineitems}
\phantomsection\label{\detokenize{api:nanostream.message.trigger.Trigger}}\pysiglinewithargsret{\sphinxbfcode{\sphinxupquote{class }}\sphinxcode{\sphinxupquote{nanostream.message.trigger.}}\sphinxbfcode{\sphinxupquote{Trigger}}}{\emph{previous\_trigger\_time=None}, \emph{trigger\_name=None}}{}
Bases: \sphinxcode{\sphinxupquote{object}}

\end{fulllineitems}

\index{hello\_world() (in module nanostream.message.trigger)}

\begin{fulllineitems}
\phantomsection\label{\detokenize{api:nanostream.message.trigger.hello_world}}\pysiglinewithargsret{\sphinxcode{\sphinxupquote{nanostream.message.trigger.}}\sphinxbfcode{\sphinxupquote{hello\_world}}}{}{}
\end{fulllineitems}

\phantomsection\label{\detokenize{api:module-nanostream.message.batch}}\index{nanostream.message.batch (module)}

\section{Batch module}
\label{\detokenize{api:batch-module}}
We’ll use markers to delimit batches of things, such as serialized
files and that kind of thing.
\index{BatchEnd (class in nanostream.message.batch)}

\begin{fulllineitems}
\phantomsection\label{\detokenize{api:nanostream.message.batch.BatchEnd}}\pysiglinewithargsret{\sphinxbfcode{\sphinxupquote{class }}\sphinxcode{\sphinxupquote{nanostream.message.batch.}}\sphinxbfcode{\sphinxupquote{BatchEnd}}}{\emph{*args}, \emph{**kwargs}}{}
Bases: \sphinxcode{\sphinxupquote{object}}

\end{fulllineitems}

\index{BatchStart (class in nanostream.message.batch)}

\begin{fulllineitems}
\phantomsection\label{\detokenize{api:nanostream.message.batch.BatchStart}}\pysiglinewithargsret{\sphinxbfcode{\sphinxupquote{class }}\sphinxcode{\sphinxupquote{nanostream.message.batch.}}\sphinxbfcode{\sphinxupquote{BatchStart}}}{\emph{*args}, \emph{**kwargs}}{}
Bases: \sphinxcode{\sphinxupquote{object}}

\end{fulllineitems}

\phantomsection\label{\detokenize{api:module-nanostream.message.canary}}\index{nanostream.message.canary (module)}\index{Canary (class in nanostream.message.canary)}

\begin{fulllineitems}
\phantomsection\label{\detokenize{api:nanostream.message.canary.Canary}}\pysigline{\sphinxbfcode{\sphinxupquote{class }}\sphinxcode{\sphinxupquote{nanostream.message.canary.}}\sphinxbfcode{\sphinxupquote{Canary}}}
Bases: \sphinxcode{\sphinxupquote{object}}

\end{fulllineitems}

\phantomsection\label{\detokenize{api:module-nanostream.node_queue.queue}}\index{nanostream.node\_queue.queue (module)}

\section{NanoStreamQueue module}
\label{\detokenize{api:nanostreamqueue-module}}
These are queues that form the directed edges between nodes.
\index{NanoStreamQueue (class in nanostream.node\_queue.queue)}

\begin{fulllineitems}
\phantomsection\label{\detokenize{api:nanostream.node_queue.queue.NanoStreamQueue}}\pysiglinewithargsret{\sphinxbfcode{\sphinxupquote{class }}\sphinxcode{\sphinxupquote{nanostream.node\_queue.queue.}}\sphinxbfcode{\sphinxupquote{NanoStreamQueue}}}{\emph{max\_queue\_size}, \emph{name=None}}{}
Bases: \sphinxcode{\sphinxupquote{object}}
\index{approximately\_full() (nanostream.node\_queue.queue.NanoStreamQueue method)}

\begin{fulllineitems}
\phantomsection\label{\detokenize{api:nanostream.node_queue.queue.NanoStreamQueue.approximately_full}}\pysiglinewithargsret{\sphinxbfcode{\sphinxupquote{approximately\_full}}}{\emph{error=0.95}}{}
\end{fulllineitems}

\index{empty (nanostream.node\_queue.queue.NanoStreamQueue attribute)}

\begin{fulllineitems}
\phantomsection\label{\detokenize{api:nanostream.node_queue.queue.NanoStreamQueue.empty}}\pysigline{\sphinxbfcode{\sphinxupquote{empty}}}
\end{fulllineitems}

\index{get() (nanostream.node\_queue.queue.NanoStreamQueue method)}

\begin{fulllineitems}
\phantomsection\label{\detokenize{api:nanostream.node_queue.queue.NanoStreamQueue.get}}\pysiglinewithargsret{\sphinxbfcode{\sphinxupquote{get}}}{}{}
\end{fulllineitems}

\index{put() (nanostream.node\_queue.queue.NanoStreamQueue method)}

\begin{fulllineitems}
\phantomsection\label{\detokenize{api:nanostream.node_queue.queue.NanoStreamQueue.put}}\pysiglinewithargsret{\sphinxbfcode{\sphinxupquote{put}}}{\emph{message}, \emph{*args}, \emph{previous\_message=None}, \emph{**kwargs}}{}
Places a message on the output queues. If the message is \sphinxcode{\sphinxupquote{None}},
then the queue is skipped.

Messages are \sphinxcode{\sphinxupquote{NanoStreamMessage}} objects; the payload of the
message is message.message\_content.

\end{fulllineitems}

\index{size() (nanostream.node\_queue.queue.NanoStreamQueue method)}

\begin{fulllineitems}
\phantomsection\label{\detokenize{api:nanostream.node_queue.queue.NanoStreamQueue.size}}\pysiglinewithargsret{\sphinxbfcode{\sphinxupquote{size}}}{}{}
\end{fulllineitems}


\end{fulllineitems}



\chapter{License}
\label{\detokenize{license:license}}\label{\detokenize{license::doc}}
Copyright (C) 2016 Zachary Ernst
\sphinxhref{mailto:zac.ernst@gmail.com}{zac.ernst@gmail.com}

This program is free software: you can redistribute it and/or modify
it under the terms of the GNU General Public License as published by
the Free Software Foundation, either version 3 of the License, or
(at your option) any later version.

This program is distributed in the hope that it will be useful,
but WITHOUT ANY WARRANTY; without even the implied warranty of
MERCHANTABILITY or FITNESS FOR A PARTICULAR PURPOSE.  See the
GNU General Public License for more details.

You should have received a copy of the GNU General Public License
along with this program.  If not, see \textless{}\sphinxurl{http://www.gnu.org/licenses/}\textgreater{}.


\chapter{Indices and tables}
\label{\detokenize{index:indices-and-tables}}\begin{itemize}
\item {} 
\DUrole{xref,std,std-ref}{genindex}

\item {} 
\DUrole{xref,std,std-ref}{modindex}

\item {} 
\DUrole{xref,std,std-ref}{search}

\end{itemize}


\renewcommand{\indexname}{Python Module Index}
\begin{sphinxtheindex}
\let\bigletter\sphinxstyleindexlettergroup
\bigletter{n}
\item\relax\sphinxstyleindexentry{nanostream.civis\_nodes}\sphinxstyleindexpageref{api:\detokenize{module-nanostream.civis_nodes}}
\item\relax\sphinxstyleindexentry{nanostream.message.batch}\sphinxstyleindexpageref{api:\detokenize{module-nanostream.message.batch}}
\item\relax\sphinxstyleindexentry{nanostream.message.canary}\sphinxstyleindexpageref{api:\detokenize{module-nanostream.message.canary}}
\item\relax\sphinxstyleindexentry{nanostream.message.message}\sphinxstyleindexpageref{api:\detokenize{module-nanostream.message.message}}
\item\relax\sphinxstyleindexentry{nanostream.message.poison\_pill}\sphinxstyleindexpageref{api:\detokenize{module-nanostream.message.poison_pill}}
\item\relax\sphinxstyleindexentry{nanostream.message.trigger}\sphinxstyleindexpageref{api:\detokenize{module-nanostream.message.trigger}}
\item\relax\sphinxstyleindexentry{nanostream.node}\sphinxstyleindexpageref{api:\detokenize{module-nanostream.node}}
\item\relax\sphinxstyleindexentry{nanostream.node\_classes.network\_nodes}\sphinxstyleindexpageref{api:\detokenize{module-nanostream.node_classes.network_nodes}}
\item\relax\sphinxstyleindexentry{nanostream.node\_queue.queue}\sphinxstyleindexpageref{api:\detokenize{module-nanostream.node_queue.queue}}
\item\relax\sphinxstyleindexentry{nanostream.utils.data\_structures}\sphinxstyleindexpageref{api:\detokenize{module-nanostream.utils.data_structures}}
\end{sphinxtheindex}

\renewcommand{\indexname}{Index}
\printindex
\end{document}